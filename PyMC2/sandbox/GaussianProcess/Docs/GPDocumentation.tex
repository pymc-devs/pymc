\documentclass{manual}
\usepackage{epsfig}
\usepackage{pdfsync}
\usepackage{amsfonts}



\begin{document}

\title{The GaussianProcess package}
\author{Anand Patil}
\maketitle
\tableofcontents

\chapter{Introduction}\label{cha:introduction} % (fold)

Gaussian processes (GPs) are probability distributions for functions. They're useful if a functional form is unknown a priori.

GPs are not hard to understand at a conceptual level, but implementing them on a computer can require fairly involved linear algebra. This makes it hard for beginners to get started, and even when you have experience it's annoying.

This package provides Python classes that represent the components of the Gaussian process. They are meant to support many types of Gaussian process usage, from intuitive exploration to high-performance deployment in MCMC, with smooth transitions between.

The package also provides a class which produces Gaussian process-valued PyMC parameters and a PyMC sampling method to handle it. That means the Gaussian process-related objects can be directly incorporated into larger probability models.

You'll need to understand Python and a little bit of Bayesian statistics. Refs.

% chapter introduction (end)

\chapter{Tutorial}\label{cha:tutorial} 

\section{Preliminaries}\label{sec:preliminaries}
You need to know a bit about Python and numpy. Also a bit about Bayesian statistics and the normal distribution. Refs. All the code in the tutorial is in module/folder \texttt{examples}, which is distributed with this package.

\section{A first look at Gaussian processes}\label{sec:first_look} % (fold)

Gaussian processes are probability distributions for functions. The statement `$f$ has a Gaussian process distribution with mean $M$ and covariance $C$' is usually written as follows:
\begin{equation}
    f\sim\textup{GP}(M,C).
\end{equation}
Gaussian processes have two parameters, which are analogous to the parameters of the normal distribution:
\begin{itemize}
    \item $M$ is the mean function. Like the mean parameter of the normal distribution, $M$ gives the central tendency for $f$. In Bayesian statistics, $M$ is usually considered a prior guess for $f$. $M(x)$ gives the expectation of $f(x)$.
    \item $C$  is the covariance function. Its role in the distribution of $f$ is harder to understand than the mean function, but among other things it regulates:
    \begin{itemize}
        \item the amount by which $f$ may deviate from $M$
        \item the smoothness of $f$
        \item the wiggliness of $f$.
    \end{itemize}
    $C(x,y)$ gives the covariance of $f(x)$ and $f(y)$; $C(x,x)$ gives the covariance of $f(x)$.
\end{itemize}
Understanding covariance functions is essential for sensible application of Gaussian processes, but for the time being don't worry about them too much. Section \ref{sec:cov} is all about covariance functions.


\subsection{Instantiating a Gaussian process} 

It will be much easier to understand the role of the covariance function if we generate a Gaussian process to play around with. 

The following code will produce a covariance function:
\verbatiminput{../examples/cov.py}

\bigskip
The first argument, \texttt{eval\_fun}, gives the Python function from which the covariance function will be made, in this case \texttt{exp\_cov}. The arguments \texttt{pow, amp} and \texttt{scale} will be passed to \texttt{exp\_cov}. See The last line displays a filled contour plot of $C(x,y)$ on the mesh ($-1<x<1$,$-1<y<1$). Try playing around with the parameters of \texttt{exp\_cov} and see how $C$ looks. Try calling it.

The following code will produce a mean function $M$ which is associated with $C$:
\verbatiminput{../examples/mean.py}

\bigskip
As with \texttt{Covariance}, the first argument to \texttt{Mean}'s init method is a Python function, in this case \texttt{linfun}. The second argument, \texttt{C}, is the \texttt{Covariance} instance with which the mean function is associated. The arguments \texttt{m} and \texttt{b} will be passed to \texttt{linfun}. The last line plots $M(x)$ on $-1<x<1$, and as expected the plot is a straight line through the origin with slope 1. Try calling it.

Now let's generate some realizations (draws) from the Gaussian process defined by $M$ and $C$ and take a look at them.
\verbatiminput{../examples/realizations.py}

\bigskip
The dashdot black line in the middle is $M$, and the gray band is the $\pm 1$ standard deviation envelope generated by $C$. Each realization is a callable function, so you can see their values.

% section first_look (end)


\section{The role of the covariance function}\label{sec:cov} % (fold)

Various visualization methods of covariance, relationship between covariance's shape and differentiability of realizations.

% section cov_mean (end)


\section{Nonparametric regression: observing Gaussian processes}\label{sec:observing} % (fold)

The condition function, what it does. Some ecological examples.

% section observing (end)


\section{The array aspect}\label{sec:array} % (fold)

\begin{equation}
    \begin{array}{ll}
        x\sim\textup N(\mu,V) & \textup{$x$, $y$, $V$ are scalars}\\\\
        \vec x\sim\textup N(\vec \mu,C)& \textup{$\vec x$ and $\vec \mu$ are vectors, $C$ is a scalar}\\\\
        f\sim\textup{GP}(M, C) & \textup{$f$ and $M$ are functions of one variable, $C$ is a function of two variables}
    \end{array}
\end{equation}

The base mesh

% section array (end)


\section{Incorporating Gaussian processes in probability models with PyMC}\label{sec:PyMC} % (fold)

% section PyMC (end)


\chapter{API reference}\label{cha:reference} 

Doxygen-style class reference here.


\end{document}
