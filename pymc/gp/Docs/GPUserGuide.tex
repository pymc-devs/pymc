%!TEX TS-program = pdflatex
%!TEX TS-program = skim
%
%  PyMC User's Guide
%
%  Created by Chris Fonnesbeck on 2006-05-03.
%  Copyright (c) 2006 . All rights reserved.
%
\documentclass[]{manual}

% Use utf-8 encoding for foreign characters
%\usepackage[utf8]{inputenc}

% Setup for fullpage use
\usepackage{fullpage}
\usepackage{amsmath}
\usepackage{epsfig}
\usepackage{upquote} 

% \usepackage{pdfsync}

% Flexible citation syntax
\usepackage{natbib}
% Uncomment some of the following if you use the features
%

% Multipart figures
%\usepackage{subfigure}

% More symbols
\usepackage{amsmath}
\usepackage{amssymb}
% \usepackage{latexsym}

% Package for including code in the document
\usepackage{listings}

% Surround parts of graphics with box
%\usepackage{boxedminipage}

% This is now the recommended way for checking for PDFLaTeX:
\usepackage{ifpdf}

% Enable hyperlinks
% \usepackage[pdfpagemode=FullScreen,colorlinks=true,linkcolor=red]{hyperref}

% \ifpdf
% \usepackage[pdftex]{graphicx}
% \else
% \usepackage{graphicx}
% \fi

%%% EPYDOC STUFF %%%
\usepackage{underscore}
\usepackage[english]{babel}
% \usepackage{alltt, parskip, boxedminipage}
\usepackage{makeidx, multirow, longtable, amssymb}% tocbibind}
\usepackage{fullpage}
\usepackage[usenames]{color}
% \usepackage{ifthen}
\usepackage{ae}
\usepackage{aeguill}
\usepackage{shortvrb}
\usepackage{ucs}
\usepackage{tabularx}
\usepackage{color}
\definecolor{Red}{rgb}{0.5,0,0}
\definecolor{Blue}{rgb}{0,0,0.5}
\usepackage[pdfpagemode=FullScreen,colorlinks=true,linkcolor=Red,citecolor=Blue,urlcolor=Red]{hyperref}

% \usepackage{alltt, parskip, fancyhdr, boxedminipage}
% \usepackage{makeidx, multirow, longtable, tocbibind, amssymb}
% \usepackage{fullpage}
% % \usepackage[usenames]{color}
% \setlength{\headheight}{16pt}
% \setlength{\headsep}{24pt}
% \setlength{\topmargin}{-\headsep}
% \setlength{\parindent}{0ex}
% \setlength{\parskip}{2ex}
\setlength{\fboxrule}{0\fboxrule}
\setlength{\fboxsep}{16pt} 
\newlength{\BCL} % base class length, for base trees.
% \pagestyle{fancy}
% \renewcommand{\sectionmark}[1]{\markboth{#1}{}}
% \renewcommand{\subsectionmark}[1]{\markright{#1}}
\definecolor{py@keywordcolour}{rgb}{1,0.45882,0}
\definecolor{py@stringcolour}{rgb}{0,0.666666,0}
\definecolor{py@commentcolour}{rgb}{1,0,0}
\definecolor{py@ps1colour}{rgb}{0.60784,0,0}
\definecolor{py@ps2colour}{rgb}{0.60784,0,1}
\definecolor{py@inputcolour}{rgb}{0,0,0}
\definecolor{py@outputcolour}{rgb}{0,0,1}
\definecolor{py@exceptcolour}{rgb}{1,0,0}
\definecolor{py@defnamecolour}{rgb}{1,0.5,0.5}
\definecolor{py@builtincolour}{rgb}{0.58039,0,0.58039}
\definecolor{py@identifiercolour}{rgb}{0,0,0}
\definecolor{py@linenumcolour}{rgb}{0.4,0.4,0.4}
\definecolor{py@inputcolour}{rgb}{0,0,0}
% Prompt
\newcommand{\pysrcprompt}[1]{\textcolor{py@ps1colour}{\small\textbf{#1}}}
\newcommand{\pysrcmore}[1]{\textcolor{py@ps2colour}{\small\textbf{#1}}}
% Source code
\newcommand{\pysrckeyword}[1]{\textcolor{py@keywordcolour}{\small\textbf{#1}}}
\newcommand{\pysrcbuiltin}[1]{\textcolor{py@builtincolour}{\small\textbf{#1}}}
\newcommand{\pysrcstring}[1]{\textcolor{py@stringcolour}{\small\textbf{#1}}}
\newcommand{\pysrcdefname}[1]{\textcolor{py@defnamecolour}{\small\textbf{#1}}}
\newcommand{\pysrcother}[1]{\small\textbf{#1}}
% Comments
\newcommand{\pysrccomment}[1]{\textcolor{py@commentcolour}{\small\textbf{#1}}}
% Output
\newcommand{\pysrcoutput}[1]{\textcolor{py@outputcolour}{\small\textbf{#1}}}
% Exceptions
\newcommand{\pysrcexcept}[1]{\textcolor{py@exceptcolour}{\small\textbf{#1}}}
\newlength{\funcindent}
\newlength{\funcwidth}
\setlength{\funcindent}{1cm}
\setlength{\funcwidth}{\textwidth}
\addtolength{\funcwidth}{-2\funcindent}
\newlength{\varindent}
\newlength{\varnamewidth}
\newlength{\vardescrwidth}
\newlength{\varwidth}
\setlength{\varindent}{1cm}
\setlength{\varnamewidth}{.3\textwidth}
\setlength{\varwidth}{\textwidth}
\addtolength{\varwidth}{-4\tabcolsep}
\addtolength{\varwidth}{-3\arrayrulewidth}
\addtolength{\varwidth}{-2\varindent}
\setlength{\vardescrwidth}{\varwidth}
\addtolength{\vardescrwidth}{-\varnamewidth}
\newenvironment{Ventry}[1]%
 {\begin{list}{}{%
   \renewcommand{\makelabel}[1]{\texttt{##1:}\hfil}%
   \settowidth{\labelwidth}{\texttt{#1:}}%
   \setlength{\leftmargin}{\labelsep}%
   \addtolength{\leftmargin}{\labelwidth}}}%
 {\end{list}}
% \usepackage[utf8]{inputenc}
\definecolor{UrlColor}{rgb}{0,0.08,0.45}
% \usepackage[dvips, pagebackref, pdftitle={API Documentation}, pdfcreator={epydoc 3.0.1}, bookmarks=true, bookmarksopen=false, pdfpagemode=UseOutlines, colorlinks=true, linkcolor=black, anchorcolor=black, citecolor=black, filecolor=black, menucolor=black, pagecolor=black, urlcolor=UrlColor]{hyperref}
% %\makeindex
% \usepackage{ae}
% \usepackage{aeguill}
% \usepackage{shortvrb}
% \usepackage{ucs}
% \usepackage{tabularx}
\setlength{\extrarowheight}{2pt}
% \usepackage{amsmath}
% \usepackage{graphicx}
% \usepackage{ifthen}
% \usepackage[DIV12]{typearea}
% generated by Docutils <http://docutils.sourceforge.net/>
\newlength{\admonitionwidth}
\setlength{\admonitionwidth}{0.9\textwidth}
\newlength{\docinfowidth}
\setlength{\docinfowidth}{0.9\textwidth}
%\newlength{\locallinewidth}
\newcommand{\optionlistlabel}[1]{\bf #1 \hfill}
\newenvironment{optionlist}[1]
{\begin{list}{}
  {\setlength{\labelwidth}{#1}
   \setlength{\rightmargin}{1cm}
   \setlength{\leftmargin}{\rightmargin}
   \addtolength{\leftmargin}{\labelwidth}
   \addtolength{\leftmargin}{\labelsep}
   \renewcommand{\makelabel}{\optionlistlabel}}
}{\end{list}}
\newlength{\lineblockindentation}
\setlength{\lineblockindentation}{2.5em}
\newenvironment{lineblock}[1]
{\begin{list}{}
  {\setlength{\partopsep}{\parskip}
   \addtolength{\partopsep}{\baselineskip}
   \topsep0pt\itemsep0.15\baselineskip\parsep0pt
   \leftmargin#1}
 \raggedright}
{\end{list}}
% begin: floats for footnotes tweaking.
\setlength{\floatsep}{0.5em}
\setlength{\textfloatsep}{\fill}
\addtolength{\textfloatsep}{3em}
\renewcommand{\textfraction}{0.5}
\renewcommand{\topfraction}{0.5}
\renewcommand{\bottomfraction}{0.5}
\setcounter{totalnumber}{50}
\setcounter{topnumber}{50}
\setcounter{bottomnumber}{50}
% end floats for footnotes
% some commands, that could be overwritten in the style file.
%\newcommand{\rubric}[1]{\subsection*{~\hfill {\it #1} \hfill ~}}
%\newcommand{\titlereference}[1]{\textsl{#1}}
% end of "some commands"
%\ifthenelse{\isundefined{\hypersetup}}{}{}
%%% END OF EPYDOC STUFF %%%


\title{PyMC Gaussian process module \\User's guide}
\author{ Anand Patil }
\pdfoutput=1
% \date


%%%%%%%%%%%%%%% Commands from rst2latex %%%%%%%%%%%%%%%%%%%%%%%%
\newcommand{\rubric}[1]{\subsection*{~\hfill {\it #1} \hfill ~}}
\newcommand{\titlereference}[1]{\textsl{#1}}
\newlength{\locallinewidth}
\setlength{\locallinewidth}{7in}
%%%%%%%%%%%%%%%%%%%%%%%%%%%%%%%%%%%%%%%%%%%%%%%%%%%%%%%%%%%%%%%%%

\begin{document}

\maketitle

\tableofcontents

% \chapter{Introduction}\label{cha:introduction} % (fold)

% \section{Prerequisites}\label{sec:prerequisites}
% \begin{itemize}
%     \item Familiarity with \citetitle[www.python.org]{Python}. Several online tutorials can be found by clicking around the Python website, including \citetitle[www.ibiblio.org/obp/thinkCSpy/]{How to think like a computer scientist: Learning with Python} by Downey, Elkner and Meyers. More experienced programmers may prefer \citetitle[docs.python.org/tut/]{another tutorial} by Python's author. Python is generally regarded as having unusually good documentation for open-source software, and it is also considered one of the easiest general-purpose programming languages to learn.
%     \item Familiarity with Numerical Python, or \citetitle[www.scipy.org/numpy]{NumPy}. This package provides array and matrix objects and associated functions that are much faster than their equivalents in pure Python, and also more convenient for scientific and numerical work. If you're familiar with a matrix language such as Matlab or R, learning numpy shouldn't be too difficult. However, there are a few differences that can function as `gotchas' early on. Although numpy itself is available free of charge, its documentation costs \$40 and is available from \citetitle[www.scipy.org/Documentation]{the Scientific Python website}.
%     \item A conceptual understanding of Bayesian statistics. Some familiarity with the normal distribution would help. Gelman et al. \cite{gelman} and Berry \cite{berry} are good general references, as is \citetitle[http://www.ams.ucsc.edu/~draper/draper-BHM-2005.pdf]{Draper's book}. For a colorful online tutorial on the logical aspects of Bayesian statistics, see \citetitle[http://www.yudkowsky.net/bayes/bayes.html]{An Intuitive Explanation of Bayesian Reasoning}.
% \end{itemize}

% chapter introduction (end)

\chapter{The basics}\label{cha:basics}
Gaussian processes (GPs) are probability distributions for functions. In Bayesian statistics, they are often used as priors for functions whose forms are unknown because they can encode many types of knowledge about functions, yet remain much less restrictive than priors based on particular functional forms. GPs are not hard to understand at a conceptual level, but implementing them efficiently on a computer can require fairly involved linear algebra.

This package implements Gaussian processes as a set of Python classes that can support many types of usage, from intuitive exploration to embedding in larger probability models and fitting with MCMC.

All the code in the tutorial is in the folder \file{pymc/examples/gp} in the PyMC source tree.

\section{Creating a Gaussian process}\label{sub:inst}

This section demonstrates creation of a covariance function, a mean function, and finally several random functions drawn from the Gaussian process distribution defined by those objects.

\subsection{Creating a mean function}\label{subsub:mean}

\begin{figure}
    \centering
        \epsfig{file=figs/mean.pdf,width=8cm}
    \caption{The mean function generated by {\sffamily `examples/mean.py'}.}
    \label{fig:mean}
\end{figure}

The mean function of a univariate Gaussian process can be interpreted as a prior guess for the GP, so it is also a univariate function. Mean functions are represented by class \code{Mean}, which is a wrapper for an ordinary \proglang{Python} function. The following code (from \code{pymc/examples/gp/Mean.py}) will produce an instance of class \code{Mean} called $M$:
\begin{CodeChunk}
\begin{CodeInput}
from pymc.gp import *
def quadfun(x, a, b, c):
    return (a * x ** 2 + b * x + c)
M = Mean(quadfun, a = 1., b = .5, c = 2.)        
\end{CodeInput}
\end{CodeChunk}

The first argument of \code{Mean}'s init method is the underlying \proglang{Python} function, in this case \code{quadfun}. The extra arguments $a$, $b$  and $c$ will be memorized and passed to \code{quadfun} whenever $M$ is called; the call $M(x)$ in the plotting portion of the script does not need to pass them in.

Mean functions broadcast over their arguments in the same way as \href{http://docs.scipy.org/doc/numpy/reference/ufuncs.html}{\pkg{NumPy} universal functions} \citep{numpybook}, which means that the call $M(x)$, where $x$ is a vector, returns the vector
\begin{eqnarray*}
    [M(x_0),\ldots, M(x_{N-1})].
\end{eqnarray*}

The last part of the code plots $M(x)$ on $-1<x<1$, and its output is shown in figure \ref{fig:mean}. As expected, the plot is a parabola.

\subsection{Creating a covariance function}\label{subsub:cov}
\begin{figure}
    \centering
        \epsfig{file=figs/cov.pdf,width=12cm}
    \caption{The covariance function generated by {\sffamily `examples/cov.py'}. On the left is the covariance function $C(x,y)$ evaluated over a square: $-1\le x\le 1,\ -1\le y\le 1$. On the right is a slice of the covariance: $C(x,0)$ for $0\le x \le 1$}
    \label{fig:cov}
\end{figure}

Covariance functions are represented by the class \code{Covariance}, which like \code{Mean} is essentially a wrapper for ordinary \proglang{Python} functions. The example in \code{pymc/examples/gp/cov.py} uses the popular Mat\`ern function \citep{banerjee}, which is provided in module \code{cov_funs}. In addition to the two arguments $x$ and $y$, the Mat\`ern function takes three parameters: \code{amp} controls the amount by which realizations may deviate from their mean, \code{diff_degree} controls the roughness of realizations (the degree of differentiability), and \code{scale} controls the lengthscale over which realizations change.

The user is free to write functions to wrap in \code{Covariance} objects. See \href{http://code.google.com/p/pymc}{the package documentation} for more information.

The code in \code{pymc/examples/gp/cov.py} will produce an instance of class \code{Covariance} called $C$:
\begin{CodeChunk}
\begin{CodeInput}
from pymc.gp import *
from pymc.gp.cov_funs import matern

C = Covariance(eval_fun = matern.euclidean, diff_degree = 1.4, amp = .4, scale = 1.)
\end{CodeInput}
\end{CodeChunk}

The first argument to \code{Covariance}'s init method is the \proglang{Python} function from which the covariance function will be made. In this case, \code{eval_fun} is \code{matern.euclidean}. Covariance functions' calling conventions are slightly different from ordinary \pkg{NumPy} universal functions' \citep{numpybook} in two ways. First, broadcasting works differently. If $C$ were a \pkg{NumPy} universal function, $C(x,y)$ would return the following array:
    \begin{eqnarray*}
        \begin{array}{ccc}
            [C(x_0,y_0)& \ldots& C(x_{N-1},y_{N-1})],
        \end{array}
    \end{eqnarray*}
    where $x$ and $y$ would need to be vectors of the same length. In fact $C(x,y)$ returns a matrix:
    \begin{eqnarray*}
        \left[\begin{array}{ccc}
            C(x_0,y_0)& \ldots& C(x_0,y_{N_y-1})\\
            \vdots&\ddots&\vdots\\
            C(x_{N_x-1},y_0)& \ldots& C(x_{N_x-1},y_{N_y-1})
        \end{array}\right],
    \end{eqnarray*}
    and input arguments $x$ and $y$ don't need to be the same length. Second, covariance functions can be called with just one argument. $C(x)$ returns
    \begin{eqnarray*}
         [C(x_0,x_0)& \ldots& C(x_{N_x-1},x_{N_x-1})] = \textup{diag}(C(x,x)),
    \end{eqnarray*}
    but is computed much faster than diag$(C(x,x))$ would be.
The extra arguments \code{diff_degree, amp} and \code{scale}, which are required by \code{matern.euclidean}, will be passed to \code{matern.euclidean} by $C$ every time is called.
 
The output of \code{examples/cov.py} is shown in figure \ref{fig:cov}.

\subsubsection{Cholesky algorithms}

The numerical `heavy lifting' done by this package is primarily handled by \code{Covariance} and its subclasses. \texttt{Covariance} itself bases all its computations on the incomplete Cholesky decomposition algorithm used by the \proglang{Matlab} package \pkg{chol_incomplete} \citep{seeger}. \code{Covariance} computes rows of covariance matrices as they are needed, so if the function it wraps tends to produce covariance matrices with only a few large eigenvalues it can approximate the Cholesky decomposition in less than $O(n^2)$ arithmetic operations \citep{predictivechol}.

\code{Covariance} calls back to \proglang{Python} from \proglang{Fortran} every time it needs a new row. If the function it wraps tends to produce full-rank covariance matrices (for which all rows are required), this is inefficient. \code{FullRankCovariance} is a drop-in replacement for \code{Covariance} that is much faster, but fails (with a helpful error message) if it attempts to factor a matrix that is not full rank. \code{NearlyFullRankCovariance} provides a compromise between the two: it computes covariance matrices in full in \proglang{Fortran}, then factors them using the robust algorithm of \pkg{chol_incomplete}.

\subsection{Drawing realizations}\label{subsub:realizations}
\begin{figure}
    \centering
        \epsfig{file=figs/realizations.pdf,width=8cm}
    \caption{Three realizations from a Gaussian process displayed with mean $\pm$ 1 sd envelope. Generated by {\sffamily `examples/realizations.py'}.}
    \label{fig:realizations}
\end{figure}

The code in \texttt{pymc/examples/gp/realizations.py} generates a list of \code{Realization} objects, which represent realizations (draws) from the Gaussian process defined by $M$ and $C$:
\begin{CodeChunk}
\begin{CodeInput}
from mean import M
from cov import C
from pymc.gp import *

f_list = [Realization(M,C) for i in range(3)]
\end{CodeInput}
\end{CodeChunk}

The init method of \code{Realization} takes only two required arguments, a \code{Mean} object and a \code{Covariance} object. Each element of \code{f_list} is a Gaussian process realization, which is essentially a randomly-generated \proglang{Python} function. Like \code{Mean} objects, \code{Realization} objects use the same broadcasting rules as \pkg{NumPy} universal functions. The call $f(x)$ returns the vector
\begin{eqnarray*}
    [f(x_0)\ldots f(x_{N-1})].
\end{eqnarray*}

Each of the three realizations in \code{f_list} is plotted in figure \ref{fig:realizations}, superimposed on a $\pm$ 1 standard deviation envelope.


\section{Nonparametric regression: observing Gaussian processes}\label{sec:observing}

\begin{figure}
    \centering
        \epsfig{file=figs/obs.pdf,width=5cm}
        \epsfig{file=figs/cond.pdf,width=5cm}
    \caption{The output of {\sffamily `examples/observations.py'}: the observed GP with \code{obs_V = .002} (left) and \code{obs_V = 0} (right). Note that in the conditioned case, the $\pm$ 1 SD envelope shrinks to zero at the points where the observations were made, and all realizations pass through the observed values. Compare these plots to those in figure \ref{fig:realizations}.}
    \label{fig:obs}
\end{figure}

Consider the following common statistical situation: A Gaussian process prior for an unknown function $f$ is chosen, then the value of $f$ is observed at $N$ input points $[o_0\ldots o_{N-1}]$, possibly with uncertainty. If the observation error is normally distributed, it turns out that $f$'s posterior distribution given the new information is another Gaussian process, with new mean and covariance functions.

The probability model that represents this situation is as follows:
\begin{equation}
    \label{regprior}
    \left.\begin{array}{l}
        \textup{data}_i \stackrel{\tiny{\textup{ind}}}{\sim} \textup{N}(f(o_i), V_i)\\
        f \sim \textup{GP}(M,C)\\
    \end{array}\right\}\Rightarrow f|\textup{data} \sim \textup{GP}(M_o, C_o).
\end{equation}
Function \code{observe} imposes normally-distributed observations on Gaussian process distributions. This function converts $f$'s prior to its posterior by transforming $M$ and $C$ in equation \ref{regprior} to $M_o$ and $C_o$:

The code in \code{pymc/examples/gp/observation.py} imposes the observations
\begin{eqnarray*}
    f(-.5) = 3.1\\
    f(.5) = 2.9
\end{eqnarray*}
with observation variance $V=.002$ on the GP distribution defined in \code{mean.py} and \code{cov.py}:
\begin{CodeChunk}
\begin{CodeInput}
from mean import M
from cov import C
from pymc.gp import *
from numpy import *

obs_x = array([-.5,.5])
V = array([.002,.002])
data = array([3.1, 2.9])
observe(M=M, C=C, obs_mesh=obs_x, obs_V=V, obs_vals=data)

f_list = [Realization(M,C) for i in range(3)]
\end{CodeInput}
\end{CodeChunk}

The function \code{observe} takes a covariance $C$ and a mean $M$ as arguments, and tells them that their `true' realization's value on \code{obs_mesh} has been observed to be \code{obs_vals} with variance \code{obs_V}. 

The output of \code{observation.py}  is shown in figure \ref{fig:obs}, along with the output with \code{obs_V=0}. Compare these to the analogous figure for the unobserved GP, figure \ref{fig:realizations}. The covariance after observation is visualized in figure \ref{fig:obscov}. The covariance `tent' has been pressed down at points where $x\approx \pm .5$ and/or $y\approx\pm .5$, which are the values where the observations were made.

\begin{figure}
    \centering
        \epsfig{file=figs/obscov.pdf,width=5cm}
    \caption{The covariance function from {\sffamily `observation.py'} after observation. Compare this with the covariance function before observation, visualized in figure \ref{fig:cov} }
    \label{fig:obscov}
\end{figure}

\section{Higher-dimensional GPs}\label{sec:highdim}

In addition to functions of one variable such as $f(x)$, this package supports Gaussian process priors for functions of many variables such as $f(\mathbf{x})$, where $\mathbf{x}=[x_0\ldots x_{n-1}]$. This is useful for modeling dynamical or biological functions of many variables as well as for spatial statistics.

When array is passed into a \code{Mean}, \code{Covariance} or \code{Realization}'s init method or one of these objects is evaluated on an array, the array's last index is understood to iterate over spatial dimension. To evaluate a covariance $C$ on the ordered pairs $(0,1)$, $(2,3)$, $(4,5)$ and $(6,7)$, the user could pass in the following two-dimensional \pkg{NumPy} array:
\begin{verbatim}
[[0,1]
 [2,3]
 [4,5]
 [6,7]]
\end{verbatim}
or the following three-dimensional array:
\begin{verbatim}
[[[0,1]
  [2,3]],

  [4,5]
  [6,7]]]
\end{verbatim}
Either is fine, since in both the last index iterates over elements of the ordered pairs.

The exception to this rule is one-dimensional input arrays. The array
\begin{verbatim}
[0, 1, 2, 3, 4, 5, 6, 7]
\end{verbatim}
is interpreted as an array of eight one-dimensional values, whereas the array
\begin{verbatim}
[[0, 1, 2, 3, 4, 5, 6, 7]]
\end{verbatim}
is interpreted as a single eight-dimensional value according to the convention above.

Means and covariances learn their spatial dimension the first time they are called or observed. Some covariances, such as those specified in geographic coordinates, have an intrinsic spatial dimension. Realizations inherit their spatial dimension from their means and covariances when possible, otherwise they infer it the first time they are called. If one of these objects is subsequently called with an input of a different dimension, it raises an error.

\subsection{Covariance function bundles and coordinate systems}
The examples so far, starting with \code{examples/cov.py}, have used the covariance function \code{matern.euclidean}. This function is an attribute of the \code{matern} object, which is an instance of class \code{covariance_function_bundle}.

Instances of \code{covariance_function_bundle} have three attributes, \code{euclidean}, \code{geo_deg} and \code{geo_rad}, which correspond to standard coordinate systems:
\begin{itemize}
    \item \code{euclidean}: $n$-dimensional Euclidean coordinates.
    \item \code{geo_deg}: Geographic coordinates (longitude, latitude) in degrees, with unit radius.
    \item \code{geo_rad}: Geographic coordinates (longitude, latitude) in radians, with unit radius.
\end{itemize}

See \href{http://code.google.com/p/pymc}{the package documentation} for information regarding creation and extension of covariance function bundles.

\section{Basis covariances}\label{sec:basis}

\begin{figure}[htbp]
    \centering
        \epsfig{file=figs/basiscov.pdf,width=8cm}
        \caption{Three realizations of an observed Gaussian process whose covariance is an instance of \code{BasisCovariance}. The basis in this case is function \code{fourier_basis} from module \code{cov_funs}. 25 basis functions are used.}
    \label{fig:basiscov}
\end{figure}

It is possible to create random functions from linear combinations of finite sets of basis functions $\{e\}$ with random coefficients $\{c\}$:
\begin{eqnarray*}
    f(x) = M(x) + \sum_{i_0=0}^{n_0-1}\ldots \sum_{i_{N-1}=0}^{n_{N-1}-1} c_{i_1\ldots i_{N-1}} e_{i_1\ldots i_{N-1}}(x), \\
    \{c\}\sim \textup{N}(0,K).
\end{eqnarray*}
It follows that $f$ is a Gaussian process with mean $M$ and covariance defined by
\begin{eqnarray*}
    C(x,y)=\sum_{i_0=0}^{n_0-1}\ldots \sum_{i_{N-1}=0}^{n_{N-1}-1} \sum_{j_0=0}^{n_0-1}\ldots \sum_{j_{N-1}=0}^{n_{N-1}-1} e_{i_0\ldots i_{N-1}}(x) e_{j_0\ldots j_{N-1}}(x) K_{i_0\ldots i_{N-1}, j_0\ldots j_{N-1}},
\end{eqnarray*}
where $K$ is the covariance of the coefficients $c$.

Particularly successful applications of this general idea are:
\begin{description}
    \item[Random Fourier series:] $e_i(x) = \sin(i\pi x/L)$ or $\cos(i\pi x/L)$. See \cite{spanos}.
    \item[Gaussian process convolutions:] $e_i(x) = \exp(-(x-\mu_n)^2)$. See \cite{convolution}.
    \item[B-splines:] $e_i(x) = $ a polynomial times an interval indicator. See \href{http://en.wikipedia.org/wiki/Basis_B-spline}{Wikipedia}'s article.
\end{description}
Such representations can be very efficient when there are many observations in a low-dimensional space, but are relatively inflexible in that they generally produce realizations that are infinitely differentiable. In some applications, this tradeoff makes sense.

This package supports basis representations via the \code{BasisCovariance} class:
\begin{verbatim}
    C = BasisCovariance(basis, cov, **basis_params)
\end{verbatim}
The arguments are:
\begin{description}
    \item[\code{basis}:] Must be an array of functions, of any shape. Each basis function will be evaluated at $x$ with the extra parameters. The basis functions should obey the same calling conventions as mean functions: return values should have shape \code{x.shape[:-1]} unless $x$ is one-dimensional, in which case return values should be of the same shape as \code{x}. Note that each function should take the entire input array as an argument.
    \item[\code{cov}:] An array whose shape is either:
        \begin{itemize}
            \item Of the same shape as \code{basis}. In this case the coefficients are assumed independent, and \code{cov[i[0],...,i[N-1]]} (an $N$-dimensional index) simply gives the prior variance of the corresponding coefficient.
            \item Of shape \code{basis.shape * 2}, using \proglang{Python}'s convention for tuple multiplication. In this case \code{cov[i[0],...,i[N-1], j[0],...,j[N-1]]} (a $2N$-dimensional index) gives the covariance of $c_{i_0\ldots i_{N-1}}$ and $c_{j_1\ldots j_{N-1}}$.
        \end{itemize}
        Internally, the basis array is ravelled and this covariance tensor is reshaped into a matrix. This input convention makes it easier to keep track of which covariance value corresponds to which coefficients. The covariance tensor must be symmetric (\code{cov[i[0],...,i[N-1], j[0],...,j[N-1]]} $=$ \code{cov[j[0],...,j[N-1], i[0],...,i[N-1]]}), and positive semidefinite when reshaped to a matrix.
    \item[\code{basis_params}:] Any extra parameters required by the basis functions.
\end{description}

\section{Separable bases}

Many bases, such as Fourier series, can be decomposed into products of functions as follows:
\begin{eqnarray*}
    e_{i_0\ldots i_{N-1}}(x) = \prod_{j=0}^{N-1}e_{i_j}^j(x)
\end{eqnarray*}
Basis covariances constructed using such bases can be represented more efficiently using \code{SeparableBasisCovariance} objects. These objects are constructed just like \code{BasisCovariance} objects, but instead of an $n_0\times \ldots \times n_{N-1}$ array of basis functions they take a nested lists of functions as follows:
\begin{verbatim}
    basis = [ [e[0][0], ... ,e[0][n[0]-1]]
                       ...
              [e[N-1][0], ... ,e[N-1][n[N-1]-1]] ].
\end{verbatim}
For an $N$-dimensional Fourier basis, each of the \code{e}'s would be a sine or cosine; frequency would increase with the second index. As with \code{BasisCovariance}, each basis needs to take the entire input array \code{x} and \code{basis_params} as arguments. See \code{fourier_basis} in \code{examples/gp/basiscov.py} for an example.

\subsection{Example}

Once created, a \code{BasisCovariance} or \code{SeparableBasisCovariance} object behaves just like a \code{Covariance} object, but it and any \code{Mean} and \code{Realization} objects associated with it will take advantage of the efficient basis representation in their internal computations. An example of \code{SeparableBasisCovariance} usage is given in \code{pymc/examples/gp/basis_cov.py}. Compare its output in figure \ref{fig:basiscov} to that in figure \ref{fig:obs}.
 

\chapter{Incorporating Gaussian processes in PyMC probability models}\label{cha:PyMC}
This chapter will show you how to build and fit statistical models that go beyond simple nonparametric regression.







\section{Gaussian process submodels}

This package represents a Gaussian process $f\sim \textup{GP}(M,C)$ as a \code{GaussianProcess} object which, as you might expect, is a \pkg{PyMC} \code{stochastic} \citep{pymc} whose value is a \code{Realization} object. It is not feasible to endow a full \code{GaussianProcess} with a \code{logp} attribute, so \code{GaussianProcess} objects cannot be handled by \pkg{PyMC}'s standard MCMC machinery \citep{pymc}.

However, the evaluation $f(x_*)$ on a mesh $x_*$ is a simple multivariate normal random variable, which can be handled by the standard machinery. If $f(x_*)$ is incorporated in the model as a variable, a minor extension to the standard machinery (implemented by this package) makes it possible to handle $f$ itself as well.

Pairs of $f$ and $f(x_*)$ variables are housed in container objects of class \code{GPSubmodel}. \code{GaussianProcess} objects can only be incorporated in \pkg{PyMC} probability models in \code{GPSubmodel} objects.






\subsection{Example: nonparametric regression with unknown mean and covariance parameters}\label{sub:BasicMCMC}

A GP submodel is created in \code{pymc/examples/gp/\pkg{PyMC}model.py} with the following call:
\begin{CodeChunk}
\begin{CodeInput}
sm = gp.GPSubmodel('sm',M,C,fmesh)
\end{CodeInput}
\end{CodeChunk}
There are two stochastic variables in the submodel: \code{sm.f} and \code{sm.f_eval}. The first is the actual Gaussian process $f$: a stochastic variable valued as a \code{Realization} object. The second is $f(x_*)$, where $x_*$ is the input argument \code{fmesh}.

Once the GP submodel is created, we can create other variables that depend on $f$ and $f(x_*)$. In \code{\pkg{PyMC}model.py}, the observation $d$ depends on $f(x_*)$: 
\begin{CodeChunk}
\begin{CodeInput}
d = pymc.Normal('d',mu=sm.f_eval, tau=1./V, value=init_val, observed=True)
\end{CodeInput}
\end{CodeChunk}

The full probability model is shown as a directed acyclic graph in figure \ref{fig:unobservedModel}. It illustrates the dependency relationships between the variables in a GP submodel.

The file \code{pymc/examples/gp/MCMC.py} fits the probability model created in \code{\pkg{PyMC}model.py} using MCMC. The `business part' of the file is very simple:
\begin{CodeChunk}
\begin{CodeInput}
GPSampler = MCMC(PyMCmodel)
GPSampler.isample(iter=5000,burn=1000,thin=100)    
\end{CodeInput}
\end{CodeChunk}
Most of the code in the file is devoted to plotting, and its output is shown in figure \ref{fig:MCMCOutput}. Note that after the MCMC run \code{GPSampler.trace('sm_f')[:]} yields a sequence of \code{Realization} objects, which can be evaluated on new arrays for plotting. GP realizations can even be tallied on disk using the HDF5 backend, see \cite{pymc}.

\begin{figure}
    \centering
        \epsfig{file=figs/unobservedModel.pdf, width=15cm}
    \caption{The \pkg{PyMC}-generated directed acyclic graph representation of the extended nonparametric regression model created by \code{pymc/examples/gp/PyMCModel.py}. Ellipses represent \code{Stochastic} objects (variables whose values are unknown even if their parents' values are known), triangles represent \code{Deterministic} objects (variables whose values can be determined if their parents' values are known), and rectangles represent \code{Stochastic} objects with the \code{isdata} flag set to \code{True} (data). Rectangles represent potentials. Arrows point from parent to child. The submodel contains the Gaussian process \code{sm\_f} and its evaluation \code{sm\_f\_eval} on input array \code{sm\_mesh}. It also contains the mean \code{sm\_M\_eval} of \code{sm\_f\_eval} and the lower-triangular Cholesky factor \code{sm\_S\_eval} of its covariance matrix, and a potential \code{sm_fr_check} that forces that covariance matrix to remain positive definite. The actual covariance evaluation \code{sm\_C\_eval} is not needed by the model, but it is exposed for use by Gibbs step methods.}
    \label{fig:unobservedModel}
\end{figure}

\begin{figure}
    \centering
        \epsfig{file=figs/gibbsSamples.pdf,width=8cm}
        % \epsfig{file=figs/metroSamples.pdf,width=10cm}
    \caption{The output of \code{pymc/examples/gp/MCMC.py}. The left-hand panel shows all the samples generated for the Gaussian process $f$, and the right-hand panel shows the trace of $f(0)$.}
    \label{fig:MCMCOutput}
\end{figure}






\section{Step methods}
% \label{sec:step-methods} 
Since $f$ has no \code{logp} attribute, the Metropolis-Hastings family of step methods \citep{pymc} cannot be used to update $f$, $f(x_*)$ or any of their mean or covariance parameters. This package uses a relatively simple work-around that will be described here. Throughout this section, the parents of $f$ are denoted $P$ and the children $K$.


\subsection{Step methods that handle parents of Gaussian processes}
If we could come up with a probability density function for $f$, the Metropolis-Hastings acceptance ratio for a proposed value $P_p$ of the parents \emph{and} a proposed value $\tilde f$ for $f$ would be:
\begin{eqnarray*}
    \frac{p(K|f_p)\ p(f_p|P_p)\ q(P)}{p(K|f)\ p(f|P)\ q(P_p)}
\end{eqnarray*}
where $q$ denotes the proposal density. Now, suppose we proposed a value for $f$ conditional on the proposed values for the parents $P$ \emph{and} conditional on $f(x_*)$. The new acceptance ratio would become
\begin{eqnarray*}
    \frac{p(K|f_p)\ p(f_p|f(x_*), P_p)\ p(f(x_*) | P_p)\ q(f_p|f(x_*),f_p, P_p)\ q(P)}{p(K|f)\ p(f|f(x_*), P)\ p(f(x_*) | P)\ q(f_p|f(x_*),f,P)\ q(P_p)}
\end{eqnarray*}
 We want to avoid computing all terms with $f$ or $f_p$ in the consequent position:
\begin{eqnarray*}
    p(f_p|f(x_*), P),\\ q(f|f(x_*),f_p,P),\\ p(f|f(x_*), P),\\ q(f_p|f(x_*),f,P_p),
\end{eqnarray*}
but all other terms are fine. We can make the problem terms cancel by choosing our proposal distribution as follows:
\begin{eqnarray*}
    q(f_p|f(x_*),f,P_p) = p(f_p|f(x_*), P).
\end{eqnarray*}
In other words, if we propose $f$ from its prior distribution conditional on $f(x_*)$ and its parents whenever we propose $f(x_*)$, we don't have to worry about computing the intractable terms. This argument can be made more rigorous by replacing $f$ with its evaluation at all the points at which we would ever want to know it.

By choosing the same proposal distribution for $\tilde f$ as above, we again avoid having to compute the intractable terms. In other words, every time a value is proposed for a \code{GP}'s parent, a value must be proposed for the \code{GP}  conditional on its value's evaluation on its mesh, and the prior probability of the \code{GP}'s children must be included in the acceptance ratio.

\bigskip
To summarize, any Metropolis-Hastings step method can handle the parents of $f$, as well as $f(x_*)$, if it proposes values for $f$ jointly with its target variable as outlined above. 

This minor alteration can be done using the function \code{wrap_metropolis_for_gp_parents}, which takes a subclass of \code{Metropolis} as an argument and returns a new step method class with altered \code{propose} and \code{reject} methods. The function automatically produces modified versions of all Metropolis step methods in \pkg{PyMC}'s step method registry (\code{Metropolis}, \code{AdaptiveMetropolis}, etc.) \citep{pymc}. The modified step methods are automatically assigned to parents of Gaussian processes.

\subsection{Choosing a mesh} 

The mesh points $x_*$ are the points where Metropolis-Hastings step methods can `grab' the value of $f$ to moderate the variance of its proposal distribution. If $x_*$ is an empty array, $f$'s value will be proposed from its prior, and rejection rates are likely to be quite large. If $x_*$ is too dense, on the other hand, computation of the log-probability of $f(x_*)$ will be expensive, as it scales as the cube of the number of points in the mesh.This continuum is illustrated in figure \ref{fig:meshpropose}. Finding the happy medium requires some experimentation.

Another important point to bear in mind is that if $f$'s children depend on its value only via its evaluation on the mesh, the likelihood terms $p(K|f_p)$ and $p(K|f)$ will cancel. In other words, if the mesh is chosen so that $p(K|f)=p(K|f(x_*))$ then the proposed value of $f$ will have no bearing on the acceptance probability of the proposed value of $f(x_*)$ or of the parents $P$. This is the situation in \code{\pkg{PyMC}Model.py}. Such a mesh choice will generally improve the acceptance rate.

\begin{figure}
    \centering
        \epsfig{file=figs/nomeshpropose.pdf,width=4cm}
        \epsfig{file=figs/lightmeshpropose.pdf,width=4cm}
        \epsfig{file=figs/densemeshpropose.pdf,width=4cm}
    \caption{Several possible proposals of $f$ (curves) given proposed values for $f(x_*)$ (heavy dots) with no mesh (top), a sparse mesh (middle), and a dense mesh (bottom). Proposal distributions' envelopes are shown as shaded regions, with means shown as broken lines. With no mesh, $f$ is proposed from its prior and the acceptance rate will be very low. A denser mesh permits a high degree of control over $f$, but computing the log-probability will be more expensive.}
    \label{fig:meshpropose}
\end{figure}

\subsection{Gibbs steps} 
If all of $f$'s children, $K$, depend on it as follows:
\begin{eqnarray*}
    K_i|f \stackrel{\textup{\tiny ind}}{\sim} \textup{Normal}(f(x_{*i}),V_i)
\end{eqnarray*}
then $f(x_*)$ can be handled by the \code{GPEvaluationGibbs} step method. This step method is used in \code{MCMC.py}:
\begin{CodeChunk}
\begin{CodeInput}
GPSampler.use_step_method(gp.GPEvaluationGibbs, GPSampler.submod, \
    GPSampler.V, GPSampler.d)
\end{CodeInput}
\end{CodeChunk}
The initialization arguments are the Gaussian process submodel that contains $f$, the observation variance of $f$'s children, and the children, in this case the vector-valued normal variable $d$. 

\code{GPEvaluationGibbs} covers the standard submodel encountered in geostatistics, but there are many conjugate situations to which it does not apply. If necessary, special step methods can be written to handle these situations. If \code{GPEvaluationGibbs} is not assigned manually, $f(x_*)$ will generally be handled by a wrapped version of \code{AdaptiveMetropolis}.





\section{Geostatistical example}\label{sub:geostat}
\begin{figure}
    \centering
        \epsfig{file=figs/elevmean.pdf, width=5cm}
        \epsfig{file=figs/elevvar.pdf, width=5cm}
    \caption{The posterior mean and variance surfaces for the $v$ variable of the Walker lake example. The posterior variance is relatively small in the neighborhood of observations, but large in regions where no observations were made.}
    \label{fig:walker}
\end{figure}
\begin{figure}
    \centering
        \epsfig{file=figs/elevdraw0.pdf, width=5cm}
        \epsfig{file=figs/elevdraw1.pdf, width=5cm}
    \caption{Two realizations from the posterior distribution of the $v$ surface for the Walker Lake example. Elevation is measured in meters.}
    \label{fig:walkerreal}
\end{figure}
Bayesian geostatistics is demonstrated in the folder \code{pymc/examples/gp/more_examples/Geostatistics}. File \code{getdata.py} downloads the Walker Lake dataset of Isaaks and Srivastava \citep{isaaks} from the internet and manipulates the $x$ and $y$ coordinates into the array format described in section \ref{sec:highdim}. File \code{model.py} contains the geostatistical model specification, which is

\begin{eqnarray*}
    d|f \sim \textup{Normal}(f(x),V)\\
    f|M,C \sim \textup{GP}(M,C) \\
    M:x\rightarrow m\\
    C:x,y,\mathtt{amp},\mathtt{scale},\mathtt{diff\_degree}\rightarrow \mathtt{matern.euclidean}(x,y;\mathtt{amp},\mathtt{scale},\mathtt{diff\_degree})\\
    p(m)\propto 1\\
    \mathtt{amp}\sim \textup{Exponential}(7e-5) \\
    \mathtt{scale}\sim \textup{Exponential}(4e-3) \\
    \mathtt{diff\_degree}\sim \textup{Uniform}(.5,2)\\ 
    V\sim \textup{Exponential}(5e-9)\\
\end{eqnarray*}

File \code{mcmc.py} fits the model and produces output maps.  The output of \code{mcmc.py} is shown in figures \ref{fig:walker} and \ref{fig:walkerreal}. Figure \ref{fig:walker} shows the posterior mean and variance of the $v$ variable of the dataset, which is a function of elevation (see \cite{isaaks}, appendix A). 

\chapter{Extending the covariance functions: Writing your own, using alternate coordinate systems, building in anisotropy and nonstationarity}\label{cha:usercov}
This section will give you a brief tour of the \module{cov_funs} module and point out plugs where you can add functionality that isn't included in this package.
\section{Nonstandard Stochastics}
See decorator syntax.

\subsection{Subclassing Stochastic}
Numerical-valued stochastics, just use stochastic_from_distribution. Otherwise ...

\section{Nonstandard Deterministics}
Consider Lambda first, otherwise...


\section{User-defined step methods}
\subsection{Metropolis-Hastings step methods}

\subsection{Gibbs step methods}

\subsection{General step methods}

\subsection{Using step methods from a Sampler}


\section{Specialized submodels}
Use Container


\section{New fitting algorithms}

\subsection{Monte Carlo fitting algorithms}
Subclass Sampler

\subsection{General fitting algorithms}

\chapter{Overview of algorithms} \label{cha:numerics} 
More detail on these algorithms, as well as the internal workings of the mean, covariance and realization objects and the observe function, are given in the algorithm documentation. 
\chapter{Overview of algorithms} 
\label{cha:numerics} 

More detail on these algorithms, as well as the internal workings of the mean, covariance and realization objects and the observe function, are given in the algorithm documentation.

\section{Gaussian processes and the multivariate normal distribution}

Gaussian processes generalize the multivariate normal distribution from vectors to functions, like the multivariate normal distribution generalizes the univariate normal distribution from scalars to vectors. The progression is as follows:
\begin{equation}
    \begin{array}{ll}
        y\sim\textup N(\mu,V): & \textup{$y$, $\mu$, $V$ are scalars}\\\\
        \vec y\sim\textup N(\vec \mu,C): & \textup{$\vec y$ and $\vec \mu$ are vectors, $C$ is a matrix}\\\\
        f\sim\textup{GP}(M, C): & \textup{$f$ and $M$ are functions of one variable, $C$ is a function of two variables}
    \end{array}
\end{equation}

One of the nice things about all Gaussian distributions (parameterized by mean and covariance) is that they're easy to marginalize. For example, each element of a vector with a multivariate normal distribution has a univariate normal distribution:
\begin{eqnarray*}
    \vec y\sim\textup N(\vec \mu,C)\\
    \Rightarrow \vec y_i\sim\textup N(\vec \mu_i,C_{i,i}),
\end{eqnarray*}
and any subvector of a vector with a multivariate normal distribution has a multivariate normal distribution:
\begin{eqnarray*}
    \vec y\sim\textup N(\vec \mu,C)\\
    \Rightarrow \vec y_{i_1\ldots i_2}\sim\textup N(\vec \mu_{i_1\ldots i_2},C_{i_1\ldots i_2,i_1\ldots i_2}).
\end{eqnarray*}

This marginalizability applies to GP's as well. If $\vec x$ is a vector of values,
\begin{equation}
    \begin{array}{l}
        f\sim\textup{GP}(M, C)\\\\
        \Rightarrow f(\vec x) \sim\textup N(M(\vec x), C(\vec x,\vec x)).
    \end{array}
\end{equation}
In other words, any evaluation of a Gaussian process realization has a multivariate normal distribution. Its mean is the corresponding evaluation of the associated mean function, and its covariance is the corresponding evaluation of the associated covariance function. You can probably start to see why this fact is important for working with GPs on a computer.

\subsection{Observations}
As mentioned earlier, if $f$ has a GP prior and normally-distributed observations of $f$ are made at a finite set of values, $f$'s posterior is a Gaussian process also:
\begin{eqnarray*}
    \left.\begin{array}{l}
        d_i \stackrel{\tiny{\textup{ind}}}{\sim} \textup{N}(f(o_i), V_i)\\
        f \sim \textup{GP}(M,C)
    \end{array}\right\}\Rightarrow f|d \sim \texttt{GP}(M_o, C_o).
\end{eqnarray*}
Denoting by $o$ the array of observation values $[o_0\ldots o_{N-1}]$ and $V$ a matrix with observation variances $[V_0\ldots V_{N-1}]$ on its diagonal, $M_o(x)$ and $C_o(x,y)$ are as follows for arbitrary input vectors $x$ and $y$:
\begin{eqnarray*}
    M_o(x) = M(x) + C(x,o)[C(o,o) + V]^{-1}(f(o)-M(o))\\
    C_o(x,y) = C(x,y) - C(x,o)[C(o,o) + V]^{-1}C(o,y).
\end{eqnarray*}

\subsection{Low-rank observations}

If $C(o,o)+V$ is singular, some elements of $f(o)$ can be computed from others with no uncertainty. In other words, there exists a partition $[o_*, o_{**}]$ of $o$ and corresponding partition of the data such that $C(o_*,o_*)+V$ is full-rank and
\begin{eqnarray*}
    f(o_{**}) = M_{o_*}(o_{**}),
\end{eqnarray*}
where $M_{o_*}$ can be computed from the formula above.

In such cases, this package's strategy is to observe $f$ at a subvector $o_*$ of $o$, such that $C(o_*,o_*)+V$ is full-rank but if any elements were added to $o_*$ it would pick up a very small eigenvalue. The function \function{predictive_check} optionally checks the remaining data values $d_{**}$ against $M_{o_*}(o_{**})$, and raises an error if the two aren't equal up to a user-defined threshold.

This threshold is parameter \code{relative_precision} from \texttt{Covariance}'s init method, multiplied by the largest value on the diagonal of $C(o_*,o_*)+V$; intuitively, the maximal variance of $f(o_{**})$ given $f(o_*)$ is a multiple of the maximal variance of $f(o_*)$.

\section{Incomplete Cholesky factorizations}\label{sec:ichol} % (fold)

In addition to numpy's linear algebra support, this package uses some Fortran subroutines wrapped using \citetitle[http://cens.ioc.ee/projects/f2py2e/]{f2py}. They require \citetitle[http://www.netlib.org/blas/]{BLAS} and \citetitle[http://www.netlib.org/lapack/]{LAPACK} libraries (eg, \citetitle[http://math-atlas.sourceforge.net/]{ATLAS}) on your system, and the \file{setup.py} script will attempt to find these. If you don't have optimized BLAS and LAPACK installed, it's a good idea to install them; this package and numpy in general will be much faster. Some operating systems (such as Mac OS X) ship with optimized BLAS and LAPACK libraries included.

The following incomplete Cholesky factorization functions are found in the Fortran files \file{linalg_utils.f} and \file{incomplete_chol.f} :
\begin{description}

    % \item[\function{dtrsm_wrap(a,b,uplo='U',transa='N',alpha=1.)}:] A wrapper for the BLAS routine \citetitle[http://www.netlib.org/blas/dtrsmf.f]{DTRSM}, which solves triangular systems \texttt{$\alpha$a$^{op}$ x = b} in-place (that is, \texttt{b} is overwritten by $x$).
    % \begin{itemize}
    %     \item If \texttt{uplo='U'}, \texttt{a} is assumed to be an upper triangle; if \texttt{uplo='L'}, \texttt{a} is assumed to be a lower triangle.
    %     \item If \texttt{transa='T'}, \texttt{$\alpha$a.T*x = b} is solved. If \texttt{transa='N'}, \texttt{$\alpha$a*x = b} is solved.
    % \end{itemize}
    % 
    % \item[\function{dtrmm_wrap(a,b,uplo='U',transa='N',alpha=1.)}:] A wrapper for the BLAS routine \citetitle[http://www.netlib.org/blas/dtrsmf.f]{DTRMM}, which does the triangular matrix multiplication \texttt{$\alpha$a$^{op}$x = b} in-place (that is, \texttt{b} is overwritten by $x$).
    % \begin{itemize}
    %     \item If \texttt{uplo='U'}, \texttt{a} is assumed to be an upper triangle; if \texttt{uplo='L'}, \texttt{a} is assumed to be a lower triangle.
    %     \item If \texttt{transa='T'}, \texttt{$\alpha$a.T*x} is computed. If \texttt{transa='N'}, \texttt{$\alpha$a*x} is computed.
    % \end{itemize}
    % 
    % \item[\function{info = dpotrf_wrap(a)}:] A wrapper for the LAPACK routine \citetitle[http://www.netlib.org/lapack/double/dpotrf.f]{DPOTRF}, which tries to overwrite positive-definite matrix \texttt{a} with an upper-triangular Cholesky factor in-place. If the return value \texttt{info} is positive, \texttt{a} is not positive definite and the algorithm has failed.

    \item[\function{U, m, piv = ichol(diag, reltol, rowfun)}:] An implementation of the incomplete Cholesky decomposition, based on a port of one of the functions in the \citetitle[http://www.kyb.tuebingen.mpg.de/bs/people/seeger/]{chol_incomplete} package by Matthias Seeger.

The arguments are:
    \begin{description}
        \item[\texttt{diag}:] The diagonal of an \texttt{n} by \texttt{n} covariance matrix C.
        \item[\texttt{reltol}:] If the ratio of the \texttt{i}'th pivot to the maximum pivot is found to be less than \texttt{reltol}, the \texttt{i}'th pivot is assumed to be zero.
        \item[\texttt{rowfun}:] A Python function. The call \function{rowfun(i,p,rowvec)} should perform the update \texttt{rowvec[i,i+1:]=C[i,p[i+1:]]} in-place.
    \end{description}

The outputs are:
    \begin{description}
        \item[M:] The rank of C. Note that M$\le$\texttt{n}.
        \item[\texttt{piv}:] A length-\texttt{n} vector of pivots.
        \item[\texttt{U}:] An M-by-\texttt{n} upper-triangular matrix that satisfies \texttt{U[:,argsort(piv)].T * U[:,argsort(piv)] = C}.
    \end{description}

Because the full matrix $C$ does not need to be computed ahead of time, the algorithm is able to factor $C$ in $O(m^2 n)$ operations \cite{incompchol}. The algorithm is implemented in Fortran, but a Python version would be as follows:

\begin{verbatim}

def swap(vec,i,j):
    temp = vec[i]
    vec[i] = vec[j]
    vec[j] = temp

def ichol(diag, reltol, rowfun):

    piv = arange(n)
    U = zeros((n,n),dtype=float).view(matrix)
    rowvec = zeros(n,dtype=float)

    for i in range(n):
        l = diag[i:].argmax()
        maxdiag = diag[l]

        if maxdiag < reltol:
            m=i
            return U[:m,:], m, piv

        swap(diag,i,l)
        swap(p,i,l)

        temp = U[:i,i]
        U[:i,i] = U[:i,l]
        U[:i,l] = temp


        U[i,i] = sqrt(diag[i])
        rowvec[i:] = C[i,piv[i+1:]]

        if i > 0:
            rowvec -= U[:i,i].T * U[:i,i+1:]

        U[i,i+1:] = rowvec[i+1:] / U[i,i]
        diag[i+1:] -= U[i,i+1:].view(ndarray) ** 2

    m=n
    return U, m, piv
\end{verbatim}

Function \function{ichol} is wrapped by the \class{Covariance} method \method{cholesky}.

\item[\function{m, piv = ichol_continue(U, diag, reltol, rowfun, piv)}:]
This function computes the Cholesky factorization of an \texttt{n} by \texttt{n} covariance matrix C from the factor of its upper-left \texttt{n}$_*$ by \texttt{n}$_*$ submatrix C$_*$. Its input arguments are as follows:
\begin{description}
    \item[\texttt{U}:] Unlike \function{ichol}, this function overwrites a matrix in-place. Suppose the Cholesky factor of C$_*$, \texttt{U}$_*$, is of rank M$_*$. On input, \texttt{U} must be an \texttt{[m + (n-n$_*$)]}-by-\texttt{n} matrix arranged like this:
    \begin{eqnarray*}
        \left[
        \begin{array}{ccc}
            \texttt{U}_*\texttt{[:,:m}_*\texttt{]} & \texttt{U}_*\texttt{[:,:m}_*\texttt{].T.I C[:m}_*\texttt{,n}_*\texttt{:]} & \texttt{U}_*\texttt{[:,m}_*\texttt{:]}\\
            \texttt{0}&\texttt{0}&\texttt{0}
        \end{array}
        \right]
    \end{eqnarray*}
On exit, \texttt{U} will be an M-by-\texttt{n} upper-triangular matrix that satisfies \texttt{U[:,argsort(piv)].T * U[:,argsort(piv)]=C}.
    \item[\texttt{piv}:] Denote by \texttt{piv}$_*$ the pivot vector associated with \texttt{U}$_*$ On input, \texttt{piv} must be a length-\texttt{n} vector laid out like this:
    \begin{eqnarray*}
        \begin{array}{ccc}
            [\texttt{piv}_*\texttt{[:m}_*\texttt{]} & \texttt{arange(n-n$_*$)} & \texttt{piv}_*\texttt{[m}_*\texttt{:]}]
        \end{array}
    \end{eqnarray*}
    \item[\texttt{diag}:] The length \texttt{n-n}$_*$ diagonal of \texttt{C[n$_*$:,n$_*$]}.
    \item[\texttt{rowfun}:] The call \function{rowfun(i,p,rowvec)} should perform the update \texttt{rowvec[i,i+1:]} \texttt{=} \texttt{C[i,p[i+1:]]} in-place.
\end{description}

    The input parameter \texttt{reltol}, as well as the output parameters \texttt{piv} and M and the updated matrix \texttt{U}, should be interpreted just like their counterparts in \function{ichol}.

    The algorithm is just like the algorithm in ichol, but the index \texttt{i} iterates over \texttt{range(m$_*$,m$_*$+n-n$_*$)} (and the index of \texttt{diag} is downshifted appropriately).

    \function{ichol_continue} is wrapped by the \class{Covariance} method \method{continue_cholesky}, so it should rarely be necessary to call it directly.

    \item[\function{U, m, piv = ichol_full(c, reltol)}:] Just like \texttt{ichol}, but instead of a diagonal and a `get-row' function a full covariance matrix C is required as an input.

    \item[\function{U, m, piv = ichol_full(basis, nug, reltol)}:] Just like \texttt{ichol}, but the following arguments are required:
    \begin{description}
        \item[\texttt{basis}:] The evaluation of a basis function on the mesh, multiplied by the Cholesky factor of the coefficients' covariance matrix. This is itself a square root of the covariance matrix, but running it through \texttt{ichol_basis} allows for observations with nonzero variance and essentially pivots for better accuracy even in the zero-variance case.
        \item[\texttt{nug}:] A vector that will effectively be added to the diagonal of the covariance matrix.
    \end{description}
\end{description}

%     \item[\function{x = trisolve(A,b,uplo='U',transa='N',inplace=False)}:] Solves the triangular system \texttt{$\alpha$a$^{op}$ x = b} using \function{dtrsm_wrap} and returns $x$. If \texttt{A} is found to be singular, an error is raised.
%     \begin{itemize}
%         \item \texttt{A} is assumed upper-triangular if \code{uplo='U'} and lower-triangular if \code{uplo='L'}.
%         \item If \texttt{transa='T'}, \texttt{$\alpha$a.T*x = b} is solved. If \texttt{transa='N'}, \texttt{$\alpha$a*x = b} is solved.
%         \item If \texttt{inplace=True}, \texttt{b} is overwritten with $x$ in-place and returned. If \texttt{inplace=False}, \texttt{b} is not modified.
%     \end{itemize}
% 
%     \item[\function{b = trimult(A,x,uplo='U',transa='N',inplace=False)}:] Does the triangular multiplication \texttt{$\alpha$a$^{op}$x = b} using \function{dtrmm_wrap} and returns \texttt{b}.
%     \begin{itemize}
%         \item \texttt{A} is assumed upper-triangular if \code{uplo='U'} and lower-triangular if \code{uplo='L'}.
%         \item If \texttt{transa='T'}, \texttt{$\alpha$a.T8x} is computed. If \texttt{transa='N'}, \texttt{$\alpha$a*x} is computed.
%         \item If \texttt{inplace=True}, $x$ is overwritten with \texttt{b} in-place and returned. If \texttt{inplace=False}, $x$ is not modified.
%     \end{itemize}
% \end{description}

    % \item[\function{observe(M, C, obs_mesh, obs_vals, obs_V, lintrans, cross_validate = True)}:] This function calls \texttt{C.observe}, calls \texttt{M.observe} with the output, and then if \texttt{cross_validate=True} calls \texttt{predictive_check}.

%     If the return of \texttt{predictive_check} value is \texttt{False}, the values of some observations can be predicted with negligible uncertainty from the values of others, but they don't match their predicted values; in other words, the data are extremely improbable. A \texttt{ZeroProbability} exception is raised with some helpful suggestions if this is the case.
%
%     \item[\function{OK = predictive_check(obs_vals, obs_mesh, M, posdef_indices, tolerance)} :] Checks the value of M evaluated at \texttt{obs_mesh} sliced at the complement of \texttt{posdef_indices} against the corresponding \texttt{obs_vals}. If any of the differences are greater than \texttt{tolerance}, \texttt{False} is returned. Otherwise \texttt{True} is returned.
% \end{description}
% \section{Object internals}\label{sec:internals}
%
% The methods of \class{Mean}, \class{Covariance} and \class{Realization} will be described here.
%
% \section{Covariance}\label{sec:covarianceInt}
% \begin{description}
%
%     \item[\method{cholesky(x, apply_pivot = True, observed=True)}:] Returns an incomplete Cholesky factor of \texttt{C(x,x)}. The other arguments' meanings are as follows:
%     \begin{description}
%         \item[\texttt{apply_pivot}:] If \texttt{True}, the return value is a matrix \texttt{U} such that \texttt{U.T*U = C(x,x)}. If \texttt{False}, the return value is a dictionary. Element \texttt{'pivots'} is a vector of pivots returned by function \function{ichol}, and element \texttt{'U'} is the matrix \texttt{sig} returned by \function{ichol}.
%         \item[\texttt{observed}:] If \texttt{True}, the matrix \texttt{C(x,x)} is obtained conditional on observations that have been made involving C. If \texttt{False}, the matrix is obtained without regard to those observations.
%     \end{description}
%
%         \item[\method{continue_cholesky(x, x_old, chol_dict_old[, apply_pivot, observed])}:] Returns an incomplete Cholesky factor of \texttt{C(concatenate(x_old,x), concatenate(x_old,x))}. Arguments \texttt{apply_pivot} and \texttt{observed} are the same as for \texttt{cholesky}. \texttt{chol_dict_old} should be the incomplete Cholesky factorization of \texttt{C(x_old, x_old)} in the form of a dictionary such as the one produced by \texttt{C.cholesky}.
%
%         This method returns either a matrix or a dictionary depending on the value of \texttt{apply_pivot}. See \texttt{cholesky}.
%
%
%         \item[\method{relevant_slice, obs_mesh_new, U_for_draw = observe(obs_mesh, obs_V=0.)}:] All subsequent calls will be made under the assumption that observations of the random field at inputs \texttt{obs_mesh} have been made with variance \texttt{obs_V}. This method updates the following attributes of C:
%         \begin{itemize}
%             \item \texttt{observed}: This flag is set to \texttt{True}.
%             \item \texttt{obs_U} and \texttt{obs_piv}: The output of \texttt{self.cholesky(obs_mesh, apply_pivot=False, observed=False)}.
%             \item \texttt{obs_mesh}: The mesh on which self has been observed, sliced by \texttt{obs_piv[:m]}, where M is the rank of \texttt{obs_U}.
%             \item \texttt{obs_V}: The variances associated with the observations, sliced as \texttt{obs_mesh} is.
%             \item \texttt{obs_len}: The length of \texttt{obs_V}.
%         \end{itemize}
%
%         The return values are:
%         \begin{itemize}
%             \item \texttt{relevant_slice}: The indices included in the incomplete Cholesky factorization. These correspond to the values of \texttt{obs_mesh} that determine the other values, but not one another.
%             \item \texttt{obs_mesh_new}: \texttt{obs_mesh} sliced according to \texttt{relevant_slice}.
%             \item \texttt{U_for_draw}: An upper-triangular Cholesky factor of \texttt{self}'s evaluation on \texttt{obs_mesh} conditional on all previous observations.
%         \end{itemize}
%
%
%
%     \item[\method{__call__(x[, y, observed])}:]
%
%     \begin{itemize}
%         \item If only one argument $x$ is provided, \texttt{C(x,x) = V(f(x))} will be returned conditional on any observations that have been made.
%         \begin{itemize}
%             \item If C is unobserved or \texttt{observed = False}, its underlying function C is evaluated at each element of $x$. The \texttt{i}'th element of the return value is \texttt{c(x_i,x_i)}.
%             \item If C is observed and \texttt{observed = True}, its underlying function C is evaluated at each element of $x$. The \texttt{i}'th element of the return value is
%             \begin{eqnarray*}
%                 \texttt{c(x_i,x_i) - (U.T.I c(o,x_i)).T (U.T.I c(o,x_i))},
%             \end{eqnarray*}
%             where \texttt{o} is \code{self.obs_mesh} and \texttt{U} is \texttt{self.obs_U}, both of which were provided by \function{observe}.
%             \end{itemize}
%
%         \item If two arguments \texttt{(x,y)} are provided, \texttt{C(x,y)} will be returned conditional on any observations that have been made.
%         \begin{itemize}
%             \item If C is unobserved, \texttt{c(x,y)} will be returned where C is the underlying function. If $x$ and $y$ are references to the same array, only half the matrix will actually be computed; the other half will be filled in.
%         \item If C is observed, the return value will be
%         \begin{eqnarray*}
%             \texttt{c(x,y) - (U.T.I c(o,x)).T (U.T.I c(o,y))}
%         \end{eqnarray*}
%         where again \texttt{o} is \code{self.obs_mesh} and \texttt{U} is \texttt{self.obs_U}.
%         \end{itemize}
%
%     \end{itemize}
%     \end{description}
%
% \section{BasisCovariance}\label{sec:basisCovariance}
% \class{BasisCovariance} has methods and attributes that act just like those of \texttt{Covariance}, but of course the internal computations are different and tend to be faster.
%
% \section{Mean}\label{sec:meanInt}
% \begin{description}
%     \item[\method{observe(C, obs_mesh_new, obs_vals_new)}:] All subsequent calls will be made under the assumption that observations of the random field at inputs \texttt{obs_mesh} have been made with variance \texttt{obs_V}. Assumes that \texttt{C(obs_mesh_new, obs_mesh_new)} is positive definite; the \emph{function} \function{observe} slices \texttt{obs_vals} according to \texttt{Covariance.observe}'s output value \texttt{relevant_slice} before passing it to \texttt{Mean.observe}. This method updates the following attributes of M:
%     \begin{itemize}
%         \item \texttt{self.C}: The covariance of the random field (which has self as mean) that was observed. Several of $C$'s attributes are used by this method.
%         \item \texttt{self.obs_U}: \texttt{C.obs_U}.
%         \item \texttt{self.obs_mesh}: \texttt{C.obs_mesh}.
%         \item \texttt{obs_V}: \texttt{C.obs_V}.
%         \item \texttt{obs_len}: \texttt{C.obs_len}.
%         \item \texttt{self.dev}: \texttt{obs_mesh - self.__call__(obs_mesh, observed=False)}.
%         \item \texttt{self.reg_vec}: \texttt{self.obs_U[:,:m].T*self.dev}, where M is the rank of \texttt{self.obs_U}.
%     \end{itemize}
%
%     Note that \texttt{Mean.observe} will not cross-validate; if inconsistent observation values are used, it will simply ignore them. You'll need to call \texttt{predictive_check} yourself if you want cross-validation.
%
%     \item[\method{__call__(x[, observed])}:]
%     \begin{itemize}
%         \item If M has not been observed or \texttt{observed = False}, the return value of \texttt{M(x)} will just be \texttt{m(x)} where M is the underlying mean function.
%         \item If M has been observed and \texttt{observed = True}, the return value will be
%         \begin{eqnarray*}
%             \texttt{c(x,o) U.I r},
%         \end{eqnarray*}
%         where \texttt{r} is \code{self.reg_vec}, \texttt{U} is \code{self.obs_U} and C is \code{self.cov_fun}, all of which were provided by \function{observe}.
%
%     \end{itemize}
%
% \end{description}

% \section{Realization}\label{sec:realInt}
% A realization f maintains the following attributes:
% \begin{itemize}
%     \item \texttt{x_sofar}: The values at which self has already been evaluated.
%     \item \texttt{f_sofar}: Self's evaluation at \texttt{x_sofar}.
%     \item \texttt{M_internal}: An observed copy of self's mean.
%     \item \texttt{C_internal}: An observed copy of self's covariance.
% \end{itemize}
%
% When the evaluation \texttt{f(x)} is requested, the following happen:
% \begin{enumerate}
%     \item \texttt{x_sofar} is searched for elements of $x$. If any are found, the corresponding values of f aren't recomputed. Call the remaining elements \texttt{x_new}.
%     \item \texttt{C_internal} is observed at \texttt{x_new} with \texttt{obs_V=0}.
%     \item The output value \texttt{U_for_draw} is used to generate a normal random variable \texttt{f_new} with mean \texttt{M_internal(x_new)} and covariance \texttt{C_internal(x_new, x_new)}.
%     \item \texttt{M_internal} is observed with \texttt{obs_mesh = x_new}, \texttt{obs_vals = f_new}, \texttt{obs_taus = Inf}.
%     \item \texttt{f_new} is appropriately combined with f's evaluation at previously computed values of $x$ and the result is returned.
% \end{enumerate}
%
% If a realization's covariance is an instance of \texttt{BasisCovariance}, the behavior is different. At instantiation, the realization will draw coefficients for the basis terms; calls to the realization will simply be handled by calling the basis.

% chapter numerics (end)

% \chapter{Wishlist}\label{cha:wishlist}
% \section{Features}
% \begin{itemize}
%     \item Linear transformations.
%     \begin{itemize}
%         \item Create new GP's from old GP's via linear transformations. Maybe try to make a \class{LinearOperator} class.
%         \item Observe linear transformations of evaluations of GP.
%     \end{itemize}
%     \item Observations can be dependent (\texttt{obs_V} can be square rather than a diagonal).
%     \item Specific support for anisotropy.
%     \item Specific support for spline covariances.
%     \item Specific support for vector-valued GPs (cokriging).
% \end{itemize}
%
% \section{Optimizations}
% \begin{itemize}
%     % \item Markov random field covariances, which take advantage of the sparseness of the covariance for speed.
%     \item Speed up realization calls with just one value. All the array manipulation really bogs this case down, and this makes dynamical/autoregressive applications slow.
%     \begin{itemize}
%         \item There isn't a clear bottleneck, unfortunately. It might be necessary to write all the call methods in f2py or Pyrex.
%     \end{itemize}
% \item \function{ichol} and friends:
% \begin{itemize}
%     \item Don't allocate full-rank memory up-front.
%     \item Use a symbolic Cholesky decomposition to save some work.
%     \item Increase granularity of BLAS calls to take better advantage of threaded libraries.
%     \item Distributed memory.
% \end{itemize}
% \end{itemize}
%
% \section{Other}
% \begin{itemize}
%     \item Clean up the return values from \class{Covariance.observe}.
%     \item Try to figure out relative performance of incomplete Cholesky w/low-rank covariances to basis covariances, and relative performance of sparse Cholesky to Markov Random Field covariances.
% \end{itemize}

% section new_features (end)

% \chapter{Quick reference}\label{cha:reference}
 

\nocite{*}
\bibliographystyle{plain}
\bibliography{gp}

\end{document}
