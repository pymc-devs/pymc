%
% API Documentation for API Documentation
% Module pymc.distributions
%
% Generated by epydoc 3.0beta1
% [Tue Apr 29 09:12:28 2008]
%

%%%%%%%%%%%%%%%%%%%%%%%%%%%%%%%%%%%%%%%%%%%%%%%%%%%%%%%%%%%%%%%%%%%%%%%%%%%
%%                          Module Description                           %%
%%%%%%%%%%%%%%%%%%%%%%%%%%%%%%%%%%%%%%%%%%%%%%%%%%%%%%%%%%%%%%%%%%%%%%%%%%%

    \index{pymc \textit{(package)}!pymc.distributions \textit{(module)}|(}
\section{Module pymc.distributions}

    \label{pymc:distributions}

%%%%%%%%%%%%%%%%%%%%%%%%%%%%%%%%%%%%%%%%%%%%%%%%%%%%%%%%%%%%%%%%%%%%%%%%%%%
%%                               Functions                               %%
%%%%%%%%%%%%%%%%%%%%%%%%%%%%%%%%%%%%%%%%%%%%%%%%%%%%%%%%%%%%%%%%%%%%%%%%%%%

  \subsection{Functions}

    \label{pymc:distributions:bind_size}
    \index{pymc \textit{(package)}!pymc.distributions \textit{(module)}!pymc.distributions.bind\_size \textit{(function)}}

    \vspace{0.5ex}

    \begin{boxedminipage}{\textwidth}

    \raggedright \textbf{bind\_size}(\textit{randfun}, \textit{size})

    \end{boxedminipage}

    \label{pymc:distributions:new_dist_class}
    \index{pymc \textit{(package)}!pymc.distributions \textit{(module)}!pymc.distributions.new\_dist\_class \textit{(function)}}

    \vspace{0.5ex}

    \begin{boxedminipage}{\textwidth}

    \raggedright \textbf{new\_dist\_class}(*\textit{new\_class\_args})

    \vspace{-1.5ex}

    \rule{\textwidth}{0.5\fboxrule}

Returns a new class from a distribution.
    \vspace{1ex}

    \end{boxedminipage}

    \label{pymc:distributions:stochastic_from_dist}
    \index{pymc \textit{(package)}!pymc.distributions \textit{(module)}!pymc.distributions.stochastic\_from\_dist \textit{(function)}}

    \vspace{0.5ex}

    \begin{boxedminipage}{\textwidth}

    \raggedright \textbf{stochastic\_from\_dist}(\textit{name}, \textit{logp}, \textit{random}=\texttt{None}, \textit{dtype}=\texttt{np.float}, \textit{mv}=\texttt{False})

    \vspace{-1.5ex}

    \rule{\textwidth}{0.5\fboxrule}

Return a Stochastic subclass made from a particular distribution.
\begin{quote}
\end{quote}

Arguments can be passed in positionally; in this case, argument order is: self{\_}name, parents.
value is set to None by default, and can only be provided as a keyword argument.
    \vspace{1ex}

      \textbf{Example}
      \begin{quote}
          \begin{alltt}
\pysrcprompt{{\textgreater}{\textgreater}{\textgreater} }Exponential = stochastic\_from\_dist(\pysrcstring{'exponential'},
\pysrcoutput{                                        logp=exponential\_like,}
\pysrcoutput{                                        random=rexponential,}
\pysrcoutput{                                        dtype=np.float,}
\pysrcoutput{                                        mv=False)}
\pysrcoutput{}\pysrcprompt{{\textgreater}{\textgreater}{\textgreater} }A = Exponential(self\_name, value, beta)\end{alltt}
      \end{quote}

    \vspace{1ex}

    \end{boxedminipage}

    \label{pymc:distributions:Vectorize}
    \index{pymc \textit{(package)}!pymc.distributions \textit{(module)}!pymc.distributions.Vectorize \textit{(function)}}

    \vspace{0.5ex}

    \begin{boxedminipage}{\textwidth}

    \raggedright \textbf{Vectorize}(\textit{f})

    \vspace{-1.5ex}

    \rule{\textwidth}{0.5\fboxrule}

Wrapper to vectorize a scalar function.
    \vspace{1ex}

    \end{boxedminipage}

    \label{pymc:distributions:randomwrap}
    \index{pymc \textit{(package)}!pymc.distributions \textit{(module)}!pymc.distributions.randomwrap \textit{(function)}}

    \vspace{0.5ex}

    \begin{boxedminipage}{\textwidth}

    \raggedright \textbf{randomwrap}(\textit{func})

    \vspace{-1.5ex}

    \rule{\textwidth}{0.5\fboxrule}

Decorator for random value generators

Allows passing of sequence of parameters, as well as a size argument.

Convention:
\begin{quote}
\begin{itemize}
\item {} 
If size=1 and the parameters are all scalars, return a scalar.

\item {} 
If size=1, the random variates are 1D.

\item {} 
If the parameters are scalars and size {\textgreater} 1, the random variates are 1D.

\item {} 
If size {\textgreater} 1 and the parameters are sequences, the random variates are
aligned as (size, max(length)), where length is the parameters size.

\end{itemize}
\end{quote}
    \vspace{1ex}

      \textbf{Example}
      \begin{quote}
          \begin{alltt}
\pysrcprompt{{\textgreater}{\textgreater}{\textgreater} }rbernoulli(.1)
\pysrcoutput{0}
\pysrcoutput{}\pysrcprompt{{\textgreater}{\textgreater}{\textgreater} }rbernoulli([.1,.9])
\pysrcoutput{asarray([0, 1])}
\pysrcoutput{}\pysrcprompt{{\textgreater}{\textgreater}{\textgreater} }rbernoulli(.9, size=2)
\pysrcoutput{asarray([1, 1])}
\pysrcoutput{}\pysrcprompt{{\textgreater}{\textgreater}{\textgreater} }rbernoulli([.1,.9], 2)
\pysrcoutput{asarray([[0, 1],}
\pysrcoutput{       [0, 1]])}\end{alltt}
      \end{quote}

    \vspace{1ex}

    \end{boxedminipage}

    \label{pymc:distributions:debugwrapper}
    \index{pymc \textit{(package)}!pymc.distributions \textit{(module)}!pymc.distributions.debugwrapper \textit{(function)}}

    \vspace{0.5ex}

    \begin{boxedminipage}{\textwidth}

    \raggedright \textbf{debugwrapper}(\textit{func}, \textit{name})

    \end{boxedminipage}

    \label{pymc:distributions:constrain}
    \index{pymc \textit{(package)}!pymc.distributions \textit{(module)}!pymc.distributions.constrain \textit{(function)}}

    \vspace{0.5ex}

    \begin{boxedminipage}{\textwidth}

    \raggedright \textbf{constrain}(\textit{value}, \textit{lower}=\texttt{-np.Inf}, \textit{upper}=\texttt{np.Inf}, \textit{allow\_equal}=\texttt{False})

    \vspace{-1.5ex}

    \rule{\textwidth}{0.5\fboxrule}

Apply interval constraint on stochastic value.
    \vspace{1ex}

    \end{boxedminipage}

    \label{pymc:distributions:standardize}
    \index{pymc \textit{(package)}!pymc.distributions \textit{(module)}!pymc.distributions.standardize \textit{(function)}}

    \vspace{0.5ex}

    \begin{boxedminipage}{\textwidth}

    \raggedright \textbf{standardize}(\textit{x}, \textit{loc}=\texttt{0}, \textit{scale}=\texttt{1})

    \vspace{-1.5ex}

    \rule{\textwidth}{0.5\fboxrule}

Standardize x

Return (x-loc)/scale
    \vspace{1ex}

    \end{boxedminipage}

    \label{pymc:distributions:gammaln}
    \index{pymc \textit{(package)}!pymc.distributions \textit{(module)}!pymc.distributions.gammaln \textit{(function)}}

    \vspace{0.5ex}

    \begin{boxedminipage}{\textwidth}

    \raggedright \textbf{gammaln}(\textit{x})

    \vspace{-1.5ex}

    \rule{\textwidth}{0.5\fboxrule}

Logarithm of the Gamma function
    \vspace{1ex}

    \end{boxedminipage}

    \label{pymc:distributions:expand_triangular}
    \index{pymc \textit{(package)}!pymc.distributions \textit{(module)}!pymc.distributions.expand\_triangular \textit{(function)}}

    \vspace{0.5ex}

    \begin{boxedminipage}{\textwidth}

    \raggedright \textbf{expand\_triangular}(\textit{X}, \textit{k})

    \vspace{-1.5ex}

    \rule{\textwidth}{0.5\fboxrule}

Expand flattened triangular matrix.
    \vspace{1ex}

    \end{boxedminipage}

    \label{pymc:distributions:GOFpoints}
    \index{pymc \textit{(package)}!pymc.distributions \textit{(module)}!pymc.distributions.GOFpoints \textit{(function)}}

    \vspace{0.5ex}

    \begin{boxedminipage}{\textwidth}

    \raggedright \textbf{GOFpoints}(\textit{x}, \textit{y}, \textit{expval}, \textit{loss})

    \end{boxedminipage}

    \label{pymc:distributions:rarlognormal}
    \index{pymc \textit{(package)}!pymc.distributions \textit{(module)}!pymc.distributions.rarlognormal \textit{(function)}}

    \vspace{0.5ex}

    \begin{boxedminipage}{\textwidth}

    \raggedright \textbf{rarlognormal}(\textit{a}, \textit{sigma}, \textit{rho})

    \vspace{-1.5ex}

    \rule{\textwidth}{0.5\fboxrule}

Autoregressive normal random variates.

If a is a scalar, generates one series of length size.
If a is a sequence, generates size series of the same length
as a.
    \vspace{1ex}

    \end{boxedminipage}

    \label{pymc:distributions:arlognormal_like}
    \index{pymc \textit{(package)}!pymc.distributions \textit{(module)}!pymc.distributions.arlognormal\_like \textit{(function)}}

    \vspace{0.5ex}

    \begin{boxedminipage}{\textwidth}

    \raggedright \textbf{arlognormal\_like}(\textit{x}, \textit{a}, \textit{sigma}, \textit{rho}, \textit{beta}=\texttt{1})

    \vspace{-1.5ex}

    \rule{\textwidth}{0.5\fboxrule}

Autoregressive lognormal log-likelihood.
\begin{equation*}\begin{split}x_i & = a_i \exp(e_i) \\e_i & = \rho e_{i-1} + \epsilon_i\end{split}\end{equation*}
where $\epsilon_i \sim N(0,\sigma)$.
    \vspace{1ex}

    \end{boxedminipage}

    \label{pymc:distributions:rbernoulli}
    \index{pymc \textit{(package)}!pymc.distributions \textit{(module)}!pymc.distributions.rbernoulli \textit{(function)}}

    \vspace{0.5ex}

    \begin{boxedminipage}{\textwidth}

    \raggedright \textbf{rbernoulli}(\textit{p}, \textit{size}=\texttt{1})

    \vspace{-1.5ex}

    \rule{\textwidth}{0.5\fboxrule}

Random Bernoulli variates.
    \vspace{1ex}

    \end{boxedminipage}

    \label{pymc:distributions:bernoulli_expval}
    \index{pymc \textit{(package)}!pymc.distributions \textit{(module)}!pymc.distributions.bernoulli\_expval \textit{(function)}}

    \vspace{0.5ex}

    \begin{boxedminipage}{\textwidth}

    \raggedright \textbf{bernoulli\_expval}(\textit{p})

    \vspace{-1.5ex}

    \rule{\textwidth}{0.5\fboxrule}

Expected value of bernoulli distribution.
    \vspace{1ex}

    \end{boxedminipage}

    \label{pymc:distributions:bernoulli_like}
    \index{pymc \textit{(package)}!pymc.distributions \textit{(module)}!pymc.distributions.bernoulli\_like \textit{(function)}}

    \vspace{0.5ex}

    \begin{boxedminipage}{\textwidth}

    \raggedright \textbf{bernoulli\_like}(\textit{x}, \textit{p})

    \vspace{-1.5ex}

    \rule{\textwidth}{0.5\fboxrule}

Bernoulli log-likelihood

The Bernoulli distribution describes the probability of successes (x=1) and
failures (x=0).
\begin{equation*}\begin{split}f(x \mid p) = p^{x- 1} (1-p)^{1-x}\end{split}\end{equation*}    \vspace{1ex}

      \textbf{Parameters}
      \begin{quote}
        \begin{Ventry}{x}

          \item[x]


Series of successes (1) and failures (0). $x=0,1$
          \item[p]


Probability of success. $0 < p < 1$
        \end{Ventry}

      \end{quote}

    \vspace{1ex}

      \textbf{Example}
      \begin{quote}
          \begin{alltt}
\pysrcprompt{{\textgreater}{\textgreater}{\textgreater} }bernoulli\_like([0,1,0,1], .4)
\pysrcoutput{-2.8542325496673584}\end{alltt}
      \end{quote}

    \vspace{1ex}

\textbf{Note:} \begin{itemize}
\item {} 
$E(x)= p$

\item {} 
$Var(x)= p(1-p)$

\end{itemize}


    \end{boxedminipage}

    \label{pymc:distributions:rbeta}
    \index{pymc \textit{(package)}!pymc.distributions \textit{(module)}!pymc.distributions.rbeta \textit{(function)}}

    \vspace{0.5ex}

    \begin{boxedminipage}{\textwidth}

    \raggedright \textbf{rbeta}(\textit{alpha}, \textit{beta}, \textit{size}=\texttt{1})

    \vspace{-1.5ex}

    \rule{\textwidth}{0.5\fboxrule}

Random beta variates.
    \vspace{1ex}

    \end{boxedminipage}

    \label{pymc:distributions:beta_expval}
    \index{pymc \textit{(package)}!pymc.distributions \textit{(module)}!pymc.distributions.beta\_expval \textit{(function)}}

    \vspace{0.5ex}

    \begin{boxedminipage}{\textwidth}

    \raggedright \textbf{beta\_expval}(\textit{alpha}, \textit{beta})

    \vspace{-1.5ex}

    \rule{\textwidth}{0.5\fboxrule}

Expected value of beta distribution.
    \vspace{1ex}

    \end{boxedminipage}

    \label{pymc:distributions:beta_like}
    \index{pymc \textit{(package)}!pymc.distributions \textit{(module)}!pymc.distributions.beta\_like \textit{(function)}}

    \vspace{0.5ex}

    \begin{boxedminipage}{\textwidth}

    \raggedright \textbf{beta\_like}(\textit{x}, \textit{alpha}, \textit{beta})

    \vspace{-1.5ex}

    \rule{\textwidth}{0.5\fboxrule}

Beta log-likelihood.
\begin{equation*}\begin{split}f(x \mid \alpha, \beta) = \frac{\Gamma(\alpha + \beta)}{\Gamma(\alpha) \Gamma(\beta)} x^{\alpha - 1} (1 - x)^{\beta - 1}\end{split}\end{equation*}    \vspace{1ex}

      \textbf{Parameters}
      \begin{quote}
        \begin{Ventry}{xxxxx}

          \item[x]


0 {\textless} x {\textless} 1
            \textit{(type=float)}

          \item[alpha]


{\textgreater} 0
            \textit{(type=float)}

          \item[beta]


{\textgreater} 0
            \textit{(type=float)}

        \end{Ventry}

      \end{quote}

    \vspace{1ex}

      \textbf{Example}
      \begin{quote}
          \begin{alltt}
\pysrcprompt{{\textgreater}{\textgreater}{\textgreater} }beta\_like(.4,1,2)
\pysrcoutput{0.18232160806655884}\end{alltt}
      \end{quote}

    \vspace{1ex}

\textbf{Note:} \begin{itemize}
\item {} 
$E(X)=\frac{\alpha}{\alpha+\beta}$

\item {} 
$Var(X)=\frac{\alpha \beta}{(\alpha+\beta)^2(\alpha+\beta+1)}$

\end{itemize}


    \end{boxedminipage}

    \label{pymc:distributions:rbinomial}
    \index{pymc \textit{(package)}!pymc.distributions \textit{(module)}!pymc.distributions.rbinomial \textit{(function)}}

    \vspace{0.5ex}

    \begin{boxedminipage}{\textwidth}

    \raggedright \textbf{rbinomial}(\textit{n}, \textit{p}, \textit{size}=\texttt{1})

    \vspace{-1.5ex}

    \rule{\textwidth}{0.5\fboxrule}

Random binomial variates.
    \vspace{1ex}

    \end{boxedminipage}

    \label{pymc:distributions:binomial_expval}
    \index{pymc \textit{(package)}!pymc.distributions \textit{(module)}!pymc.distributions.binomial\_expval \textit{(function)}}

    \vspace{0.5ex}

    \begin{boxedminipage}{\textwidth}

    \raggedright \textbf{binomial\_expval}(\textit{n}, \textit{p})

    \vspace{-1.5ex}

    \rule{\textwidth}{0.5\fboxrule}

Expected value of binomial distribution.
    \vspace{1ex}

    \end{boxedminipage}

    \label{pymc:distributions:binomial_like}
    \index{pymc \textit{(package)}!pymc.distributions \textit{(module)}!pymc.distributions.binomial\_like \textit{(function)}}

    \vspace{0.5ex}

    \begin{boxedminipage}{\textwidth}

    \raggedright \textbf{binomial\_like}(\textit{x}, \textit{n}, \textit{p})

    \vspace{-1.5ex}

    \rule{\textwidth}{0.5\fboxrule}

Binomial log-likelihood.  The discrete probability distribution of the
number of successes in a sequence of n independent yes/no experiments,
each of which yields success with probability p.
\begin{equation*}\begin{split}f(x \mid n, p) = \frac{n!}{x!(n-x)!} p^x (1-p)^{1-x}\end{split}\end{equation*}    \vspace{1ex}

      \textbf{Parameters}
      \begin{quote}
        \begin{Ventry}{x}

          \item[x]


Number of successes, {\textgreater} 0.
            \textit{(type=float)}

          \item[n]


Number of Bernoulli trials, {\textgreater} x.
            \textit{(type=int)}

          \item[p]


Probability of success in each trial, $p \in [0,1]$.
            \textit{(type=float)}

        \end{Ventry}

      \end{quote}

    \vspace{1ex}

\textbf{Note:} \begin{itemize}
\item {} 
$E(X)=np$

\item {} 
$Var(X)=np(1-p)$

\end{itemize}


    \end{boxedminipage}

    \label{pymc:distributions:rcategorical}
    \index{pymc \textit{(package)}!pymc.distributions \textit{(module)}!pymc.distributions.rcategorical \textit{(function)}}

    \vspace{0.5ex}

    \begin{boxedminipage}{\textwidth}

    \raggedright \textbf{rcategorical}(\textit{p}, \textit{minval}=\texttt{0}, \textit{step}=\texttt{1}, \textit{size}=\texttt{1})

    \end{boxedminipage}

    \label{pymc:distributions:categorical_expval}
    \index{pymc \textit{(package)}!pymc.distributions \textit{(module)}!pymc.distributions.categorical\_expval \textit{(function)}}

    \vspace{0.5ex}

    \begin{boxedminipage}{\textwidth}

    \raggedright \textbf{categorical\_expval}(\textit{p}, \textit{minval}=\texttt{0}, \textit{step}=\texttt{1})

    \vspace{-1.5ex}

    \rule{\textwidth}{0.5\fboxrule}

Expected value of categorical distribution.
    \vspace{1ex}

    \end{boxedminipage}

    \label{pymc:distributions:categorical_like}
    \index{pymc \textit{(package)}!pymc.distributions \textit{(module)}!pymc.distributions.categorical\_like \textit{(function)}}

    \vspace{0.5ex}

    \begin{boxedminipage}{\textwidth}

    \raggedright \textbf{categorical\_like}(\textit{x}, \textit{p}, \textit{minval}=\texttt{0}, \textit{step}=\texttt{1})

    \vspace{-1.5ex}

    \rule{\textwidth}{0.5\fboxrule}

Categorical log-likelihood.
Accepts an array of probabilities associated with the histogram,
the minimum value of the histogram (defaults to zero),
and a step size (defaults to 1).
    \vspace{1ex}

    \end{boxedminipage}

    \label{pymc:distributions:rcauchy}
    \index{pymc \textit{(package)}!pymc.distributions \textit{(module)}!pymc.distributions.rcauchy \textit{(function)}}

    \vspace{0.5ex}

    \begin{boxedminipage}{\textwidth}

    \raggedright \textbf{rcauchy}(\textit{alpha}, \textit{beta}, \textit{size}=\texttt{1})

    \vspace{-1.5ex}

    \rule{\textwidth}{0.5\fboxrule}

Returns Cauchy random variates.
    \vspace{1ex}

    \end{boxedminipage}

    \label{pymc:distributions:cauchy_expval}
    \index{pymc \textit{(package)}!pymc.distributions \textit{(module)}!pymc.distributions.cauchy\_expval \textit{(function)}}

    \vspace{0.5ex}

    \begin{boxedminipage}{\textwidth}

    \raggedright \textbf{cauchy\_expval}(\textit{alpha}, \textit{beta})

    \vspace{-1.5ex}

    \rule{\textwidth}{0.5\fboxrule}

Expected value of cauchy distribution.
    \vspace{1ex}

    \end{boxedminipage}

    \label{pymc:distributions:cauchy_like}
    \index{pymc \textit{(package)}!pymc.distributions \textit{(module)}!pymc.distributions.cauchy\_like \textit{(function)}}

    \vspace{0.5ex}

    \begin{boxedminipage}{\textwidth}

    \raggedright \textbf{cauchy\_like}(\textit{x}, \textit{alpha}, \textit{beta})

    \vspace{-1.5ex}

    \rule{\textwidth}{0.5\fboxrule}

Cauchy log-likelihood. The Cauchy distribution is also known as the
Lorentz or the Breit-Wigner distribution.
\begin{equation*}\begin{split}f(x \mid \alpha, \beta) = \frac{1}{\pi \beta [1 + (\frac{x-\alpha}{\beta})^2]}\end{split}\end{equation*}    \vspace{1ex}

      \textbf{Parameters}
      \begin{quote}
        \begin{Ventry}{xxxxx}

          \item[alpha]


Location parameter.
            \textit{(type=float)}

          \item[beta]


Scale parameter {\textgreater} 0.
            \textit{(type=float)}

        \end{Ventry}

      \end{quote}

    \vspace{1ex}

\textbf{Note:} \begin{itemize}
\item {} 
Mode and median are at alpha.

\end{itemize}


    \end{boxedminipage}

    \label{pymc:distributions:rchi2}
    \index{pymc \textit{(package)}!pymc.distributions \textit{(module)}!pymc.distributions.rchi2 \textit{(function)}}

    \vspace{0.5ex}

    \begin{boxedminipage}{\textwidth}

    \raggedright \textbf{rchi2}(\textit{nu}, \textit{size}=\texttt{1})

    \vspace{-1.5ex}

    \rule{\textwidth}{0.5\fboxrule}

Random :math:'chi{\textasciicircum}2' variates.
    \vspace{1ex}

    \end{boxedminipage}

    \label{pymc:distributions:chi2_expval}
    \index{pymc \textit{(package)}!pymc.distributions \textit{(module)}!pymc.distributions.chi2\_expval \textit{(function)}}

    \vspace{0.5ex}

    \begin{boxedminipage}{\textwidth}

    \raggedright \textbf{chi2\_expval}(\textit{nu})

    \vspace{-1.5ex}

    \rule{\textwidth}{0.5\fboxrule}

Expected value of Chi-squared distribution.
    \vspace{1ex}

    \end{boxedminipage}

    \label{pymc:distributions:chi2_like}
    \index{pymc \textit{(package)}!pymc.distributions \textit{(module)}!pymc.distributions.chi2\_like \textit{(function)}}

    \vspace{0.5ex}

    \begin{boxedminipage}{\textwidth}

    \raggedright \textbf{chi2\_like}(\textit{x}, \textit{nu})

    \vspace{-1.5ex}

    \rule{\textwidth}{0.5\fboxrule}

Chi-squared $\chi^2$ log-likelihood.
\begin{equation*}\begin{split}f(x \mid \nu) = \frac{x^{(\nu-2)/2}e^{-x/2}}{2^{\nu/2}\Gamma(\nu/2)}\end{split}\end{equation*}    \vspace{1ex}

      \textbf{Parameters}
      \begin{quote}
        \begin{Ventry}{xx}

          \item[x]


$\ge 0$
            \textit{(type=float)}

          \item[nu]


Degrees of freedom ( $nu > 0$)
            \textit{(type=int)}

        \end{Ventry}

      \end{quote}

    \vspace{1ex}

\textbf{Note:} \begin{itemize}
\item {} 
$E(X)=\nu$

\item {} 
$Var(X)=2\nu$

\end{itemize}


    \end{boxedminipage}

    \label{pymc:distributions:rdirichlet}
    \index{pymc \textit{(package)}!pymc.distributions \textit{(module)}!pymc.distributions.rdirichlet \textit{(function)}}

    \vspace{0.5ex}

    \begin{boxedminipage}{\textwidth}

    \raggedright \textbf{rdirichlet}(\textit{theta}, \textit{size}=\texttt{1})

    \vspace{-1.5ex}

    \rule{\textwidth}{0.5\fboxrule}

Dirichlet random variates.

NOTE only the first k-1 values are returned.
The k'th value is equal to one minus the sum of the first k-1 values.
    \vspace{1ex}

    \end{boxedminipage}

    \label{pymc:distributions:dirichlet_expval}
    \index{pymc \textit{(package)}!pymc.distributions \textit{(module)}!pymc.distributions.dirichlet\_expval \textit{(function)}}

    \vspace{0.5ex}

    \begin{boxedminipage}{\textwidth}

    \raggedright \textbf{dirichlet\_expval}(\textit{theta})

    \vspace{-1.5ex}

    \rule{\textwidth}{0.5\fboxrule}

Expected value of Dirichlet distribution.

NOTE only the expectations of the first k-1 values are returned.
The expectation of the k'th value is one minus the expectations of
the first k-1 values.
    \vspace{1ex}

    \end{boxedminipage}

    \label{pymc:distributions:dirichlet_like}
    \index{pymc \textit{(package)}!pymc.distributions \textit{(module)}!pymc.distributions.dirichlet\_like \textit{(function)}}

    \vspace{0.5ex}

    \begin{boxedminipage}{\textwidth}

    \raggedright \textbf{dirichlet\_like}(\textit{x}, \textit{theta})

    \vspace{-1.5ex}

    \rule{\textwidth}{0.5\fboxrule}

Dirichlet log-likelihood.

This is a multivariate continuous distribution.
\begin{equation*}\begin{split}f(\mathbf{x}) = \frac{\Gamma(\sum_{i=1}^k \theta_i)}{\prod \Gamma(\theta_i)} \prod_{i=1}^k x_i^{\theta_i - 1}\end{split}\end{equation*}    \vspace{1ex}

      \textbf{Parameters}
      \begin{quote}
        \begin{Ventry}{xxxxx}

          \item[x]


Where \texttt{n} is the number of samples and \texttt{k} the dimension.
$0 < x_i < 1$,  $\sum_{i=1}^{k-1} x_i < 1$
            \textit{(type=(n,k-1) array)}

          \item[theta]


$\theta > 0$
            \textit{(type=(n,k) or (1,k) float)}

        \end{Ventry}

      \end{quote}

    \vspace{1ex}

\textbf{Note:} 
There is an \texttt{implicit} k'th value of x, equal to $\sum_{i=1}^{k-1} x_i$.


    \end{boxedminipage}

    \label{pymc:distributions:rexponential}
    \index{pymc \textit{(package)}!pymc.distributions \textit{(module)}!pymc.distributions.rexponential \textit{(function)}}

    \vspace{0.5ex}

    \begin{boxedminipage}{\textwidth}

    \raggedright \textbf{rexponential}(\textit{beta})

    \vspace{-1.5ex}

    \rule{\textwidth}{0.5\fboxrule}

Exponential random variates.
    \vspace{1ex}

    \end{boxedminipage}

    \label{pymc:distributions:exponential_expval}
    \index{pymc \textit{(package)}!pymc.distributions \textit{(module)}!pymc.distributions.exponential\_expval \textit{(function)}}

    \vspace{0.5ex}

    \begin{boxedminipage}{\textwidth}

    \raggedright \textbf{exponential\_expval}(\textit{beta})

    \vspace{-1.5ex}

    \rule{\textwidth}{0.5\fboxrule}

Expected value of exponential distribution.
    \vspace{1ex}

    \end{boxedminipage}

    \label{pymc:distributions:exponential_like}
    \index{pymc \textit{(package)}!pymc.distributions \textit{(module)}!pymc.distributions.exponential\_like \textit{(function)}}

    \vspace{0.5ex}

    \begin{boxedminipage}{\textwidth}

    \raggedright \textbf{exponential\_like}(\textit{x}, \textit{beta})

    \vspace{-1.5ex}

    \rule{\textwidth}{0.5\fboxrule}

Exponential log-likelihood.

The exponential distribution is a special case of the gamma distribution
with alpha=1. It often describes the duration of an event.
\begin{equation*}\begin{split}f(x \mid \beta) = \frac{1}{\beta}e^{-x/\beta}\end{split}\end{equation*}    \vspace{1ex}

      \textbf{Parameters}
      \begin{quote}
        \begin{Ventry}{xxxx}

          \item[x]


$x \ge 0$
            \textit{(type=float)}

          \item[beta]


Survival parameter $\beta > 0$
            \textit{(type=float)}

        \end{Ventry}

      \end{quote}

    \vspace{1ex}

\textbf{Note:} \begin{itemize}
\item {} 
$E(X) = \beta$

\item {} 
$Var(X) = \beta^2$

\end{itemize}


    \end{boxedminipage}

    \label{pymc:distributions:rexponweib}
    \index{pymc \textit{(package)}!pymc.distributions \textit{(module)}!pymc.distributions.rexponweib \textit{(function)}}

    \vspace{0.5ex}

    \begin{boxedminipage}{\textwidth}

    \raggedright \textbf{rexponweib}(\textit{alpha}, \textit{k}, \textit{loc}=\texttt{0}, \textit{scale}=\texttt{1}, \textit{size}=\texttt{1})

    \vspace{-1.5ex}

    \rule{\textwidth}{0.5\fboxrule}

Random exponentiated Weibull variates.
    \vspace{1ex}

    \end{boxedminipage}

    \label{pymc:distributions:exponweib_expval}
    \index{pymc \textit{(package)}!pymc.distributions \textit{(module)}!pymc.distributions.exponweib\_expval \textit{(function)}}

    \vspace{0.5ex}

    \begin{boxedminipage}{\textwidth}

    \raggedright \textbf{exponweib\_expval}(\textit{alpha}, \textit{k}, \textit{loc}, \textit{scale})

    \end{boxedminipage}

    \label{pymc:distributions:exponweib_like}
    \index{pymc \textit{(package)}!pymc.distributions \textit{(module)}!pymc.distributions.exponweib\_like \textit{(function)}}

    \vspace{0.5ex}

    \begin{boxedminipage}{\textwidth}

    \raggedright \textbf{exponweib\_like}(\textit{x}, \textit{alpha}, \textit{k}, \textit{loc}=\texttt{0}, \textit{scale}=\texttt{1})

    \vspace{-1.5ex}

    \rule{\textwidth}{0.5\fboxrule}

Exponentiated Weibull log-likelihood.
\begin{equation*}\begin{split}f(x \mid \alpha,k,loc,scale)  & = \frac{\alpha k}{scale} (1-e^{-z^c})^{\alpha-1} e^{-z^c} z^{k-1} \\z & = \frac{x-loc}{scale}\end{split}\end{equation*}    \vspace{1ex}

      \textbf{Parameters}
      \begin{quote}
        \begin{Ventry}{xxxxx}

          \item[x]


{\textgreater} 0
            \textit{(type=float)}

          \item[alpha]


Shape parameter
            \textit{(type=float)}

          \item[k]


{\textgreater} 0
            \textit{(type=float)}

          \item[loc]


Location parameter
            \textit{(type=float)}

          \item[scale]


Scale parameter {\textgreater} 0.
            \textit{(type=float)}

        \end{Ventry}

      \end{quote}

    \vspace{1ex}

    \end{boxedminipage}

    \label{pymc:distributions:rgamma}
    \index{pymc \textit{(package)}!pymc.distributions \textit{(module)}!pymc.distributions.rgamma \textit{(function)}}

    \vspace{0.5ex}

    \begin{boxedminipage}{\textwidth}

    \raggedright \textbf{rgamma}(\textit{alpha}, \textit{beta}, \textit{size}=\texttt{1})

    \vspace{-1.5ex}

    \rule{\textwidth}{0.5\fboxrule}

Random gamma variates.
    \vspace{1ex}

    \end{boxedminipage}

    \label{pymc:distributions:gamma_expval}
    \index{pymc \textit{(package)}!pymc.distributions \textit{(module)}!pymc.distributions.gamma\_expval \textit{(function)}}

    \vspace{0.5ex}

    \begin{boxedminipage}{\textwidth}

    \raggedright \textbf{gamma\_expval}(\textit{alpha}, \textit{beta})

    \vspace{-1.5ex}

    \rule{\textwidth}{0.5\fboxrule}

Expected value of gamma distribution.
    \vspace{1ex}

    \end{boxedminipage}

    \label{pymc:distributions:gamma_like}
    \index{pymc \textit{(package)}!pymc.distributions \textit{(module)}!pymc.distributions.gamma\_like \textit{(function)}}

    \vspace{0.5ex}

    \begin{boxedminipage}{\textwidth}

    \raggedright \textbf{gamma\_like}(\textit{x}, \textit{alpha}, \textit{beta})

    \vspace{-1.5ex}

    \rule{\textwidth}{0.5\fboxrule}

Gamma log-likelihood.

Represents the sum of alpha exponentially distributed random variables, each
of which has mean beta.
\begin{equation*}\begin{split}f(x \mid \alpha, \beta) = \frac{\beta^{\alpha}x^{\alpha-1}e^{-\beta x}}{\Gamma(\alpha)}\end{split}\end{equation*}    \vspace{1ex}

      \textbf{Parameters}
      \begin{quote}
        \begin{Ventry}{xxxxx}

          \item[x]


$x \ge 0$
            \textit{(type=float)}

          \item[alpha]


Shape parameter $\alpha > 0$.
            \textit{(type=float)}

          \item[beta]


Scale parameter $\beta > 0$.
            \textit{(type=float)}

        \end{Ventry}

      \end{quote}

    \vspace{1ex}

    \end{boxedminipage}

    \label{pymc:distributions:rgev}
    \index{pymc \textit{(package)}!pymc.distributions \textit{(module)}!pymc.distributions.rgev \textit{(function)}}

    \vspace{0.5ex}

    \begin{boxedminipage}{\textwidth}

    \raggedright \textbf{rgev}(\textit{xi}, \textit{mu}=\texttt{0}, \textit{sigma}=\texttt{0}, \textit{size}=\texttt{1})

    \vspace{-1.5ex}

    \rule{\textwidth}{0.5\fboxrule}

Random generalized extreme value (GEV) variates.
    \vspace{1ex}

    \end{boxedminipage}

    \label{pymc:distributions:gev_expval}
    \index{pymc \textit{(package)}!pymc.distributions \textit{(module)}!pymc.distributions.gev\_expval \textit{(function)}}

    \vspace{0.5ex}

    \begin{boxedminipage}{\textwidth}

    \raggedright \textbf{gev\_expval}(\textit{xi}, \textit{mu}=\texttt{0}, \textit{sigma}=\texttt{1})

    \vspace{-1.5ex}

    \rule{\textwidth}{0.5\fboxrule}

Expected value of generalized extreme value distribution.
    \vspace{1ex}

    \end{boxedminipage}

    \label{pymc:distributions:gev_like}
    \index{pymc \textit{(package)}!pymc.distributions \textit{(module)}!pymc.distributions.gev\_like \textit{(function)}}

    \vspace{0.5ex}

    \begin{boxedminipage}{\textwidth}

    \raggedright \textbf{gev\_like}(\textit{x}, \textit{xi}, \textit{mu}=\texttt{0}, \textit{sigma}=\texttt{1})

    \vspace{-1.5ex}

    \rule{\textwidth}{0.5\fboxrule}

Generalized Extreme Value log-likelihood
\begin{equation*}\begin{split}pdf(x \mid \xi,\mu,\sigma) = \frac{1}{\sigma}(1 + \xi \left[\frac{x-\mu}{\sigma}\right])^{-1/\xi-1}\exp{-(1+\xi \left[\frac{x-\mu}{\sigma}\right])^{-1/\xi}}\end{split}\end{equation*}\begin{equation*}\begin{split}\sigma & > 0,\\x & > \mu-\sigma/\xi \text{ if } \xi > 0,\\x & < \mu-\sigma/\xi \text{ if } \xi < 0\\x & \in [-\infty,\infty] \text{ if } \xi = 0\end{split}\end{equation*}    \vspace{1ex}

    \end{boxedminipage}

    \label{pymc:distributions:rgeometric}
    \index{pymc \textit{(package)}!pymc.distributions \textit{(module)}!pymc.distributions.rgeometric \textit{(function)}}

    \vspace{0.5ex}

    \begin{boxedminipage}{\textwidth}

    \raggedright \textbf{rgeometric}(\textit{p}, \textit{size}=\texttt{1})

    \vspace{-1.5ex}

    \rule{\textwidth}{0.5\fboxrule}

Random geometric variates.
    \vspace{1ex}

    \end{boxedminipage}

    \label{pymc:distributions:geometric_expval}
    \index{pymc \textit{(package)}!pymc.distributions \textit{(module)}!pymc.distributions.geometric\_expval \textit{(function)}}

    \vspace{0.5ex}

    \begin{boxedminipage}{\textwidth}

    \raggedright \textbf{geometric\_expval}(\textit{p})

    \vspace{-1.5ex}

    \rule{\textwidth}{0.5\fboxrule}

Expected value of geometric distribution.
    \vspace{1ex}

    \end{boxedminipage}

    \label{pymc:distributions:geometric_like}
    \index{pymc \textit{(package)}!pymc.distributions \textit{(module)}!pymc.distributions.geometric\_like \textit{(function)}}

    \vspace{0.5ex}

    \begin{boxedminipage}{\textwidth}

    \raggedright \textbf{geometric\_like}(\textit{x}, \textit{p})

    \vspace{-1.5ex}

    \rule{\textwidth}{0.5\fboxrule}

Geometric log-likelihood. The probability that the first success in a
sequence of Bernoulli trials occurs after x trials.
\begin{equation*}\begin{split}f(x \mid p) = p(1-p)^{x-1}\end{split}\end{equation*}    \vspace{1ex}

      \textbf{Parameters}
      \begin{quote}
        \begin{Ventry}{x}

          \item[x]


Number of trials before first success, {\textgreater} 0.
            \textit{(type=int)}

          \item[p]


Probability of success on an individual trial, :math:'p in {[}0,1{]}'
            \textit{(type=float)}

        \end{Ventry}

      \end{quote}

    \vspace{1ex}

\textbf{Note:} \begin{itemize}
\item {} 
:math:'E(X)=1/p'

\item {} 
:math:'Var(X)=frac{\{}1-p{\}}{\{}p{\textasciicircum}2{\}}'

\end{itemize}


    \end{boxedminipage}

    \label{pymc:distributions:rhalf_normal}
    \index{pymc \textit{(package)}!pymc.distributions \textit{(module)}!pymc.distributions.rhalf\_normal \textit{(function)}}

    \vspace{0.5ex}

    \begin{boxedminipage}{\textwidth}

    \raggedright \textbf{rhalf\_normal}(\textit{tau}, \textit{size}=\texttt{1})

    \vspace{-1.5ex}

    \rule{\textwidth}{0.5\fboxrule}

Random half-normal variates.
    \vspace{1ex}

    \end{boxedminipage}

    \label{pymc:distributions:half_normal_expval}
    \index{pymc \textit{(package)}!pymc.distributions \textit{(module)}!pymc.distributions.half\_normal\_expval \textit{(function)}}

    \vspace{0.5ex}

    \begin{boxedminipage}{\textwidth}

    \raggedright \textbf{half\_normal\_expval}(\textit{tau})

    \vspace{-1.5ex}

    \rule{\textwidth}{0.5\fboxrule}

Expected value of half normal distribution.
    \vspace{1ex}

    \end{boxedminipage}

    \label{pymc:distributions:half_normal_like}
    \index{pymc \textit{(package)}!pymc.distributions \textit{(module)}!pymc.distributions.half\_normal\_like \textit{(function)}}

    \vspace{0.5ex}

    \begin{boxedminipage}{\textwidth}

    \raggedright \textbf{half\_normal\_like}(\textit{x}, \textit{tau})

    \vspace{-1.5ex}

    \rule{\textwidth}{0.5\fboxrule}

Half-normal log-likelihood, a normal distribution with mean 0 and limited
to the domain :math:'x in {[}0, infty)'.
\begin{equation*}\begin{split}f(x \mid \tau) = \sqrt{\frac{2\tau}{\pi}}\exp\left\{ {\frac{-x^2 \tau}{2}}\right\}\end{split}\end{equation*}    \vspace{1ex}

      \textbf{Parameters}
      \begin{quote}
        \begin{Ventry}{xxx}

          \item[x]


:math:'x ge 0'
            \textit{(type=float)}

          \item[tau]


:math:'tau {\textgreater} 0'
            \textit{(type=float)}

        \end{Ventry}

      \end{quote}

    \vspace{1ex}

    \end{boxedminipage}

    \label{pymc:distributions:rhypergeometric}
    \index{pymc \textit{(package)}!pymc.distributions \textit{(module)}!pymc.distributions.rhypergeometric \textit{(function)}}

    \vspace{0.5ex}

    \begin{boxedminipage}{\textwidth}

    \raggedright \textbf{rhypergeometric}(\textit{n}, \textit{m}, \textit{N}, \textit{size}=\texttt{1})

    \vspace{-1.5ex}

    \rule{\textwidth}{0.5\fboxrule}

Returns hypergeometric random variates.
    \vspace{1ex}

    \end{boxedminipage}

    \label{pymc:distributions:hypergeometric_expval}
    \index{pymc \textit{(package)}!pymc.distributions \textit{(module)}!pymc.distributions.hypergeometric\_expval \textit{(function)}}

    \vspace{0.5ex}

    \begin{boxedminipage}{\textwidth}

    \raggedright \textbf{hypergeometric\_expval}(\textit{n}, \textit{m}, \textit{N})

    \vspace{-1.5ex}

    \rule{\textwidth}{0.5\fboxrule}

Expected value of hypergeometric distribution.
    \vspace{1ex}

    \end{boxedminipage}

    \label{pymc:distributions:hypergeometric_like}
    \index{pymc \textit{(package)}!pymc.distributions \textit{(module)}!pymc.distributions.hypergeometric\_like \textit{(function)}}

    \vspace{0.5ex}

    \begin{boxedminipage}{\textwidth}

    \raggedright \textbf{hypergeometric\_like}(\textit{x}, \textit{n}, \textit{m}, \textit{N})

    \vspace{-1.5ex}

    \rule{\textwidth}{0.5\fboxrule}
\begin{alltt}
Hypergeometric log-likelihood. Discrete probability distribution that
describes the number of successes in a sequence of draws from a finite
population without replacement.

.. math::
    f(x {\textbackslash}mid n, m, N) = {\textbackslash}frac\{{\textbackslash}binom\{m\}\{x\}{\textbackslash}binom\{N-m\}\{n-x\}\}\{{\textbackslash}binom\{N\}\{n\}\}

:Parameters:
  x : int
    Number of successes in a sample drawn from a population.
    :math:`{\textbackslash}max(0, draws-failures) {\textbackslash}leq x {\textbackslash}leq {\textbackslash}min(draws, success)`
  n : int
    Size of sample drawn from the population.
  m : int
    Number of successes in the population.
  N : int
    Total number of units in the population.

:Note:
  :math:`E(X) = {\textbackslash}frac\{n n\}\{N\}`
\end{alltt}

    \vspace{1ex}

    \end{boxedminipage}

    \label{pymc:distributions:rinverse_gamma}
    \index{pymc \textit{(package)}!pymc.distributions \textit{(module)}!pymc.distributions.rinverse\_gamma \textit{(function)}}

    \vspace{0.5ex}

    \begin{boxedminipage}{\textwidth}

    \raggedright \textbf{rinverse\_gamma}(\textit{alpha}, \textit{beta}, \textit{size}=\texttt{1})

    \vspace{-1.5ex}

    \rule{\textwidth}{0.5\fboxrule}

Random inverse gamma variates.
    \vspace{1ex}

    \end{boxedminipage}

    \label{pymc:distributions:inverse_gamma_expval}
    \index{pymc \textit{(package)}!pymc.distributions \textit{(module)}!pymc.distributions.inverse\_gamma\_expval \textit{(function)}}

    \vspace{0.5ex}

    \begin{boxedminipage}{\textwidth}

    \raggedright \textbf{inverse\_gamma\_expval}(\textit{alpha}, \textit{beta})

    \vspace{-1.5ex}

    \rule{\textwidth}{0.5\fboxrule}

Expected value of inverse gamma distribution.
    \vspace{1ex}

    \end{boxedminipage}

    \label{pymc:distributions:inverse_gamma_like}
    \index{pymc \textit{(package)}!pymc.distributions \textit{(module)}!pymc.distributions.inverse\_gamma\_like \textit{(function)}}

    \vspace{0.5ex}

    \begin{boxedminipage}{\textwidth}

    \raggedright \textbf{inverse\_gamma\_like}(\textit{x}, \textit{alpha}, \textit{beta})

    \vspace{-1.5ex}

    \rule{\textwidth}{0.5\fboxrule}

Inverse gamma log-likelihood, the reciprocal of the gamma distribution.
\begin{equation*}\begin{split}f(x \mid \alpha, \beta) = \frac{\beta^{\alpha}}{\Gamma(\alpha)} x^{-\alpha - 1} \exp\left(\frac{-\beta}{x}\right)\end{split}\end{equation*}    \vspace{1ex}

      \textbf{Parameters}
      \begin{quote}
        \begin{Ventry}{xxxxx}

          \item[x]


x {\textgreater} 0
            \textit{(type=float)}

          \item[alpha]


Shape parameter, $\alpha > 0$.
            \textit{(type=float)}

          \item[beta]


Scale parameter, $\beta > 0$.
            \textit{(type=float)}

        \end{Ventry}

      \end{quote}

    \vspace{1ex}

\textbf{Note:} 
$E(X)=\frac{1}{\beta(\alpha-1)}$  for $\alpha > 1$.


    \end{boxedminipage}

    \label{pymc:distributions:rlognormal}
    \index{pymc \textit{(package)}!pymc.distributions \textit{(module)}!pymc.distributions.rlognormal \textit{(function)}}

    \vspace{0.5ex}

    \begin{boxedminipage}{\textwidth}

    \raggedright \textbf{rlognormal}(\textit{mu}, \textit{tau}, \textit{size}=\texttt{1})

    \vspace{-1.5ex}

    \rule{\textwidth}{0.5\fboxrule}

Return random lognormal variates.
    \vspace{1ex}

    \end{boxedminipage}

    \label{pymc:distributions:lognormal_expval}
    \index{pymc \textit{(package)}!pymc.distributions \textit{(module)}!pymc.distributions.lognormal\_expval \textit{(function)}}

    \vspace{0.5ex}

    \begin{boxedminipage}{\textwidth}

    \raggedright \textbf{lognormal\_expval}(\textit{mu}, \textit{tau})

    \vspace{-1.5ex}

    \rule{\textwidth}{0.5\fboxrule}

Expected value of log-normal distribution.
    \vspace{1ex}

    \end{boxedminipage}

    \label{pymc:distributions:lognormal_like}
    \index{pymc \textit{(package)}!pymc.distributions \textit{(module)}!pymc.distributions.lognormal\_like \textit{(function)}}

    \vspace{0.5ex}

    \begin{boxedminipage}{\textwidth}

    \raggedright \textbf{lognormal\_like}(\textit{x}, \textit{mu}, \textit{tau})

    \vspace{-1.5ex}

    \rule{\textwidth}{0.5\fboxrule}

Log-normal log-likelihood. Distribution of any random variable whose
logarithm is normally distributed. A variable might be modeled as
log-normal if it can be thought of as the multiplicative product of many
small independent factors.
\begin{equation*}\begin{split}f(x \mid \mu, \tau) = \sqrt{\frac{\tau}{2\pi}}\frac{\exp\left\{ -\frac{\tau}{2} (\ln(x)-\mu)^2 \right\}}{x}\end{split}\end{equation*}    \vspace{1ex}

      \textbf{Parameters}
      \begin{quote}
        \begin{Ventry}{xxx}

          \item[x]


x {\textgreater} 0
            \textit{(type=float)}

          \item[mu]


Location parameter.
            \textit{(type=float)}

          \item[tau]


Scale parameter, {\textgreater} 0.
            \textit{(type=float)}

        \end{Ventry}

      \end{quote}

    \vspace{1ex}

\textbf{Note:} 
$E(X)=e^{\mu+\frac{1}{2\tau}}$


    \end{boxedminipage}

    \label{pymc:distributions:rmultinomial}
    \index{pymc \textit{(package)}!pymc.distributions \textit{(module)}!pymc.distributions.rmultinomial \textit{(function)}}

    \vspace{0.5ex}

    \begin{boxedminipage}{\textwidth}

    \raggedright \textbf{rmultinomial}(\textit{n}, \textit{p}, \textit{size}=\texttt{1})

    \vspace{-1.5ex}

    \rule{\textwidth}{0.5\fboxrule}

Random multinomial variates.
    \vspace{1ex}

    \end{boxedminipage}

    \label{pymc:distributions:multinomial_expval}
    \index{pymc \textit{(package)}!pymc.distributions \textit{(module)}!pymc.distributions.multinomial\_expval \textit{(function)}}

    \vspace{0.5ex}

    \begin{boxedminipage}{\textwidth}

    \raggedright \textbf{multinomial\_expval}(\textit{n}, \textit{p})

    \vspace{-1.5ex}

    \rule{\textwidth}{0.5\fboxrule}

Expected value of multinomial distribution.
    \vspace{1ex}

    \end{boxedminipage}

    \label{pymc:distributions:multinomial_like}
    \index{pymc \textit{(package)}!pymc.distributions \textit{(module)}!pymc.distributions.multinomial\_like \textit{(function)}}

    \vspace{0.5ex}

    \begin{boxedminipage}{\textwidth}

    \raggedright \textbf{multinomial\_like}(\textit{x}, \textit{n}, \textit{p})

    \vspace{-1.5ex}

    \rule{\textwidth}{0.5\fboxrule}

Multinomial log-likelihood with k-1 bins. Generalization of the binomial
distribution, but instead of each trial resulting in ``success'' or
``failure'', each one results in exactly one of some fixed finite number k
of possible outcomes over n independent trials. 'x{[}i{]}' indicates the number
of times outcome number i was observed over the n trials.
\begin{equation*}\begin{split}f(x \mid n, p) = \frac{n!}{\prod_{i=1}^k x_i!} \prod_{i=1}^k p_i^{x_i}\end{split}\end{equation*}    \vspace{1ex}

      \textbf{Parameters}
      \begin{quote}
        \begin{Ventry}{x}

          \item[x]


Random variable indicating the number of time outcome i is observed,
$\sum_{i=1}^k x_i=n$, $x_i \ge 0$.
            \textit{(type=(ns, k) int)}

          \item[n]


Number of trials.
            \textit{(type=int)}

          \item[p]


Probability of each one of the different outcomes,
$\sum_{i=1}^k p_i = 1)$, $p_i \ge 0$.
            \textit{(type=(k,) float)}

        \end{Ventry}

      \end{quote}

    \vspace{1ex}

\textbf{Note:} \begin{itemize}
\item {} 
$E(X_i)=n p_i$

\item {} 
$var(X_i)=n p_i(1-p_i)$

\item {} 
$cov(X_i,X_j) = -n p_i p_j$

\end{itemize}


    \end{boxedminipage}

    \label{pymc:distributions:rmultivariate_hypergeometric}
    \index{pymc \textit{(package)}!pymc.distributions \textit{(module)}!pymc.distributions.rmultivariate\_hypergeometric \textit{(function)}}

    \vspace{0.5ex}

    \begin{boxedminipage}{\textwidth}

    \raggedright \textbf{rmultivariate\_hypergeometric}(\textit{n}, \textit{m}, \textit{size}=\texttt{None})

    \vspace{-1.5ex}

    \rule{\textwidth}{0.5\fboxrule}

Random multivariate hypergeometric variates.

n : Number of draws.
m : Number of items in each category.
    \vspace{1ex}

    \end{boxedminipage}

    \label{pymc:distributions:multivariate_hypergeometric_expval}
    \index{pymc \textit{(package)}!pymc.distributions \textit{(module)}!pymc.distributions.multivariate\_hypergeometric\_expval \textit{(function)}}

    \vspace{0.5ex}

    \begin{boxedminipage}{\textwidth}

    \raggedright \textbf{multivariate\_hypergeometric\_expval}(\textit{n}, \textit{m})

    \vspace{-1.5ex}

    \rule{\textwidth}{0.5\fboxrule}

Expected value of multivariate hypergeometric distribution.

n : number of items drawn.
m : number of items in each category.
    \vspace{1ex}

    \end{boxedminipage}

    \label{pymc:distributions:multivariate_hypergeometric_like}
    \index{pymc \textit{(package)}!pymc.distributions \textit{(module)}!pymc.distributions.multivariate\_hypergeometric\_like \textit{(function)}}

    \vspace{0.5ex}

    \begin{boxedminipage}{\textwidth}

    \raggedright \textbf{multivariate\_hypergeometric\_like}(\textit{x}, \textit{m})

    \vspace{-1.5ex}

    \rule{\textwidth}{0.5\fboxrule}
\begin{alltt}
The multivariate hypergeometric describes the probability of drawing x[i] 
elements of the ith category, when the number of items in each category is
given by m. 


.. math::
    {\textbackslash}frac\{{\textbackslash}prod\_i {\textbackslash}binom\{m\_i\}\{c\_i\}\}\{{\textbackslash}binom\{N\}\{n]\}

where :math:`N = {\textbackslash}sum\_i m\_i` and :math:`n = {\textbackslash}sum\_i x\_i`.

:Parameters:
    x : int sequence
        Number of draws from each category, :math:`{\textless} m`
    m : int sequence
        Number of items in each categoy. 
\end{alltt}

    \vspace{1ex}

    \end{boxedminipage}

    \label{pymc:distributions:rmv_normal}
    \index{pymc \textit{(package)}!pymc.distributions \textit{(module)}!pymc.distributions.rmv\_normal \textit{(function)}}

    \vspace{0.5ex}

    \begin{boxedminipage}{\textwidth}

    \raggedright \textbf{rmv\_normal}(\textit{mu}, \textit{tau}, \textit{size}=\texttt{1})

    \vspace{-1.5ex}

    \rule{\textwidth}{0.5\fboxrule}

Random multivariate normal variates.
    \vspace{1ex}

    \end{boxedminipage}

    \label{pymc:distributions:mv_normal_expval}
    \index{pymc \textit{(package)}!pymc.distributions \textit{(module)}!pymc.distributions.mv\_normal\_expval \textit{(function)}}

    \vspace{0.5ex}

    \begin{boxedminipage}{\textwidth}

    \raggedright \textbf{mv\_normal\_expval}(\textit{mu}, \textit{tau})

    \vspace{-1.5ex}

    \rule{\textwidth}{0.5\fboxrule}

Expected value of multivariate normal distribution.
    \vspace{1ex}

    \end{boxedminipage}

    \label{pymc:distributions:mv_normal_like}
    \index{pymc \textit{(package)}!pymc.distributions \textit{(module)}!pymc.distributions.mv\_normal\_like \textit{(function)}}

    \vspace{0.5ex}

    \begin{boxedminipage}{\textwidth}

    \raggedright \textbf{mv\_normal\_like}(\textit{x}, \textit{mu}, \textit{tau})

    \vspace{-1.5ex}

    \rule{\textwidth}{0.5\fboxrule}

Multivariate normal log-likelihood
\begin{equation*}\begin{split}f(x \mid \pi, T) = \frac{T^{n/2}}{(2\pi)^{1/2}} \exp\left\{ -\frac{1}{2} (x-\mu)^{\prime}T(x-\mu) \right\}\end{split}\end{equation*}
x: (n,k)
mu: (k)
tau: (k,k)
tau positive definite
    \vspace{1ex}

    \end{boxedminipage}

    \label{pymc:distributions:rmv_normal_cov}
    \index{pymc \textit{(package)}!pymc.distributions \textit{(module)}!pymc.distributions.rmv\_normal\_cov \textit{(function)}}

    \vspace{0.5ex}

    \begin{boxedminipage}{\textwidth}

    \raggedright \textbf{rmv\_normal\_cov}(\textit{mu}, \textit{C})

    \vspace{-1.5ex}

    \rule{\textwidth}{0.5\fboxrule}

Random multivariate normal variates.
    \vspace{1ex}

    \end{boxedminipage}

    \label{pymc:distributions:mv_normal_cov_expval}
    \index{pymc \textit{(package)}!pymc.distributions \textit{(module)}!pymc.distributions.mv\_normal\_cov\_expval \textit{(function)}}

    \vspace{0.5ex}

    \begin{boxedminipage}{\textwidth}

    \raggedright \textbf{mv\_normal\_cov\_expval}(\textit{mu}, \textit{C})

    \vspace{-1.5ex}

    \rule{\textwidth}{0.5\fboxrule}

Expected value of multivariate normal distribution.
    \vspace{1ex}

    \end{boxedminipage}

    \label{pymc:distributions:mv_normal_cov_like}
    \index{pymc \textit{(package)}!pymc.distributions \textit{(module)}!pymc.distributions.mv\_normal\_cov\_like \textit{(function)}}

    \vspace{0.5ex}

    \begin{boxedminipage}{\textwidth}

    \raggedright \textbf{mv\_normal\_cov\_like}(\textit{x}, \textit{mu}, \textit{C})

    \vspace{-1.5ex}

    \rule{\textwidth}{0.5\fboxrule}

Multivariate normal log-likelihood
\begin{equation*}\begin{split}f(x \mid \pi, C) = \frac{T^{n/2}}{(2\pi)^{1/2}} \exp\left\{ -\frac{1}{2} (x-\mu)^{\prime}C^{-1}(x-\mu) \right\}\end{split}\end{equation*}
x: (n,k)
mu: (k)
C: (k,k)
C positive definite
    \vspace{1ex}

    \end{boxedminipage}

    \label{pymc:distributions:rmv_normal_chol}
    \index{pymc \textit{(package)}!pymc.distributions \textit{(module)}!pymc.distributions.rmv\_normal\_chol \textit{(function)}}

    \vspace{0.5ex}

    \begin{boxedminipage}{\textwidth}

    \raggedright \textbf{rmv\_normal\_chol}(\textit{mu}, \textit{sig})

    \vspace{-1.5ex}

    \rule{\textwidth}{0.5\fboxrule}

Random multivariate normal variates.
    \vspace{1ex}

    \end{boxedminipage}

    \label{pymc:distributions:mv_normal_chol_expval}
    \index{pymc \textit{(package)}!pymc.distributions \textit{(module)}!pymc.distributions.mv\_normal\_chol\_expval \textit{(function)}}

    \vspace{0.5ex}

    \begin{boxedminipage}{\textwidth}

    \raggedright \textbf{mv\_normal\_chol\_expval}(\textit{mu}, \textit{sig})

    \vspace{-1.5ex}

    \rule{\textwidth}{0.5\fboxrule}

Expected value of multivariate normal distribution.
    \vspace{1ex}

    \end{boxedminipage}

    \label{pymc:distributions:mv_normal_chol_like}
    \index{pymc \textit{(package)}!pymc.distributions \textit{(module)}!pymc.distributions.mv\_normal\_chol\_like \textit{(function)}}

    \vspace{0.5ex}

    \begin{boxedminipage}{\textwidth}

    \raggedright \textbf{mv\_normal\_chol\_like}(\textit{x}, \textit{mu}, \textit{tau})

    \vspace{-1.5ex}

    \rule{\textwidth}{0.5\fboxrule}

Multivariate normal log-likelihood
\begin{equation*}\begin{split}f(x \mid \pi, \sigma) = \frac{T^{n/2}}{(2\pi)^{1/2}} \exp\left\{ -\frac{1}{2} (x-\mu)^{\prime}\sigma \sigma^{\prime}(x-\mu) \right\}\end{split}\end{equation*}
x: (n,k)
mu: (k)
sigma: (k,k)
sigma lower triangular
    \vspace{1ex}

    \end{boxedminipage}

    \label{pymc:distributions:rnegative_binomial}
    \index{pymc \textit{(package)}!pymc.distributions \textit{(module)}!pymc.distributions.rnegative\_binomial \textit{(function)}}

    \vspace{0.5ex}

    \begin{boxedminipage}{\textwidth}

    \raggedright \textbf{rnegative\_binomial}(\textit{mu}, \textit{alpha}, \textit{size}=\texttt{1})

    \vspace{-1.5ex}

    \rule{\textwidth}{0.5\fboxrule}

Random negative binomial variates.
    \vspace{1ex}

    \end{boxedminipage}

    \label{pymc:distributions:negative_binomial_expval}
    \index{pymc \textit{(package)}!pymc.distributions \textit{(module)}!pymc.distributions.negative\_binomial\_expval \textit{(function)}}

    \vspace{0.5ex}

    \begin{boxedminipage}{\textwidth}

    \raggedright \textbf{negative\_binomial\_expval}(\textit{mu}, \textit{alpha})

    \vspace{-1.5ex}

    \rule{\textwidth}{0.5\fboxrule}

Expected value of negative binomial distribution.
    \vspace{1ex}

    \end{boxedminipage}

    \label{pymc:distributions:negative_binomial_like}
    \index{pymc \textit{(package)}!pymc.distributions \textit{(module)}!pymc.distributions.negative\_binomial\_like \textit{(function)}}

    \vspace{0.5ex}

    \begin{boxedminipage}{\textwidth}

    \raggedright \textbf{negative\_binomial\_like}(\textit{x}, \textit{mu}, \textit{alpha})

    \vspace{-1.5ex}

    \rule{\textwidth}{0.5\fboxrule}

Negative binomial log-likelihood
\begin{equation*}\begin{split}f(x \mid r, p) = \frac{(x+r-1)!}{x! (r-1)!} p^r (1-p)^x\end{split}\end{equation*}
x {\textgreater} 0, mu {\textgreater} 0, alpha {\textgreater} 0
    \vspace{1ex}

    \end{boxedminipage}

    \label{pymc:distributions:rnormal}
    \index{pymc \textit{(package)}!pymc.distributions \textit{(module)}!pymc.distributions.rnormal \textit{(function)}}

    \vspace{0.5ex}

    \begin{boxedminipage}{\textwidth}

    \raggedright \textbf{rnormal}(\textit{mu}, \textit{tau}, \textit{size}=\texttt{1})

    \vspace{-1.5ex}

    \rule{\textwidth}{0.5\fboxrule}

Random normal variates.
    \vspace{1ex}

    \end{boxedminipage}

    \label{pymc:distributions:normal_expval}
    \index{pymc \textit{(package)}!pymc.distributions \textit{(module)}!pymc.distributions.normal\_expval \textit{(function)}}

    \vspace{0.5ex}

    \begin{boxedminipage}{\textwidth}

    \raggedright \textbf{normal\_expval}(\textit{mu}, \textit{tau})

    \vspace{-1.5ex}

    \rule{\textwidth}{0.5\fboxrule}

Expected value of normal distribution.
    \vspace{1ex}

    \end{boxedminipage}

    \label{pymc:distributions:normal_like}
    \index{pymc \textit{(package)}!pymc.distributions \textit{(module)}!pymc.distributions.normal\_like \textit{(function)}}

    \vspace{0.5ex}

    \begin{boxedminipage}{\textwidth}

    \raggedright \textbf{normal\_like}(\textit{x}, \textit{mu}, \textit{tau})

    \vspace{-1.5ex}

    \rule{\textwidth}{0.5\fboxrule}

Normal log-likelihood.
\begin{equation*}\begin{split}f(x \mid \mu, \tau) = \sqrt{\frac{\tau}{2\pi}} \exp\left\{ -\frac{\tau}{2} (x-\mu)^2 \right\}\end{split}\end{equation*}    \vspace{1ex}

      \textbf{Parameters}
      \begin{quote}
        \begin{Ventry}{xxx}

          \item[x]


Input data.
            \textit{(type=float)}

          \item[mu]


Mean of the distribution.
            \textit{(type=float)}

          \item[tau]


Precision of the distribution, {\textgreater} 0.
            \textit{(type=float)}

        \end{Ventry}

      \end{quote}

    \vspace{1ex}

\textbf{Note:} \begin{itemize}
\item {} 
$E(X) = \mu$

\item {} 
$Var(X) = 1/\tau$

\end{itemize}


    \end{boxedminipage}

    \label{pymc:distributions:rpoisson}
    \index{pymc \textit{(package)}!pymc.distributions \textit{(module)}!pymc.distributions.rpoisson \textit{(function)}}

    \vspace{0.5ex}

    \begin{boxedminipage}{\textwidth}

    \raggedright \textbf{rpoisson}(\textit{mu}, \textit{size}=\texttt{1})

    \vspace{-1.5ex}

    \rule{\textwidth}{0.5\fboxrule}

Random poisson variates.
    \vspace{1ex}

    \end{boxedminipage}

    \label{pymc:distributions:poisson_expval}
    \index{pymc \textit{(package)}!pymc.distributions \textit{(module)}!pymc.distributions.poisson\_expval \textit{(function)}}

    \vspace{0.5ex}

    \begin{boxedminipage}{\textwidth}

    \raggedright \textbf{poisson\_expval}(\textit{mu})

    \vspace{-1.5ex}

    \rule{\textwidth}{0.5\fboxrule}

Expected value of Poisson distribution.
    \vspace{1ex}

    \end{boxedminipage}

    \label{pymc:distributions:poisson_like}
    \index{pymc \textit{(package)}!pymc.distributions \textit{(module)}!pymc.distributions.poisson\_like \textit{(function)}}

    \vspace{0.5ex}

    \begin{boxedminipage}{\textwidth}

    \raggedright \textbf{poisson\_like}(\textit{x}, \textit{mu})

    \vspace{-1.5ex}

    \rule{\textwidth}{0.5\fboxrule}

Poisson log-likelihood. The Poisson is a discrete probability distribution.
It expresses the probability of a number of events occurring in a fixed
period of time if these events occur with a known average rate, and are
independent of the time since the last event. The Poisson distribution can
be derived as a limiting case of the binomial distribution.
\begin{equation*}\begin{split}f(x \mid \mu) = \frac{e^{-\mu}\mu^x}{x!}\end{split}\end{equation*}    \vspace{1ex}

      \textbf{Parameters}
      \begin{quote}
        \begin{Ventry}{xx}

          \item[x]


$x \in {0,1,2,...}$
            \textit{(type=int)}

          \item[mu]


Expected number of occurrences that occur during the given interval,
$\mu \geq 0$.
            \textit{(type=float)}

        \end{Ventry}

      \end{quote}

    \vspace{1ex}

\textbf{Note:} \begin{itemize}
\item {} 
$E(x)=\mu$

\item {} 
$Var(x)=\mu$

\end{itemize}


    \end{boxedminipage}

    \label{pymc:distributions:rtruncnorm}
    \index{pymc \textit{(package)}!pymc.distributions \textit{(module)}!pymc.distributions.rtruncnorm \textit{(function)}}

    \vspace{0.5ex}

    \begin{boxedminipage}{\textwidth}

    \raggedright \textbf{rtruncnorm}(\textit{mu}, \textit{sigma}, \textit{a}, \textit{b}, \textit{size}=\texttt{1})

    \vspace{-1.5ex}

    \rule{\textwidth}{0.5\fboxrule}

Random truncated normal variates.
    \vspace{1ex}

    \end{boxedminipage}

    \label{pymc:distributions:truncnorm_expval}
    \index{pymc \textit{(package)}!pymc.distributions \textit{(module)}!pymc.distributions.truncnorm\_expval \textit{(function)}}

    \vspace{0.5ex}

    \begin{boxedminipage}{\textwidth}

    \raggedright \textbf{truncnorm\_expval}(\textit{mu}, \textit{sigma}, \textit{a}, \textit{b})

    \vspace{-1.5ex}

    \rule{\textwidth}{0.5\fboxrule}

E(X)=mu +  rac{\{}sigma( arphi{\_}1- arphi{\_}2){\}}{\{}T{\}}, where T=Phileft( rac{\{}B-mu{\}}{\{}sigma{\}}
ight)-Phileft( rac{\{}A-mu{\}}{\{}sigma{\}}
ight) and  arphi{\_}1 =  arphileft( rac{\{}A-mu{\}}{\{}sigma{\}}
ight) and  arphi{\_}2 =  arphileft( rac{\{}B-mu{\}}{\{}sigma{\}}
ight), where  arphi is the probability density function of a standard normal random variable.
    \vspace{1ex}

    \end{boxedminipage}

    \label{pymc:distributions:truncnorm_like}
    \index{pymc \textit{(package)}!pymc.distributions \textit{(module)}!pymc.distributions.truncnorm\_like \textit{(function)}}

    \vspace{0.5ex}

    \begin{boxedminipage}{\textwidth}

    \raggedright \textbf{truncnorm\_like}(\textit{x}, \textit{mu}, \textit{sigma}, \textit{a}, \textit{b})

    \vspace{-1.5ex}

    \rule{\textwidth}{0.5\fboxrule}

Truncated normal log-likelihood.
\begin{equation*}\begin{split}f(x \mid \mu, \sigma, a, b) = \frac{\phi(\frac{x-\mu}{\sigma})} {\Phi(\frac{b-\mu}{\sigma}) - \Phi(\frac{a-\mu}{\sigma})},\end{split}\end{equation*}    \vspace{1ex}

    \end{boxedminipage}

    \label{pymc:distributions:rskew_normal}
    \index{pymc \textit{(package)}!pymc.distributions \textit{(module)}!pymc.distributions.rskew\_normal \textit{(function)}}

    \vspace{0.5ex}

    \begin{boxedminipage}{\textwidth}

    \raggedright \textbf{rskew\_normal}(\textit{mu}, \textit{tau}, \textit{alpha}, \textit{size}=\texttt{1})

    \vspace{-1.5ex}

    \rule{\textwidth}{0.5\fboxrule}

Skew-normal random variates.
    \vspace{1ex}

    \end{boxedminipage}

    \label{pymc:distributions:skew_normal_like}
    \index{pymc \textit{(package)}!pymc.distributions \textit{(module)}!pymc.distributions.skew\_normal\_like \textit{(function)}}

    \vspace{0.5ex}

    \begin{boxedminipage}{\textwidth}

    \raggedright \textbf{skew\_normal\_like}(\textit{x}, \textit{mu}, \textit{tau}, \textit{alpha})

    \vspace{-1.5ex}

    \rule{\textwidth}{0.5\fboxrule}

Azzalini's skew-normal log-likelihood
\begin{equation*}\begin{split}f(x \mid \mu, \tau, \alpha) = 2 \Phi((x-\mu)\sqrt{tau}\alpha) \phi(x,\mu,\tau)\end{split}\end{equation*}    \vspace{1ex}

      \textbf{Parameters}
      \begin{quote}
        \begin{Ventry}{xxxxx}

          \item[x]


Input data.
            \textit{(type=float)}

          \item[mu]


Mean of the distribution.
            \textit{(type=float)}

          \item[tau]


Precision of the distribution, {\textgreater} 0.
            \textit{(type=float)}

          \item[alpha]


Shape parameter of the distribution.
            \textit{(type=float)}

        \end{Ventry}

      \end{quote}

    \vspace{1ex}

\textbf{Note:} \begin{itemize}
\item {} 
See \href{http://azzalini.stat.unipd.it/SN/}{http://azzalini.stat.unipd.it/SN/}

\end{itemize}


    \end{boxedminipage}

    \label{pymc:distributions:skew_normal_expval}
    \index{pymc \textit{(package)}!pymc.distributions \textit{(module)}!pymc.distributions.skew\_normal\_expval \textit{(function)}}

    \vspace{0.5ex}

    \begin{boxedminipage}{\textwidth}

    \raggedright \textbf{skew\_normal\_expval}(\textit{mu}, \textit{tau}, \textit{alpha})

    \vspace{-1.5ex}

    \rule{\textwidth}{0.5\fboxrule}

Expectation of skew-normal random variables.
    \vspace{1ex}

    \end{boxedminipage}

    \label{pymc:distributions:rdiscrete_uniform}
    \index{pymc \textit{(package)}!pymc.distributions \textit{(module)}!pymc.distributions.rdiscrete\_uniform \textit{(function)}}

    \vspace{0.5ex}

    \begin{boxedminipage}{\textwidth}

    \raggedright \textbf{rdiscrete\_uniform}(\textit{lower}, \textit{upper}, \textit{size}=\texttt{1})

    \vspace{-1.5ex}

    \rule{\textwidth}{0.5\fboxrule}

Random discrete{\_}uniform variates.
    \vspace{1ex}

    \end{boxedminipage}

    \label{pymc:distributions:discrete_uniform_expval}
    \index{pymc \textit{(package)}!pymc.distributions \textit{(module)}!pymc.distributions.discrete\_uniform\_expval \textit{(function)}}

    \vspace{0.5ex}

    \begin{boxedminipage}{\textwidth}

    \raggedright \textbf{discrete\_uniform\_expval}(\textit{lower}, \textit{upper})

    \vspace{-1.5ex}

    \rule{\textwidth}{0.5\fboxrule}

Expected value of discrete{\_}uniform distribution.
    \vspace{1ex}

    \end{boxedminipage}

    \label{pymc:distributions:discrete_uniform_like}
    \index{pymc \textit{(package)}!pymc.distributions \textit{(module)}!pymc.distributions.discrete\_uniform\_like \textit{(function)}}

    \vspace{0.5ex}

    \begin{boxedminipage}{\textwidth}

    \raggedright \textbf{discrete\_uniform\_like}(\textit{x}, \textit{lower}, \textit{upper})

    \vspace{-1.5ex}

    \rule{\textwidth}{0.5\fboxrule}

discrete{\_}uniform log-likelihood.
\begin{equation*}\begin{split}f(x \mid lower, upper) = \frac{1}{upper-lower}\end{split}\end{equation*}    \vspace{1ex}

      \textbf{Parameters}
      \begin{quote}
        \begin{Ventry}{xxxxx}

          \item[x]


$lower \geq x \geq upper$
            \textit{(type=float)}

          \item[lower]


Lower limit.
            \textit{(type=float)}

          \item[upper]


Upper limit.
            \textit{(type=float)}

        \end{Ventry}

      \end{quote}

    \vspace{1ex}

    \end{boxedminipage}

    \label{pymc:distributions:runiform}
    \index{pymc \textit{(package)}!pymc.distributions \textit{(module)}!pymc.distributions.runiform \textit{(function)}}

    \vspace{0.5ex}

    \begin{boxedminipage}{\textwidth}

    \raggedright \textbf{runiform}(\textit{lower}, \textit{upper}, \textit{size}=\texttt{1})

    \vspace{-1.5ex}

    \rule{\textwidth}{0.5\fboxrule}

Random uniform variates.
    \vspace{1ex}

    \end{boxedminipage}

    \label{pymc:distributions:uniform_expval}
    \index{pymc \textit{(package)}!pymc.distributions \textit{(module)}!pymc.distributions.uniform\_expval \textit{(function)}}

    \vspace{0.5ex}

    \begin{boxedminipage}{\textwidth}

    \raggedright \textbf{uniform\_expval}(\textit{lower}, \textit{upper})

    \vspace{-1.5ex}

    \rule{\textwidth}{0.5\fboxrule}

Expected value of uniform distribution.
    \vspace{1ex}

    \end{boxedminipage}

    \label{pymc:distributions:uniform_like}
    \index{pymc \textit{(package)}!pymc.distributions \textit{(module)}!pymc.distributions.uniform\_like \textit{(function)}}

    \vspace{0.5ex}

    \begin{boxedminipage}{\textwidth}

    \raggedright \textbf{uniform\_like}(\textit{x}, \textit{lower}, \textit{upper})

    \vspace{-1.5ex}

    \rule{\textwidth}{0.5\fboxrule}

Uniform log-likelihood.
\begin{equation*}\begin{split}f(x \mid lower, upper) = \frac{1}{upper-lower}\end{split}\end{equation*}    \vspace{1ex}

      \textbf{Parameters}
      \begin{quote}
        \begin{Ventry}{xxxxx}

          \item[x]


$lower \geq x \geq upper$
            \textit{(type=float)}

          \item[lower]


Lower limit.
            \textit{(type=float)}

          \item[upper]


Upper limit.
            \textit{(type=float)}

        \end{Ventry}

      \end{quote}

    \vspace{1ex}

    \end{boxedminipage}

    \label{pymc:distributions:rweibull}
    \index{pymc \textit{(package)}!pymc.distributions \textit{(module)}!pymc.distributions.rweibull \textit{(function)}}

    \vspace{0.5ex}

    \begin{boxedminipage}{\textwidth}

    \raggedright \textbf{rweibull}(\textit{alpha}, \textit{beta}, \textit{size}=\texttt{1})

    \end{boxedminipage}

    \label{pymc:distributions:weibull_expval}
    \index{pymc \textit{(package)}!pymc.distributions \textit{(module)}!pymc.distributions.weibull\_expval \textit{(function)}}

    \vspace{0.5ex}

    \begin{boxedminipage}{\textwidth}

    \raggedright \textbf{weibull\_expval}(\textit{alpha}, \textit{beta})

    \vspace{-1.5ex}

    \rule{\textwidth}{0.5\fboxrule}

Expected value of weibull distribution.
    \vspace{1ex}

    \end{boxedminipage}

    \label{pymc:distributions:weibull_like}
    \index{pymc \textit{(package)}!pymc.distributions \textit{(module)}!pymc.distributions.weibull\_like \textit{(function)}}

    \vspace{0.5ex}

    \begin{boxedminipage}{\textwidth}

    \raggedright \textbf{weibull\_like}(\textit{x}, \textit{alpha}, \textit{beta})

    \vspace{-1.5ex}

    \rule{\textwidth}{0.5\fboxrule}

Weibull log-likelihood
\begin{equation*}\begin{split}f(x \mid \alpha, \beta) = \frac{\alpha x^{\alpha - 1}\exp(-(\frac{x}{\beta})^{\alpha})}{\beta^\alpha}\end{split}\end{equation*}    \vspace{1ex}

      \textbf{Parameters}
      \begin{quote}
        \begin{Ventry}{xxxxx}

          \item[x]


:math:'x ge 0'
            \textit{(type=float)}

          \item[alpha]


{\textgreater} 0
            \textit{(type=float)}

          \item[beta]


{\textgreater} 0
            \textit{(type=float)}

        \end{Ventry}

      \end{quote}

    \vspace{1ex}

\textbf{Note:} \begin{itemize}
\item {} 
:math:'E(x)=beta Gamma(1+frac{\{}1{\}}{\{}alpha{\}})'

\item {} 
:math:'Var(x)=beta{\textasciicircum}2 Gamma(1+frac{\{}2{\}}{\{}alpha{\}} - mu{\textasciicircum}2)'

\end{itemize}


    \end{boxedminipage}

    \label{pymc:distributions:rwishart}
    \index{pymc \textit{(package)}!pymc.distributions \textit{(module)}!pymc.distributions.rwishart \textit{(function)}}

    \vspace{0.5ex}

    \begin{boxedminipage}{\textwidth}

    \raggedright \textbf{rwishart}(\textit{n}, \textit{Tau})

    \vspace{-1.5ex}

    \rule{\textwidth}{0.5\fboxrule}

Return a Wishart random matrix.

Tau is the inverse of the 'covariance' matrix :math:'C'.
    \vspace{1ex}

    \end{boxedminipage}

    \label{pymc:distributions:wishart_expval}
    \index{pymc \textit{(package)}!pymc.distributions \textit{(module)}!pymc.distributions.wishart\_expval \textit{(function)}}

    \vspace{0.5ex}

    \begin{boxedminipage}{\textwidth}

    \raggedright \textbf{wishart\_expval}(\textit{n}, \textit{Tau})

    \vspace{-1.5ex}

    \rule{\textwidth}{0.5\fboxrule}

Expected value of wishart distribution.
    \vspace{1ex}

    \end{boxedminipage}

    \label{pymc:distributions:wishart_like}
    \index{pymc \textit{(package)}!pymc.distributions \textit{(module)}!pymc.distributions.wishart\_like \textit{(function)}}

    \vspace{0.5ex}

    \begin{boxedminipage}{\textwidth}

    \raggedright \textbf{wishart\_like}(\textit{X}, \textit{n}, \textit{Tau})

    \vspace{-1.5ex}

    \rule{\textwidth}{0.5\fboxrule}

Wishart log-likelihood. The Wishart distribution is the probability
distribution of the maximum-likelihood estimator (MLE) of the precision
matrix of a multivariate normal distribution. If Tau=1, the distribution
is identical to the chi-square distribution with n degrees of freedom.
\begin{equation*}\begin{split}f(X \mid n, T) = {\mid T \mid}^{n/2}{\mid X \mid}^{(n-k-1)/2} \exp\left\{ -\frac{1}{2} Tr(TX) \right\}\end{split}\end{equation*}
where :math:'k' is the rank of X.
    \vspace{1ex}

      \textbf{Parameters}
      \begin{quote}
        \begin{Ventry}{xxx}

          \item[X]


Symmetric, positive definite.
            \textit{(type=matrix)}

          \item[n]


Degrees of freedom, {\textgreater} 0.
            \textit{(type=int)}

          \item[Tau]


Symmetric and positive definite
            \textit{(type=matrix)}

        \end{Ventry}

      \end{quote}

    \vspace{1ex}

    \end{boxedminipage}

    \label{pymc:distributions:rwishart_cov}
    \index{pymc \textit{(package)}!pymc.distributions \textit{(module)}!pymc.distributions.rwishart\_cov \textit{(function)}}

    \vspace{0.5ex}

    \begin{boxedminipage}{\textwidth}

    \raggedright \textbf{rwishart\_cov}(\textit{n}, \textit{C})

    \vspace{-1.5ex}

    \rule{\textwidth}{0.5\fboxrule}

Return a Wishart random matrix.
    \vspace{1ex}

    \end{boxedminipage}

    \label{pymc:distributions:wishart_cov_expval}
    \index{pymc \textit{(package)}!pymc.distributions \textit{(module)}!pymc.distributions.wishart\_cov\_expval \textit{(function)}}

    \vspace{0.5ex}

    \begin{boxedminipage}{\textwidth}

    \raggedright \textbf{wishart\_cov\_expval}(\textit{n}, \textit{C})

    \vspace{-1.5ex}

    \rule{\textwidth}{0.5\fboxrule}

Expected value of wishart distribution.
    \vspace{1ex}

    \end{boxedminipage}

    \label{pymc:distributions:wishart_cov_like}
    \index{pymc \textit{(package)}!pymc.distributions \textit{(module)}!pymc.distributions.wishart\_cov\_like \textit{(function)}}

    \vspace{0.5ex}

    \begin{boxedminipage}{\textwidth}

    \raggedright \textbf{wishart\_cov\_like}(\textit{X}, \textit{n}, \textit{C})

    \vspace{-1.5ex}

    \rule{\textwidth}{0.5\fboxrule}

PLEASE CHECK THIS DOCSTRING
wishart{\_}like(X, n, C)

Wishart log-likelihood. The Wishart distribution is the probability
distribution of the maximum-likelihood estimator (MLE) of the covariance
matrix of a multivariate normal distribution. If Tau=1, the distribution
is identical to the chi-square distribution with n degrees of freedom.
\begin{equation*}\begin{split}f(X \mid n, T) = {\mid T \mid}^{n/2}{\mid X \mid}^{(n-k-1)/2} \exp\left\{ -\frac{1}{2} Tr(TX) \right\}\end{split}\end{equation*}
where :math:'k' is the rank of X.
    \vspace{1ex}

      \textbf{Parameters}
      \begin{quote}
        \begin{Ventry}{x}

          \item[X]


Symmetric, positive definite.
            \textit{(type=matrix)}

          \item[n]


Degrees of freedom, {\textgreater} 0.
            \textit{(type=int)}

          \item[C]


Symmetric and positive definite
            \textit{(type=matrix)}

        \end{Ventry}

      \end{quote}

    \vspace{1ex}

    \end{boxedminipage}

    \label{pymc:distributions:name_to_funcs}
    \index{pymc \textit{(package)}!pymc.distributions \textit{(module)}!pymc.distributions.name\_to\_funcs \textit{(function)}}

    \vspace{0.5ex}

    \begin{boxedminipage}{\textwidth}

    \raggedright \textbf{name\_to\_funcs}(\textit{name}, \textit{module})

    \end{boxedminipage}

    \label{pymc:distributions:valuewrapper}
    \index{pymc \textit{(package)}!pymc.distributions \textit{(module)}!pymc.distributions.valuewrapper \textit{(function)}}

    \vspace{0.5ex}

    \begin{boxedminipage}{\textwidth}

    \raggedright \textbf{valuewrapper}(\textit{f})

    \vspace{-1.5ex}

    \rule{\textwidth}{0.5\fboxrule}

Return a likelihood accepting value instead of x as a keyword argument.
This is specifically intended for the instantiator above.
    \vspace{1ex}

    \end{boxedminipage}

    \label{pymc:distributions:random_method_wrapper}
    \index{pymc \textit{(package)}!pymc.distributions \textit{(module)}!pymc.distributions.random\_method\_wrapper \textit{(function)}}

    \vspace{0.5ex}

    \begin{boxedminipage}{\textwidth}

    \raggedright \textbf{random\_method\_wrapper}(\textit{f}, \textit{size}, \textit{shape})

    \vspace{-1.5ex}

    \rule{\textwidth}{0.5\fboxrule}

Wraps functions to return values of appropriate shape.
    \vspace{1ex}

    \end{boxedminipage}

    \label{pymc:distributions:fortranlike}
    \index{pymc \textit{(package)}!pymc.distributions \textit{(module)}!pymc.distributions.fortranlike \textit{(function)}}

    \vspace{0.5ex}

    \begin{boxedminipage}{\textwidth}

    \raggedright \textbf{fortranlike}(\textit{f}, \textit{snapshot}, \textit{mv}=\texttt{False})

    \vspace{-1.5ex}

    \rule{\textwidth}{0.5\fboxrule}


%___________________________________________________________________________

\hypertarget{decorator-function-for-fortran-likelihoods}{}
\pdfbookmark[2]{Decorator function for fortran likelihoods}{decorator-function-for-fortran-likelihoods}
\subsubsection*{Decorator function for fortran likelihoods}
\label{decorator-function-for-fortran-likelihoods}

Wrap function f({\color{red}\bfseries{}*}args, {\color{red}\bfseries{}**}kwds) where f is a likelihood defined in flib.

Assume args = (x, parameter1, parameter2, ...)
Before passing the arguments to the function, the wrapper makes sure that
the parameters have the same shape as x.

mv: multivariate (True/False)


%___________________________________________________________________________

\hypertarget{add-compatibility-with-gof-goodness-of-fit-tests}{}
\pdfbookmark[3]{Add compatibility with GoF (Goodness of Fit) tests}{add-compatibility-with-gof-goodness-of-fit-tests}
\paragraph*{Add compatibility with GoF (Goodness of Fit) tests}
\label{add-compatibility-with-gof-goodness-of-fit-tests}
\begin{itemize}
\item {} 
Add a 'prior' keyword (True/False)

\item {} 
If the keyword gof is given and is True, return the GoF (Goodness of Fit)

\end{itemize}

points instead of the likelihood.
* A 'loss' keyword can be given, to specify the loss function used in the
computation of the GoF points.
* If the keyword random is given and True, return a random variate instead
of the likelihood.
    \vspace{1ex}

    \end{boxedminipage}

    \label{pymc:distributions:local_decorated_likelihoods}
    \index{pymc \textit{(package)}!pymc.distributions \textit{(module)}!pymc.distributions.local\_decorated\_likelihoods \textit{(function)}}

    \vspace{0.5ex}

    \begin{boxedminipage}{\textwidth}

    \raggedright \textbf{local\_decorated\_likelihoods}(\textit{obj})

    \vspace{-1.5ex}

    \rule{\textwidth}{0.5\fboxrule}

New interface likelihoods
    \vspace{1ex}

    \end{boxedminipage}

    \label{pymc:distributions:uninformative_like}
    \index{pymc \textit{(package)}!pymc.distributions \textit{(module)}!pymc.distributions.uninformative\_like \textit{(function)}}

    \vspace{0.5ex}

    \begin{boxedminipage}{\textwidth}

    \raggedright \textbf{uninformative\_like}(\textit{x})

    \vspace{-1.5ex}

    \rule{\textwidth}{0.5\fboxrule}

Uninformative log-likelihood. Returns 0 regardless of the value of x.
    \vspace{1ex}

    \end{boxedminipage}

    \label{pymc:distributions:one_over_x_like}
    \index{pymc \textit{(package)}!pymc.distributions \textit{(module)}!pymc.distributions.one\_over\_x\_like \textit{(function)}}

    \vspace{0.5ex}

    \begin{boxedminipage}{\textwidth}

    \raggedright \textbf{one\_over\_x\_like}(\textit{x})

    \vspace{-1.5ex}

    \rule{\textwidth}{0.5\fboxrule}

returns -np.Inf if x{\textless}0, -np.log(x) otherwise.
    \vspace{1ex}

    \end{boxedminipage}

    \label{pymc:distributions:extend_dirichlet}
    \index{pymc \textit{(package)}!pymc.distributions \textit{(module)}!pymc.distributions.extend\_dirichlet \textit{(function)}}

    \vspace{0.5ex}

    \begin{boxedminipage}{\textwidth}

    \raggedright \textbf{extend\_dirichlet}(\textit{p})

    \end{boxedminipage}

    \label{pymc:distributions:mod_categor_like}
    \index{pymc \textit{(package)}!pymc.distributions \textit{(module)}!pymc.distributions.mod\_categor\_like \textit{(function)}}

    \vspace{0.5ex}

    \begin{boxedminipage}{\textwidth}

    \raggedright \textbf{mod\_categor\_like}(\textit{x}, \textit{p}, \textit{minval}=\texttt{0}, \textit{step}=\texttt{1})

    \end{boxedminipage}

    \label{pymc:distributions:mod_rcategor}
    \index{pymc \textit{(package)}!pymc.distributions \textit{(module)}!pymc.distributions.mod\_rcategor \textit{(function)}}

    \vspace{0.5ex}

    \begin{boxedminipage}{\textwidth}

    \raggedright \textbf{mod\_rcategor}(\textit{p}, \textit{minval}, \textit{step}, \textit{size}=\texttt{1})

    \end{boxedminipage}

    \label{pymc:distributions:mod_rmultinom}
    \index{pymc \textit{(package)}!pymc.distributions \textit{(module)}!pymc.distributions.mod\_rmultinom \textit{(function)}}

    \vspace{0.5ex}

    \begin{boxedminipage}{\textwidth}

    \raggedright \textbf{mod\_rmultinom}(\textit{n}, \textit{p})

    \end{boxedminipage}

    \label{pymc:distributions:mod_multinom_like}
    \index{pymc \textit{(package)}!pymc.distributions \textit{(module)}!pymc.distributions.mod\_multinom\_like \textit{(function)}}

    \vspace{0.5ex}

    \begin{boxedminipage}{\textwidth}

    \raggedright \textbf{mod\_multinom\_like}(\textit{x}, \textit{n}, \textit{p})

    \end{boxedminipage}


%%%%%%%%%%%%%%%%%%%%%%%%%%%%%%%%%%%%%%%%%%%%%%%%%%%%%%%%%%%%%%%%%%%%%%%%%%%
%%                               Variables                               %%
%%%%%%%%%%%%%%%%%%%%%%%%%%%%%%%%%%%%%%%%%%%%%%%%%%%%%%%%%%%%%%%%%%%%%%%%%%%

  \subsection{Variables}

\begin{longtable}{|p{.30\textwidth}|p{.62\textwidth}|l}
\cline{1-2}
\cline{1-2} \centering \textbf{Name} & \centering \textbf{Description}& \\
\cline{1-2}
\endhead\cline{1-2}\multicolumn{3}{r}{\small\textit{continued on next page}}\\\endfoot\cline{1-2}
\endlastfoot\raggedright f\-l\-i\-b\-\_\-b\-l\-a\-s\-\_\-O\-K\- & \raggedright \textbf{Value:} 
{\tt False}&\\
\cline{1-2}
\raggedright r\-a\-n\-d\-o\-m\-\_\-n\-u\-m\-b\-e\-r\- & \raggedright \textbf{Value:} 
{\tt np.random.random}&\\
\cline{1-2}
\raggedright i\-n\-v\-e\-r\-s\-e\- & \raggedright \textbf{Value:} 
{\tt np.linalg.pinv}&\\
\cline{1-2}
\raggedright s\-c\-\_\-c\-o\-n\-t\-i\-n\-u\-o\-u\-s\-\_\-d\-i\-s\-t\-r\-i\-b\-u\-t\-i\-o\-n\-s\- & \raggedright \textbf{Value:} 
{\tt ['bernoulli', 'beta', 'cauchy', 'chi2', 'exponential', 'e\texttt{...}}&\\
\cline{1-2}
\raggedright s\-c\-\_\-d\-i\-s\-c\-r\-e\-t\-e\-\_\-d\-i\-s\-t\-r\-i\-b\-u\-t\-i\-o\-n\-s\- & \raggedright \textbf{Value:} 
{\tt ['binomial', 'poisson', 'negative\_binomial', 'categorical\texttt{...}}&\\
\cline{1-2}
\raggedright m\-v\-\_\-c\-o\-n\-t\-i\-n\-u\-o\-u\-s\-\_\-d\-i\-s\-t\-r\-i\-b\-u\-t\-i\-o\-n\-s\- & \raggedright \textbf{Value:} 
{\tt ['dirichlet', 'mv\_normal', 'mv\_normal\_cov', 'mv\_normal\_ch\texttt{...}}&\\
\cline{1-2}
\raggedright m\-v\-\_\-d\-i\-s\-c\-r\-e\-t\-e\-\_\-d\-i\-s\-t\-r\-i\-b\-u\-t\-i\-o\-n\-s\- & \raggedright \textbf{Value:} 
{\tt ['multivariate\_hypergeometric', 'multinomial']}&\\
\cline{1-2}
\raggedright a\-v\-a\-i\-l\-a\-b\-l\-e\-d\-i\-s\-t\-r\-i\-b\-u\-t\-i\-o\-n\-s\- & \raggedright \textbf{Value:} 
{\tt sc\_continuous\_distributions+ sc\_discrete\_distributions+ m\texttt{...}}&\\
\cline{1-2}
\raggedright c\-a\-p\-i\-t\-a\-l\-i\-z\-e\- & \raggedright \textbf{Value:} 
{\tt lambda name:}&\\
\cline{1-2}
\raggedright a\-b\-s\-o\-l\-u\-t\-e\-\_\-l\-o\-s\-s\- & \raggedright \textbf{Value:} 
{\tt lambda o, e:}&\\
\cline{1-2}
\raggedright s\-q\-u\-a\-r\-e\-d\-\_\-l\-o\-s\-s\- & \raggedright \textbf{Value:} 
{\tt lambda o, e:}&\\
\cline{1-2}
\raggedright c\-h\-i\-\_\-s\-q\-u\-a\-r\-e\-\_\-l\-o\-s\-s\- & \raggedright \textbf{Value:} 
{\tt lambda o, e:}&\\
\cline{1-2}
\raggedright s\-n\-a\-p\-s\-h\-o\-t\- & \raggedright \textbf{Value:} 
{\tt locals().copy()}&\\
\cline{1-2}
\raggedright l\-i\-k\-e\-l\-i\-h\-o\-o\-d\-s\- & \raggedright \textbf{Value:} 
{\tt \{\}}&\\
\cline{1-2}
\raggedright B\-e\-r\-n\-o\-u\-l\-l\-i\- & \raggedright \textbf{Value:} 
{\tt stochastic\_from\_dist('bernoulli', dist\_logp, dist\_random,\texttt{...}}&\\
\cline{1-2}
\raggedright U\-n\-i\-n\-f\-o\-r\-m\-a\-t\-i\-v\-e\- & \raggedright \textbf{Value:} 
{\tt stochastic\_from\_dist('uninformative', logp= uninformative\texttt{...}}&\\
\cline{1-2}
\raggedright D\-i\-s\-c\-r\-e\-t\-e\-U\-n\-i\-n\-f\-o\-r\-m\-a\-t\-i\-v\-e\- & \raggedright \textbf{Value:} 
{\tt stochastic\_from\_dist('uninformative', logp= uninformative\texttt{...}}&\\
\cline{1-2}
\raggedright O\-n\-e\-O\-v\-e\-r\-X\- & \raggedright \textbf{Value:} 
{\tt stochastic\_from\_dist('one\_over\_x\_like', logp= one\_over\_x\_\texttt{...}}&\\
\cline{1-2}
\end{longtable}


%%%%%%%%%%%%%%%%%%%%%%%%%%%%%%%%%%%%%%%%%%%%%%%%%%%%%%%%%%%%%%%%%%%%%%%%%%%
%%                           Class Description                           %%
%%%%%%%%%%%%%%%%%%%%%%%%%%%%%%%%%%%%%%%%%%%%%%%%%%%%%%%%%%%%%%%%%%%%%%%%%%%

    \index{pymc \textit{(package)}!pymc.distributions \textit{(module)}!pymc.distributions.ArgumentError \textit{(class)}|(}
\subsection{Class ArgumentError}

    \label{pymc:distributions:ArgumentError}
\begin{tabular}{cccccccccccccc}
% Line for object, linespec=[False, False, False, False, False]
\multicolumn{2}{r}{\settowidth{\BCL}{object}\multirow{2}{\BCL}{object}}
&&
&&
&&
&&
&&
  \\\cline{3-3}
  &&\multicolumn{1}{c|}{}
&&
&&
&&
&&
&&
  \\
% Line for exceptions.BaseException, linespec=[False, False, False, False]
\multicolumn{4}{r}{\settowidth{\BCL}{exceptions.BaseException}\multirow{2}{\BCL}{exceptions.BaseException}}
&&
&&
&&
&&
  \\\cline{5-5}
  &&&&\multicolumn{1}{c|}{}
&&
&&
&&
&&
  \\
% Line for exceptions.Exception, linespec=[False, False, False]
\multicolumn{6}{r}{\settowidth{\BCL}{exceptions.Exception}\multirow{2}{\BCL}{exceptions.Exception}}
&&
&&
&&
  \\\cline{7-7}
  &&&&&&\multicolumn{1}{c|}{}
&&
&&
&&
  \\
% Line for exceptions.StandardError, linespec=[False, False]
\multicolumn{8}{r}{\settowidth{\BCL}{exceptions.StandardError}\multirow{2}{\BCL}{exceptions.StandardError}}
&&
&&
  \\\cline{9-9}
  &&&&&&&&\multicolumn{1}{c|}{}
&&
&&
  \\
% Line for exceptions.AttributeError, linespec=[False]
\multicolumn{10}{r}{\settowidth{\BCL}{exceptions.AttributeError}\multirow{2}{\BCL}{exceptions.AttributeError}}
&&
  \\\cline{11-11}
  &&&&&&&&&&\multicolumn{1}{c|}{}
&&
  \\
&&&&&&&&&&\multicolumn{2}{l}{\textbf{pymc.distributions.ArgumentError}}
\end{tabular}


Incorrect class argument

%%%%%%%%%%%%%%%%%%%%%%%%%%%%%%%%%%%%%%%%%%%%%%%%%%%%%%%%%%%%%%%%%%%%%%%%%%%
%%                                Methods                                %%
%%%%%%%%%%%%%%%%%%%%%%%%%%%%%%%%%%%%%%%%%%%%%%%%%%%%%%%%%%%%%%%%%%%%%%%%%%%

  \subsubsection{Methods}

    \vspace{0.5ex}

    \begin{boxedminipage}{\textwidth}

    \raggedright \textbf{\_\_delattr\_\_}(\textit{...})

    \vspace{-1.5ex}

    \rule{\textwidth}{0.5\fboxrule}
    x.\_\_delattr\_\_('name') {\textless}=={\textgreater} del x.name

    \vspace{1ex}

      Overrides: object.\_\_delattr\_\_

    \end{boxedminipage}

    \vspace{0.5ex}

    \begin{boxedminipage}{\textwidth}

    \raggedright \textbf{\_\_getattribute\_\_}(\textit{...})

    \vspace{-1.5ex}

    \rule{\textwidth}{0.5\fboxrule}
    x.\_\_getattribute\_\_('name') {\textless}=={\textgreater} x.name

    \vspace{1ex}

      Overrides: object.\_\_getattribute\_\_

    \end{boxedminipage}

    \label{exceptions:BaseException:__getitem__}
    \index{exceptions.BaseException.\_\_getitem\_\_ \textit{(function)}}

    \vspace{0.5ex}

    \begin{boxedminipage}{\textwidth}

    \raggedright \textbf{\_\_getitem\_\_}(\textit{x}, \textit{y})

    \vspace{-1.5ex}

    \rule{\textwidth}{0.5\fboxrule}
    x[y]

    \vspace{1ex}

    \end{boxedminipage}

    \label{exceptions:BaseException:__getslice__}
    \index{exceptions.BaseException.\_\_getslice\_\_ \textit{(function)}}

    \vspace{0.5ex}

    \begin{boxedminipage}{\textwidth}

    \raggedright \textbf{\_\_getslice\_\_}(\textit{x}, \textit{i}, \textit{j})

    \vspace{-1.5ex}

    \rule{\textwidth}{0.5\fboxrule}
    x[i:j]

    Use of negative indices is not supported.

    \vspace{1ex}

    \end{boxedminipage}

    \label{object:__hash__}
    \index{object.\_\_hash\_\_ \textit{(function)}}

    \vspace{0.5ex}

    \begin{boxedminipage}{\textwidth}

    \raggedright \textbf{\_\_hash\_\_}(\textit{x})

    \vspace{-1.5ex}

    \rule{\textwidth}{0.5\fboxrule}
    hash(x)

    \vspace{1ex}

    \end{boxedminipage}

    \vspace{0.5ex}

    \begin{boxedminipage}{\textwidth}

    \raggedright \textbf{\_\_init\_\_}(\textit{...})

    \vspace{-1.5ex}

    \rule{\textwidth}{0.5\fboxrule}
    x.\_\_init\_\_(...) initializes x; see x.\_\_class\_\_.\_\_doc\_\_ for 
    signature

    \vspace{1ex}

      Overrides: exceptions.StandardError.\_\_init\_\_

    \end{boxedminipage}

    \vspace{0.5ex}

    \begin{boxedminipage}{\textwidth}

    \raggedright \textbf{\_\_new\_\_}(\textit{T}, \textit{S}, \textit{...})

      \textbf{Return Value}
      \begin{quote}
\begin{alltt}
a new object with type S, a subtype of T
\end{alltt}

      \end{quote}

    \vspace{1ex}

      Overrides: exceptions.StandardError.\_\_new\_\_

    \end{boxedminipage}

    \vspace{0.5ex}

    \begin{boxedminipage}{\textwidth}

    \raggedright \textbf{\_\_reduce\_\_}(\textit{...})

    helper for pickle

    \vspace{1ex}

      Overrides: object.\_\_reduce\_\_ 	extit{(inherited documentation)}

    \end{boxedminipage}

    \label{object:__reduce_ex__}
    \index{object.\_\_reduce\_ex\_\_ \textit{(function)}}

    \vspace{0.5ex}

    \begin{boxedminipage}{\textwidth}

    \raggedright \textbf{\_\_reduce\_ex\_\_}(\textit{...})

    \vspace{-1.5ex}

    \rule{\textwidth}{0.5\fboxrule}
    helper for pickle

    \vspace{1ex}

    \end{boxedminipage}

    \vspace{0.5ex}

    \begin{boxedminipage}{\textwidth}

    \raggedright \textbf{\_\_repr\_\_}(\textit{x})

    \vspace{-1.5ex}

    \rule{\textwidth}{0.5\fboxrule}
    repr(x)

    \vspace{1ex}

      Overrides: object.\_\_repr\_\_

    \end{boxedminipage}

    \vspace{0.5ex}

    \begin{boxedminipage}{\textwidth}

    \raggedright \textbf{\_\_setattr\_\_}(\textit{...})

    \vspace{-1.5ex}

    \rule{\textwidth}{0.5\fboxrule}
    x.\_\_setattr\_\_('name', value) {\textless}=={\textgreater} x.name = 
    value

    \vspace{1ex}

      Overrides: object.\_\_setattr\_\_

    \end{boxedminipage}

    \label{exceptions:BaseException:__setstate__}
    \index{exceptions.BaseException.\_\_setstate\_\_ \textit{(function)}}

    \vspace{0.5ex}

    \begin{boxedminipage}{\textwidth}

    \raggedright \textbf{\_\_setstate\_\_}(\textit{...})

    \end{boxedminipage}

    \vspace{0.5ex}

    \begin{boxedminipage}{\textwidth}

    \raggedright \textbf{\_\_str\_\_}(\textit{x})

    \vspace{-1.5ex}

    \rule{\textwidth}{0.5\fboxrule}
    str(x)

    \vspace{1ex}

      Overrides: object.\_\_str\_\_

    \end{boxedminipage}


%%%%%%%%%%%%%%%%%%%%%%%%%%%%%%%%%%%%%%%%%%%%%%%%%%%%%%%%%%%%%%%%%%%%%%%%%%%
%%                              Properties                               %%
%%%%%%%%%%%%%%%%%%%%%%%%%%%%%%%%%%%%%%%%%%%%%%%%%%%%%%%%%%%%%%%%%%%%%%%%%%%

  \subsubsection{Properties}

\begin{longtable}{|p{.30\textwidth}|p{.62\textwidth}|l}
\cline{1-2}
\cline{1-2} \centering \textbf{Name} & \centering \textbf{Description}& \\
\cline{1-2}
\endhead\cline{1-2}\multicolumn{3}{r}{\small\textit{continued on next page}}\\\endfoot\cline{1-2}
\endlastfoot\raggedright \_\-\_\-c\-l\-a\-s\-s\-\_\-\_\- & \raggedright \textbf{Value:} 
{\tt {\textless}attribute '\_\_class\_\_' of 'object' objects{\textgreater}}&\\
\cline{1-2}
\raggedright a\-r\-g\-s\- & \raggedright \textbf{Value:} 
{\tt {\textless}attribute 'args' of 'exceptions.BaseException' objects{\textgreater}}&\\
\cline{1-2}
\raggedright m\-e\-s\-s\-a\-g\-e\- & \raggedright \textbf{Value:} 
{\tt {\textless}member 'message' of 'exceptions.BaseException' objects{\textgreater}}&\\
\cline{1-2}
\end{longtable}

    \index{pymc \textit{(package)}!pymc.distributions \textit{(module)}!pymc.distributions.ArgumentError \textit{(class)}|)}

%%%%%%%%%%%%%%%%%%%%%%%%%%%%%%%%%%%%%%%%%%%%%%%%%%%%%%%%%%%%%%%%%%%%%%%%%%%
%%                           Class Description                           %%
%%%%%%%%%%%%%%%%%%%%%%%%%%%%%%%%%%%%%%%%%%%%%%%%%%%%%%%%%%%%%%%%%%%%%%%%%%%

    \index{pymc \textit{(package)}!pymc.distributions \textit{(module)}!pymc.distributions.Categorical \textit{(class)}|(}
\subsection{Class Categorical}

    \label{pymc:distributions:Categorical}
\begin{tabular}{cccccccccccccc}
% Line for object, linespec=[False, False, False, False, False]
\multicolumn{2}{r}{\settowidth{\BCL}{object}\multirow{2}{\BCL}{object}}
&&
&&
&&
&&
&&
  \\\cline{3-3}
  &&\multicolumn{1}{c|}{}
&&
&&
&&
&&
&&
  \\
% Line for pymc.Node.Node, linespec=[False, False, False, False]
\multicolumn{4}{r}{\settowidth{\BCL}{pymc.Node.Node}\multirow{2}{\BCL}{pymc.Node.Node}}
&&
&&
&&
&&
  \\\cline{5-5}
  &&&&\multicolumn{1}{c|}{}
&&
&&
&&
&&
  \\
% Line for pymc.Node.Variable, linespec=[False, False, False]
\multicolumn{6}{r}{\settowidth{\BCL}{pymc.Node.Variable}\multirow{2}{\BCL}{pymc.Node.Variable}}
&&
&&
&&
  \\\cline{7-7}
  &&&&&&\multicolumn{1}{c|}{}
&&
&&
&&
  \\
% Line for pymc.Node.StochasticBase, linespec=[False, False]
\multicolumn{8}{r}{\settowidth{\BCL}{pymc.Node.StochasticBase}\multirow{2}{\BCL}{pymc.Node.StochasticBase}}
&&
&&
  \\\cline{9-9}
  &&&&&&&&\multicolumn{1}{c|}{}
&&
&&
  \\
% Line for pymc.PyMCObjects.Stochastic, linespec=[False]
\multicolumn{10}{r}{\settowidth{\BCL}{pymc.PyMCObjects.Stochastic}\multirow{2}{\BCL}{pymc.PyMCObjects.Stochastic}}
&&
  \\\cline{11-11}
  &&&&&&&&&&\multicolumn{1}{c|}{}
&&
  \\
&&&&&&&&&&\multicolumn{2}{l}{\textbf{pymc.distributions.Categorical}}
\end{tabular}

\begin{description}
\item[{C = Categorical(name, p, minval, step{[}, trace=True, value=None,}] \leavevmode 
rseed=False, isdata=False, cache{\_}depth=2, plot=True, verbose=0{]})

\end{description}

A categorical random variable. Parents are p, minval, step.

If parent p is Dirichlet and has length k-1, an implicit k'th
category is assumed to exist with associated probability 1-sum(p.value).

Otherwise parent p's value should sum to 1.

%%%%%%%%%%%%%%%%%%%%%%%%%%%%%%%%%%%%%%%%%%%%%%%%%%%%%%%%%%%%%%%%%%%%%%%%%%%
%%                                Methods                                %%
%%%%%%%%%%%%%%%%%%%%%%%%%%%%%%%%%%%%%%%%%%%%%%%%%%%%%%%%%%%%%%%%%%%%%%%%%%%

  \subsubsection{Methods}

    \vspace{0.5ex}

    \begin{boxedminipage}{\textwidth}

    \raggedright \textbf{\_\_init\_\_}(\textit{self}, \textit{name}, \textit{p}, \textit{minval}=\texttt{0}, \textit{step}=\texttt{1}, \textit{value}=\texttt{None}, \textit{isdata}=\texttt{False}, \textit{size}=\texttt{1}, \textit{trace}=\texttt{True}, \textit{rseed}=\texttt{False}, \textit{cache\_depth}=\texttt{2}, \textit{plot}=\texttt{True}, \textit{verbose}=\texttt{0})

    x.\_\_init\_\_(...) initializes x; see x.\_\_class\_\_.\_\_doc\_\_ for 
    signature

    \vspace{1ex}

      Overrides: pymc.PyMCObjects.Stochastic.\_\_init\_\_

    \end{boxedminipage}

    \label{object:__delattr__}
    \index{object.\_\_delattr\_\_ \textit{(function)}}

    \vspace{0.5ex}

    \begin{boxedminipage}{\textwidth}

    \raggedright \textbf{\_\_delattr\_\_}(\textit{...})

    \vspace{-1.5ex}

    \rule{\textwidth}{0.5\fboxrule}
    x.\_\_delattr\_\_('name') {\textless}=={\textgreater} del x.name

    \vspace{1ex}

    \end{boxedminipage}

    \label{object:__getattribute__}
    \index{object.\_\_getattribute\_\_ \textit{(function)}}

    \vspace{0.5ex}

    \begin{boxedminipage}{\textwidth}

    \raggedright \textbf{\_\_getattribute\_\_}(\textit{...})

    \vspace{-1.5ex}

    \rule{\textwidth}{0.5\fboxrule}
    x.\_\_getattribute\_\_('name') {\textless}=={\textgreater} x.name

    \vspace{1ex}

    \end{boxedminipage}

    \label{object:__hash__}
    \index{object.\_\_hash\_\_ \textit{(function)}}

    \vspace{0.5ex}

    \begin{boxedminipage}{\textwidth}

    \raggedright \textbf{\_\_hash\_\_}(\textit{x})

    \vspace{-1.5ex}

    \rule{\textwidth}{0.5\fboxrule}
    hash(x)

    \vspace{1ex}

    \end{boxedminipage}

    \label{object:__new__}
    \index{object.\_\_new\_\_ \textit{(function)}}

    \vspace{0.5ex}

    \begin{boxedminipage}{\textwidth}

    \raggedright \textbf{\_\_new\_\_}(\textit{T}, \textit{S}, \textit{...})

      \textbf{Return Value}
      \begin{quote}
\begin{alltt}
a new object with type S, a subtype of T
\end{alltt}

      \end{quote}

    \vspace{1ex}

    \end{boxedminipage}

    \label{object:__reduce__}
    \index{object.\_\_reduce\_\_ \textit{(function)}}

    \vspace{0.5ex}

    \begin{boxedminipage}{\textwidth}

    \raggedright \textbf{\_\_reduce\_\_}(\textit{...})

    \vspace{-1.5ex}

    \rule{\textwidth}{0.5\fboxrule}
    helper for pickle

    \vspace{1ex}

    \end{boxedminipage}

    \label{object:__reduce_ex__}
    \index{object.\_\_reduce\_ex\_\_ \textit{(function)}}

    \vspace{0.5ex}

    \begin{boxedminipage}{\textwidth}

    \raggedright \textbf{\_\_reduce\_ex\_\_}(\textit{...})

    \vspace{-1.5ex}

    \rule{\textwidth}{0.5\fboxrule}
    helper for pickle

    \vspace{1ex}

    \end{boxedminipage}

    \vspace{0.5ex}

    \begin{boxedminipage}{\textwidth}

    \raggedright \textbf{\_\_repr\_\_}(\textit{self})

    repr(x)

    \vspace{1ex}

      Overrides: object.\_\_repr\_\_ 	extit{(inherited documentation)}

    \end{boxedminipage}

    \label{object:__setattr__}
    \index{object.\_\_setattr\_\_ \textit{(function)}}

    \vspace{0.5ex}

    \begin{boxedminipage}{\textwidth}

    \raggedright \textbf{\_\_setattr\_\_}(\textit{...})

    \vspace{-1.5ex}

    \rule{\textwidth}{0.5\fboxrule}
    x.\_\_setattr\_\_('name', value) {\textless}=={\textgreater} x.name = 
    value

    \vspace{1ex}

    \end{boxedminipage}

    \vspace{0.5ex}

    \begin{boxedminipage}{\textwidth}

    \raggedright \textbf{\_\_str\_\_}(\textit{self})

    str(x)

    \vspace{1ex}

      Overrides: pymc.Node.Node.\_\_str\_\_

    \end{boxedminipage}

    \vspace{0.5ex}

    \begin{boxedminipage}{\textwidth}

    \raggedright \textbf{gen\_lazy\_function}(\textit{self})

    \vspace{-1.5ex}

    \rule{\textwidth}{0.5\fboxrule}

Will be called by Node at instantiation.
    \vspace{1ex}

      Overrides: pymc.Node.Node.gen\_lazy\_function

    \end{boxedminipage}

    \label{pymc:PyMCObjects:Stochastic:get_logp}
    \index{pymc.PyMCObjects.Stochastic.get\_logp \textit{(function)}}

    \vspace{0.5ex}

    \begin{boxedminipage}{\textwidth}

    \raggedright \textbf{get\_logp}(\textit{self})

    \end{boxedminipage}

    \label{pymc:PyMCObjects:Stochastic:get_value}
    \index{pymc.PyMCObjects.Stochastic.get\_value \textit{(function)}}

    \vspace{0.5ex}

    \begin{boxedminipage}{\textwidth}

    \raggedright \textbf{get\_value}(\textit{self})

    \end{boxedminipage}

    \label{pymc:PyMCObjects:Stochastic:random}
    \index{pymc.PyMCObjects.Stochastic.random \textit{(function)}}

    \vspace{0.5ex}

    \begin{boxedminipage}{\textwidth}

    \raggedright \textbf{rand}(\textit{self})

    \vspace{-1.5ex}

    \rule{\textwidth}{0.5\fboxrule}

Draws a new value for a stoch conditional on its parents
and returns it.

Raises an error if no 'random' argument was passed to {\_}{\_}init{\_}{\_}.
    \vspace{1ex}

    \end{boxedminipage}

    \label{pymc:PyMCObjects:Stochastic:random}
    \index{pymc.PyMCObjects.Stochastic.random \textit{(function)}}

    \vspace{0.5ex}

    \begin{boxedminipage}{\textwidth}

    \raggedright \textbf{random}(\textit{self})

    \vspace{-1.5ex}

    \rule{\textwidth}{0.5\fboxrule}

Draws a new value for a stoch conditional on its parents
and returns it.

Raises an error if no 'random' argument was passed to {\_}{\_}init{\_}{\_}.
    \vspace{1ex}

    \end{boxedminipage}

    \label{pymc:PyMCObjects:Stochastic:set_logp}
    \index{pymc.PyMCObjects.Stochastic.set\_logp \textit{(function)}}

    \vspace{0.5ex}

    \begin{boxedminipage}{\textwidth}

    \raggedright \textbf{set\_logp}(\textit{self})

    \end{boxedminipage}

    \label{pymc:PyMCObjects:Stochastic:set_value}
    \index{pymc.PyMCObjects.Stochastic.set\_value \textit{(function)}}

    \vspace{0.5ex}

    \begin{boxedminipage}{\textwidth}

    \raggedright \textbf{set\_value}(\textit{self}, \textit{value})

    \end{boxedminipage}

    \label{pymc:Node:Variable:stats}
    \index{pymc.Node.Variable.stats \textit{(function)}}

    \vspace{0.5ex}

    \begin{boxedminipage}{\textwidth}

    \raggedright \textbf{stats}(\textit{self}, \textit{alpha}=\texttt{0.05})

    \vspace{-1.5ex}

    \rule{\textwidth}{0.5\fboxrule}

Generate posterior statistics for node.
    \vspace{1ex}

    \end{boxedminipage}


%%%%%%%%%%%%%%%%%%%%%%%%%%%%%%%%%%%%%%%%%%%%%%%%%%%%%%%%%%%%%%%%%%%%%%%%%%%
%%                              Properties                               %%
%%%%%%%%%%%%%%%%%%%%%%%%%%%%%%%%%%%%%%%%%%%%%%%%%%%%%%%%%%%%%%%%%%%%%%%%%%%

  \subsubsection{Properties}

\begin{longtable}{|p{.30\textwidth}|p{.62\textwidth}|l}
\cline{1-2}
\cline{1-2} \centering \textbf{Name} & \centering \textbf{Description}& \\
\cline{1-2}
\endhead\cline{1-2}\multicolumn{3}{r}{\small\textit{continued on next page}}\\\endfoot\cline{1-2}
\endlastfoot\raggedright \_\-\_\-c\-l\-a\-s\-s\-\_\-\_\- & \raggedright \textbf{Value:} 
{\tt {\textless}attribute '\_\_class\_\_' of 'object' objects{\textgreater}}&\\
\cline{1-2}
\end{longtable}


%%%%%%%%%%%%%%%%%%%%%%%%%%%%%%%%%%%%%%%%%%%%%%%%%%%%%%%%%%%%%%%%%%%%%%%%%%%
%%                            Class Variables                            %%
%%%%%%%%%%%%%%%%%%%%%%%%%%%%%%%%%%%%%%%%%%%%%%%%%%%%%%%%%%%%%%%%%%%%%%%%%%%

  \subsubsection{Class Variables}

\begin{longtable}{|p{.30\textwidth}|p{.62\textwidth}|l}
\cline{1-2}
\cline{1-2} \centering \textbf{Name} & \centering \textbf{Description}& \\
\cline{1-2}
\endhead\cline{1-2}\multicolumn{3}{r}{\small\textit{continued on next page}}\\\endfoot\cline{1-2}
\endlastfoot\raggedright p\-a\-r\-e\-n\-t\-\_\-n\-a\-m\-e\-s\- & \raggedright \textbf{Value:} 
{\tt ['p', 'minval', 'step']}&\\
\cline{1-2}
\raggedright c\-o\-p\-a\-r\-e\-n\-t\-s\- & \raggedright \textbf{Value:} 
{\tt property(\_get\_coparents, doc= "All the variables whose ex\texttt{...}}&\\
\cline{1-2}
\raggedright e\-x\-t\-e\-n\-d\-e\-d\-\_\-c\-h\-i\-l\-d\-r\-e\-n\- & \raggedright \textbf{Value:} 
{\tt property(\_get\_extended\_children, doc= "All the stochastic\texttt{...}}&\\
\cline{1-2}
\raggedright e\-x\-t\-e\-n\-d\-e\-d\-\_\-p\-a\-r\-e\-n\-t\-s\- & \raggedright \textbf{Value:} 
{\tt property(\_get\_extended\_parents, doc= "All the stochastic \texttt{...}}&\\
\cline{1-2}
\raggedright l\-o\-g\-p\- & \raggedright \textbf{Value:} 
{\tt property(fget= get\_logp, fset= set\_logp, doc= "Log-probab\texttt{...}}&\\
\cline{1-2}
\raggedright m\-a\-r\-k\-o\-v\-\_\-b\-l\-a\-n\-k\-e\-t\- & \raggedright \textbf{Value:} 
{\tt property(\_get\_markov\_blanket, doc= "Self's coparents, sel\texttt{...}}&\\
\cline{1-2}
\raggedright m\-o\-r\-a\-l\-\_\-n\-e\-i\-g\-h\-b\-o\-r\-s\- & \raggedright \textbf{Value:} 
{\tt property(\_get\_moral\_neighbors, doc= "Self's neighbors in \texttt{...}}&\\
\cline{1-2}
\raggedright p\-a\-r\-e\-n\-t\-s\- & \raggedright \textbf{Value:} 
{\tt property(\_get\_parents, \_set\_parents, doc= "Self's parents\texttt{...}}&\\
\cline{1-2}
\raggedright p\-l\-o\-t\- & \raggedright \textbf{Value:} 
{\tt property(\_get\_plot, doc= 'A flag indicating whether self \texttt{...}}&\\
\cline{1-2}
\raggedright v\-a\-l\-u\-e\- & \raggedright \textbf{Value:} 
{\tt property(fget= get\_value, fset= set\_value, doc= "Self's c\texttt{...}}&\\
\cline{1-2}
\end{longtable}

    \index{pymc \textit{(package)}!pymc.distributions \textit{(module)}!pymc.distributions.Categorical \textit{(class)}|)}

%%%%%%%%%%%%%%%%%%%%%%%%%%%%%%%%%%%%%%%%%%%%%%%%%%%%%%%%%%%%%%%%%%%%%%%%%%%
%%                           Class Description                           %%
%%%%%%%%%%%%%%%%%%%%%%%%%%%%%%%%%%%%%%%%%%%%%%%%%%%%%%%%%%%%%%%%%%%%%%%%%%%

    \index{pymc \textit{(package)}!pymc.distributions \textit{(module)}!pymc.distributions.Multinomial \textit{(class)}|(}
\subsection{Class Multinomial}

    \label{pymc:distributions:Multinomial}
\begin{tabular}{cccccccccccccc}
% Line for object, linespec=[False, False, False, False, False]
\multicolumn{2}{r}{\settowidth{\BCL}{object}\multirow{2}{\BCL}{object}}
&&
&&
&&
&&
&&
  \\\cline{3-3}
  &&\multicolumn{1}{c|}{}
&&
&&
&&
&&
&&
  \\
% Line for pymc.Node.Node, linespec=[False, False, False, False]
\multicolumn{4}{r}{\settowidth{\BCL}{pymc.Node.Node}\multirow{2}{\BCL}{pymc.Node.Node}}
&&
&&
&&
&&
  \\\cline{5-5}
  &&&&\multicolumn{1}{c|}{}
&&
&&
&&
&&
  \\
% Line for pymc.Node.Variable, linespec=[False, False, False]
\multicolumn{6}{r}{\settowidth{\BCL}{pymc.Node.Variable}\multirow{2}{\BCL}{pymc.Node.Variable}}
&&
&&
&&
  \\\cline{7-7}
  &&&&&&\multicolumn{1}{c|}{}
&&
&&
&&
  \\
% Line for pymc.Node.StochasticBase, linespec=[False, False]
\multicolumn{8}{r}{\settowidth{\BCL}{pymc.Node.StochasticBase}\multirow{2}{\BCL}{pymc.Node.StochasticBase}}
&&
&&
  \\\cline{9-9}
  &&&&&&&&\multicolumn{1}{c|}{}
&&
&&
  \\
% Line for pymc.PyMCObjects.Stochastic, linespec=[False]
\multicolumn{10}{r}{\settowidth{\BCL}{pymc.PyMCObjects.Stochastic}\multirow{2}{\BCL}{pymc.PyMCObjects.Stochastic}}
&&
  \\\cline{11-11}
  &&&&&&&&&&\multicolumn{1}{c|}{}
&&
  \\
&&&&&&&&&&\multicolumn{2}{l}{\textbf{pymc.distributions.Multinomial}}
\end{tabular}

\begin{description}
\item[{M = Multinomial(name, n, p, trace=True, value=None,}] \leavevmode 
rseed=False, isdata=False, cache{\_}depth=2, plot=True, verbose=0{]})

\end{description}

A multinomial random variable. Parents are p, minval, step.

If parent p is Dirichlet and has length k-1, an implicit k'th
category is assumed to exist with associated probability 1-sum(p.value).

Otherwise parent p's value should sum to 1.

%%%%%%%%%%%%%%%%%%%%%%%%%%%%%%%%%%%%%%%%%%%%%%%%%%%%%%%%%%%%%%%%%%%%%%%%%%%
%%                                Methods                                %%
%%%%%%%%%%%%%%%%%%%%%%%%%%%%%%%%%%%%%%%%%%%%%%%%%%%%%%%%%%%%%%%%%%%%%%%%%%%

  \subsubsection{Methods}

    \vspace{0.5ex}

    \begin{boxedminipage}{\textwidth}

    \raggedright \textbf{\_\_init\_\_}(\textit{self}, \textit{name}, \textit{n}, \textit{p}, \textit{trace}=\texttt{True}, \textit{value}=\texttt{None}, \textit{rseed}=\texttt{False}, \textit{isdata}=\texttt{False}, \textit{cache\_depth}=\texttt{2}, \textit{plot}=\texttt{True}, \textit{verbose}=\texttt{0})

    x.\_\_init\_\_(...) initializes x; see x.\_\_class\_\_.\_\_doc\_\_ for 
    signature

    \vspace{1ex}

      Overrides: pymc.PyMCObjects.Stochastic.\_\_init\_\_

    \end{boxedminipage}

    \label{object:__delattr__}
    \index{object.\_\_delattr\_\_ \textit{(function)}}

    \vspace{0.5ex}

    \begin{boxedminipage}{\textwidth}

    \raggedright \textbf{\_\_delattr\_\_}(\textit{...})

    \vspace{-1.5ex}

    \rule{\textwidth}{0.5\fboxrule}
    x.\_\_delattr\_\_('name') {\textless}=={\textgreater} del x.name

    \vspace{1ex}

    \end{boxedminipage}

    \label{object:__getattribute__}
    \index{object.\_\_getattribute\_\_ \textit{(function)}}

    \vspace{0.5ex}

    \begin{boxedminipage}{\textwidth}

    \raggedright \textbf{\_\_getattribute\_\_}(\textit{...})

    \vspace{-1.5ex}

    \rule{\textwidth}{0.5\fboxrule}
    x.\_\_getattribute\_\_('name') {\textless}=={\textgreater} x.name

    \vspace{1ex}

    \end{boxedminipage}

    \label{object:__hash__}
    \index{object.\_\_hash\_\_ \textit{(function)}}

    \vspace{0.5ex}

    \begin{boxedminipage}{\textwidth}

    \raggedright \textbf{\_\_hash\_\_}(\textit{x})

    \vspace{-1.5ex}

    \rule{\textwidth}{0.5\fboxrule}
    hash(x)

    \vspace{1ex}

    \end{boxedminipage}

    \label{object:__new__}
    \index{object.\_\_new\_\_ \textit{(function)}}

    \vspace{0.5ex}

    \begin{boxedminipage}{\textwidth}

    \raggedright \textbf{\_\_new\_\_}(\textit{T}, \textit{S}, \textit{...})

      \textbf{Return Value}
      \begin{quote}
\begin{alltt}
a new object with type S, a subtype of T
\end{alltt}

      \end{quote}

    \vspace{1ex}

    \end{boxedminipage}

    \label{object:__reduce__}
    \index{object.\_\_reduce\_\_ \textit{(function)}}

    \vspace{0.5ex}

    \begin{boxedminipage}{\textwidth}

    \raggedright \textbf{\_\_reduce\_\_}(\textit{...})

    \vspace{-1.5ex}

    \rule{\textwidth}{0.5\fboxrule}
    helper for pickle

    \vspace{1ex}

    \end{boxedminipage}

    \label{object:__reduce_ex__}
    \index{object.\_\_reduce\_ex\_\_ \textit{(function)}}

    \vspace{0.5ex}

    \begin{boxedminipage}{\textwidth}

    \raggedright \textbf{\_\_reduce\_ex\_\_}(\textit{...})

    \vspace{-1.5ex}

    \rule{\textwidth}{0.5\fboxrule}
    helper for pickle

    \vspace{1ex}

    \end{boxedminipage}

    \vspace{0.5ex}

    \begin{boxedminipage}{\textwidth}

    \raggedright \textbf{\_\_repr\_\_}(\textit{self})

    repr(x)

    \vspace{1ex}

      Overrides: object.\_\_repr\_\_ 	extit{(inherited documentation)}

    \end{boxedminipage}

    \label{object:__setattr__}
    \index{object.\_\_setattr\_\_ \textit{(function)}}

    \vspace{0.5ex}

    \begin{boxedminipage}{\textwidth}

    \raggedright \textbf{\_\_setattr\_\_}(\textit{...})

    \vspace{-1.5ex}

    \rule{\textwidth}{0.5\fboxrule}
    x.\_\_setattr\_\_('name', value) {\textless}=={\textgreater} x.name = 
    value

    \vspace{1ex}

    \end{boxedminipage}

    \vspace{0.5ex}

    \begin{boxedminipage}{\textwidth}

    \raggedright \textbf{\_\_str\_\_}(\textit{self})

    str(x)

    \vspace{1ex}

      Overrides: pymc.Node.Node.\_\_str\_\_

    \end{boxedminipage}

    \vspace{0.5ex}

    \begin{boxedminipage}{\textwidth}

    \raggedright \textbf{gen\_lazy\_function}(\textit{self})

    \vspace{-1.5ex}

    \rule{\textwidth}{0.5\fboxrule}

Will be called by Node at instantiation.
    \vspace{1ex}

      Overrides: pymc.Node.Node.gen\_lazy\_function

    \end{boxedminipage}

    \label{pymc:PyMCObjects:Stochastic:get_logp}
    \index{pymc.PyMCObjects.Stochastic.get\_logp \textit{(function)}}

    \vspace{0.5ex}

    \begin{boxedminipage}{\textwidth}

    \raggedright \textbf{get\_logp}(\textit{self})

    \end{boxedminipage}

    \label{pymc:PyMCObjects:Stochastic:get_value}
    \index{pymc.PyMCObjects.Stochastic.get\_value \textit{(function)}}

    \vspace{0.5ex}

    \begin{boxedminipage}{\textwidth}

    \raggedright \textbf{get\_value}(\textit{self})

    \end{boxedminipage}

    \label{pymc:PyMCObjects:Stochastic:random}
    \index{pymc.PyMCObjects.Stochastic.random \textit{(function)}}

    \vspace{0.5ex}

    \begin{boxedminipage}{\textwidth}

    \raggedright \textbf{rand}(\textit{self})

    \vspace{-1.5ex}

    \rule{\textwidth}{0.5\fboxrule}

Draws a new value for a stoch conditional on its parents
and returns it.

Raises an error if no 'random' argument was passed to {\_}{\_}init{\_}{\_}.
    \vspace{1ex}

    \end{boxedminipage}

    \label{pymc:PyMCObjects:Stochastic:random}
    \index{pymc.PyMCObjects.Stochastic.random \textit{(function)}}

    \vspace{0.5ex}

    \begin{boxedminipage}{\textwidth}

    \raggedright \textbf{random}(\textit{self})

    \vspace{-1.5ex}

    \rule{\textwidth}{0.5\fboxrule}

Draws a new value for a stoch conditional on its parents
and returns it.

Raises an error if no 'random' argument was passed to {\_}{\_}init{\_}{\_}.
    \vspace{1ex}

    \end{boxedminipage}

    \label{pymc:PyMCObjects:Stochastic:set_logp}
    \index{pymc.PyMCObjects.Stochastic.set\_logp \textit{(function)}}

    \vspace{0.5ex}

    \begin{boxedminipage}{\textwidth}

    \raggedright \textbf{set\_logp}(\textit{self})

    \end{boxedminipage}

    \label{pymc:PyMCObjects:Stochastic:set_value}
    \index{pymc.PyMCObjects.Stochastic.set\_value \textit{(function)}}

    \vspace{0.5ex}

    \begin{boxedminipage}{\textwidth}

    \raggedright \textbf{set\_value}(\textit{self}, \textit{value})

    \end{boxedminipage}

    \label{pymc:Node:Variable:stats}
    \index{pymc.Node.Variable.stats \textit{(function)}}

    \vspace{0.5ex}

    \begin{boxedminipage}{\textwidth}

    \raggedright \textbf{stats}(\textit{self}, \textit{alpha}=\texttt{0.05})

    \vspace{-1.5ex}

    \rule{\textwidth}{0.5\fboxrule}

Generate posterior statistics for node.
    \vspace{1ex}

    \end{boxedminipage}


%%%%%%%%%%%%%%%%%%%%%%%%%%%%%%%%%%%%%%%%%%%%%%%%%%%%%%%%%%%%%%%%%%%%%%%%%%%
%%                              Properties                               %%
%%%%%%%%%%%%%%%%%%%%%%%%%%%%%%%%%%%%%%%%%%%%%%%%%%%%%%%%%%%%%%%%%%%%%%%%%%%

  \subsubsection{Properties}

\begin{longtable}{|p{.30\textwidth}|p{.62\textwidth}|l}
\cline{1-2}
\cline{1-2} \centering \textbf{Name} & \centering \textbf{Description}& \\
\cline{1-2}
\endhead\cline{1-2}\multicolumn{3}{r}{\small\textit{continued on next page}}\\\endfoot\cline{1-2}
\endlastfoot\raggedright \_\-\_\-c\-l\-a\-s\-s\-\_\-\_\- & \raggedright \textbf{Value:} 
{\tt {\textless}attribute '\_\_class\_\_' of 'object' objects{\textgreater}}&\\
\cline{1-2}
\end{longtable}


%%%%%%%%%%%%%%%%%%%%%%%%%%%%%%%%%%%%%%%%%%%%%%%%%%%%%%%%%%%%%%%%%%%%%%%%%%%
%%                            Class Variables                            %%
%%%%%%%%%%%%%%%%%%%%%%%%%%%%%%%%%%%%%%%%%%%%%%%%%%%%%%%%%%%%%%%%%%%%%%%%%%%

  \subsubsection{Class Variables}

\begin{longtable}{|p{.30\textwidth}|p{.62\textwidth}|l}
\cline{1-2}
\cline{1-2} \centering \textbf{Name} & \centering \textbf{Description}& \\
\cline{1-2}
\endhead\cline{1-2}\multicolumn{3}{r}{\small\textit{continued on next page}}\\\endfoot\cline{1-2}
\endlastfoot\raggedright p\-a\-r\-e\-n\-t\-\_\-n\-a\-m\-e\-s\- & \raggedright \textbf{Value:} 
{\tt ['n', 'p']}&\\
\cline{1-2}
\raggedright c\-o\-p\-a\-r\-e\-n\-t\-s\- & \raggedright \textbf{Value:} 
{\tt property(\_get\_coparents, doc= "All the variables whose ex\texttt{...}}&\\
\cline{1-2}
\raggedright e\-x\-t\-e\-n\-d\-e\-d\-\_\-c\-h\-i\-l\-d\-r\-e\-n\- & \raggedright \textbf{Value:} 
{\tt property(\_get\_extended\_children, doc= "All the stochastic\texttt{...}}&\\
\cline{1-2}
\raggedright e\-x\-t\-e\-n\-d\-e\-d\-\_\-p\-a\-r\-e\-n\-t\-s\- & \raggedright \textbf{Value:} 
{\tt property(\_get\_extended\_parents, doc= "All the stochastic \texttt{...}}&\\
\cline{1-2}
\raggedright l\-o\-g\-p\- & \raggedright \textbf{Value:} 
{\tt property(fget= get\_logp, fset= set\_logp, doc= "Log-probab\texttt{...}}&\\
\cline{1-2}
\raggedright m\-a\-r\-k\-o\-v\-\_\-b\-l\-a\-n\-k\-e\-t\- & \raggedright \textbf{Value:} 
{\tt property(\_get\_markov\_blanket, doc= "Self's coparents, sel\texttt{...}}&\\
\cline{1-2}
\raggedright m\-o\-r\-a\-l\-\_\-n\-e\-i\-g\-h\-b\-o\-r\-s\- & \raggedright \textbf{Value:} 
{\tt property(\_get\_moral\_neighbors, doc= "Self's neighbors in \texttt{...}}&\\
\cline{1-2}
\raggedright p\-a\-r\-e\-n\-t\-s\- & \raggedright \textbf{Value:} 
{\tt property(\_get\_parents, \_set\_parents, doc= "Self's parents\texttt{...}}&\\
\cline{1-2}
\raggedright p\-l\-o\-t\- & \raggedright \textbf{Value:} 
{\tt property(\_get\_plot, doc= 'A flag indicating whether self \texttt{...}}&\\
\cline{1-2}
\raggedright v\-a\-l\-u\-e\- & \raggedright \textbf{Value:} 
{\tt property(fget= get\_value, fset= set\_value, doc= "Self's c\texttt{...}}&\\
\cline{1-2}
\end{longtable}

    \index{pymc \textit{(package)}!pymc.distributions \textit{(module)}!pymc.distributions.Multinomial \textit{(class)}|)}
    \index{pymc \textit{(package)}!pymc.distributions \textit{(module)}|)}
