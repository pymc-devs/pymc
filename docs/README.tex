




%___________________________________________________________________________

\hypertarget{purpose}{}
\pdfbookmark[0]{Purpose}{purpose}
\section*{Purpose}

PyMC is a python module that implements the Metropolis-Hastings algorithm as a
python class. It is extremely flexible and applicable to a large suite of
problems. PyMC includes methods for summarizing output, plotting, goodness-of-
fit and convergence diagnostics.


%___________________________________________________________________________

\hypertarget{features}{}
\pdfbookmark[0]{Features}{features}
\section*{Features}
\begin{itemize}
\item {} 
Implements the Metropolis-Hastings algorithm so you can focus on your
application instead of on gory numerical algorithms.

\item {} 
Define your distribution from 24 well-documented statistical distributions,

\item {} 
Summarize your results in tables and plots.

\item {} 
Run convergence diagnostics.

\end{itemize}


%___________________________________________________________________________

\hypertarget{what-s-new-in-2-0}{}
\pdfbookmark[0]{What's new in 2.0}{what-s-new-in-2-0}
\section*{What's new in 2.0}
\begin{itemize}
\item {} 
Faster internal logic by coding the bottlenecks with Pyrex,

\item {} 
Faster distributions by an optimization of the Fortran functions,

\item {} 
Added a Joint Metropolis and a Gibbs sampler,

\item {} 
Define your problem in a separate file using the Node, Data and Parameter
classes. Use decorators to improve code readability.

\item {} 
Save your samples directly to a database. Select one from sqlite, MySQL, HDF5,
pickle files, text files or write a custom database backend from a template.

\item {} 
Run interactive convergence diagnostics,

\item {} 
Stop a sampling run in the middle, save it's state and restart the sampler
later,

\item {} 
Seed multiple chains on different processors.

\end{itemize}


%___________________________________________________________________________

\hypertarget{usage}{}
\pdfbookmark[0]{Usage}{usage}
\section*{Usage}

From a python shell, type:
\begin{quote}{\ttfamily \raggedright \noindent
import~PyMC~\\
S~=~PyMC.Sampler(problem{\_}definition,~db='pickle')~\\
S.sample(iter=10000,~burn=5000,~thin=2)
}\end{quote}

where problem{\_}definition is a module or a dictionary containing Node, Data and
Parameter instances defining your problem. Read the \href{docs/pdf/new_interface.pdf}{user guide} for a
complete description of the package, classes and some examples to get started.


%___________________________________________________________________________

\hypertarget{history}{}
\pdfbookmark[0]{History}{history}
\section*{History}

Chris ...


%___________________________________________________________________________

\