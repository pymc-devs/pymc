% TODO: Use Python's documentation tools?
%
%  PyMC User's Guide
%
%  Created by Chris Fonnesbeck on 2006-05-03.
%  Copyright (c) 2006 . All rights reserved.
%
\documentclass[]{manual}

% Use utf-8 encoding for foreign characters
%\usepackage[utf8]{inputenc}

% Setup for fullpage use
\usepackage{fullpage}
\usepackage{amsmath}
\usepackage{epsfig}

% Flexible citation syntax
\usepackage{natbib}
% Uncomment some of the following if you use the features
%

% Multipart figures
%\usepackage{subfigure}

% More symbols
\usepackage{amsmath}
\usepackage{amssymb}
\usepackage{latexsym}

% Surround parts of graphics with box
%\usepackage{boxedminipage}

% Enable hyeprlinks
\usepackage[pdfpagemode=FullScreen,colorlinks=true,linkcolor=red]{hyperref}

\usepackage{graphicx}

\title{PyMC 2.0 svn User's Guide \\
Installation and tutorial}
\author{ Christopher Fonnesbeck\\ David Huard \\ Anand Patil }
\graphicspath{{docs/}}
% \date
% TODO: Logo.

%%%%%%%%%%%%%%% Commands from rst2latex %%%%%%%%%%%%%%%%%%%%%%%%
\newcommand{\rubric}[1]{\subsection*{~\hfill {\it #1} \hfill ~}}
\newcommand{\titlereference}[1]{\textsl{#1}}
%%%%%%%%%%%%%%%%%%%%%%%%%%%%%%%%%%%%%%%%%%%%%%%%%%%%%%%%%%%%%%%%%


\begin{document}

\maketitle

\tableofcontents

\chapter{Introduction} %(fold)





%___________________________________________________________________________

\hypertarget{purpose}{}
\pdfbookmark[0]{Purpose}{purpose}
\section*{Purpose}
\label{purpose}

PyMC is a python module that implements the Metropolis-Hastings algorithm as a
python class. It is extremely flexible and applicable to a large suite of
problems. PyMC includes methods for summarizing output, plotting, goodness-of-
fit and convergence diagnostics.


%___________________________________________________________________________

\hypertarget{features}{}
\pdfbookmark[0]{Features}{features}
\section*{Features}
\label{features}
\begin{itemize}
\item {} 
Implements the Metropolis-Hastings algorithm so you can focus on your
application instead of on gory numerical algorithms.

\item {} 
Define your distribution from 24 well-documented statistical distributions,

\item {} 
Summarize your results in tables and plots.

\item {} 
Run convergence diagnostics.

\end{itemize}


%___________________________________________________________________________

\hypertarget{what-s-new-in-2-0}{}
\pdfbookmark[0]{What's new in 2.0}{what-s-new-in-2-0}
\section*{What's new in 2.0}
\label{what-s-new-in-2-0}
\begin{itemize}
\item {} 
Faster internal logic by coding the bottlenecks with Pyrex,

\item {} 
Faster distributions by an optimization of the Fortran functions,

\item {} 
Added a Joint Metropolis and a Gibbs sampler,

\item {} 
Define your problem in a separate file using the Node, Data and Parameter
classes. Use decorators to improve code readability.

\item {} 
Save your samples directly to a database. Select one from sqlite, MySQL, HDF5,
pickle files, text files or write a custom database backend from a template.

\item {} 
Run interactive convergence diagnostics,

\item {} 
Stop a sampling run in the middle, save it's state and restart the sampler
later,

\item {} 
Seed multiple chains on different processors.

\end{itemize}


%___________________________________________________________________________

\hypertarget{usage}{}
\pdfbookmark[0]{Usage}{usage}
\section*{Usage}
\label{usage}

From a python shell, type:
\begin{quote}{\ttfamily \raggedright \noindent
import~PyMC~\\
S~=~PyMC.Sampler(problem{\_}definition,~db='pickle')~\\
S.sample(iter=10000,~burn=5000,~thin=2)
}\end{quote}

where problem{\_}definition is a module or a dictionary containing Node, Data and
Parameter instances defining your problem. Read the \href{docs/pdf/new_interface.pdf}{user guide} for a
complete description of the package, classes and some examples to get started.


%___________________________________________________________________________

\hypertarget{history}{}
\pdfbookmark[0]{History}{history}
\section*{History}
\label{history}

PyMC began development in 2003, as an effort to generalize the process of building Metropolis-Hastimgs samplers, with an aim to making Markov chain Monte Carlo more accessible to non-statisticians (particularly ecologists). The choice to develop PyMC as a python module, rather than a standalone application, allowed the use MCMC methods in a larger modeling framework, in contrast to the BUGS environment. By 2005, PyMC was reliable enough for version 1.0 to be released to the public. A small group of regular users, most associated with the University of Georgia, provided much of the feedback necessary for the refinement of PyMC to its current state.

In 2006, David Huard and Anand Patil joined Chris Fonnesbeck on the development team for PyMC 2.0. This iteration of the software strives for more flexibility, better performance and a better end-user experience than any previous version of PyMC.


%___________________________________________________________________________

\

\chapter{Installation} %(fold)



PyMC is known to run on Mac OS X, Linux and Windows, but in theory should be
able to work on just about any platform for which Python, a Fortran compiler
and the NumPy module are  available. However, installing some extra
depencies can greatly improve PyMC's performance and versatility.
The following describes the required and optional dependencies and takes you
through the installation process.


%___________________________________________________________________________

\hypertarget{dependencies}{}
\pdfbookmark[0]{Dependencies}{dependencies}
\section*{Dependencies}
\label{dependencies}

PyMC requires some prerequisite packages to be present on the system.
Fortunately, there are currently only a few dependencies, and all are
freely available online.
\begin{itemize}
\item {} 
\href{http://www.python.org/.}{Python} version 2.5 or later.

\item {} 
\href{http://www.scipy.org/NumPy}{NumPy} (1.2 or newer): The fundamental scientific programming package, it provides a
multidimensional array type and many useful functions for numerical analysis.

\item {} 
\href{http://matplotlib.sourceforge.net/}{Matplotlib} (optional): 2D plotting library which produces publication
quality figures in a variety of image formats and interactive environments

\item {} 
\href{http://www.pytables.org/moin}{pyTables} (optional): Package for managing hierarchical datasets and
designed to efficiently and easily cope with extremely large amounts of data.
Requires the \href{http://www.hdfgroup.org/HDF5/}{HDF5} library.

\item {} 
\href{http://code.google.com/p/pydot/}{pydot} (optional): Python interface to Graphviz's Dot language, it allows
PyMC to create both directed and non-directed graphical representations of models.
Requires the \href{http://www.graphviz.org/}{Graphviz} library.

\item {} 
\href{http://www.scipy.org/}{SciPy} (optional): Library of algorithms for mathematics, science
and engineering.

\item {} 
\href{http://ipython.scipy.org/}{IPython} (optional): An enhanced interactive Python shell and an
architecture for interactive parallel computing.

\end{itemize}

There are prebuilt distributions that include all required dependencies. For
Mac OSX users, we recommend the \href{http://www.activestate.com/Products/ActivePython/}{MacPython} distribution, the
\href{http://www.enthought.com/products/epddownload.php}{Enthought Python Distribution}, or Python 2.5.1 that ships with
OSX 10.5 (Leopard). Windows users should download and install the
\href{http://www.enthought.com/products/epddownload.php}{Enthought Python Distribution}. The Enthought Python Distribution comes
bundled with these prerequisites. Note that depending on the currency of these
distributions, some packages may need to be updated manually.

If instead of installing the prebuilt binaries you prefer (or have) to build
\texttt{pymc} yourself, make sure you have a Fortran and a C compiler. There are free
compilers (gfortran, gcc) available on all platforms. Other compilers have not been
tested with PyMC but may work nonetheless.


%___________________________________________________________________________

\hypertarget{installation-using-easyinstall}{}
\pdfbookmark[0]{Installation using EasyInstall}{installation-using-easyinstall}
\section*{Installation using EasyInstall}
\label{installation-using-easyinstall}

The easiest way to install PyMC is to type in a terminal:
\begin{quote}{\ttfamily \raggedright \noindent
easy{\_}install~pymc
}\end{quote}

Provided \href{http://peak.telecommunity.com/DevCenter/EasyInstall}{EasyInstall} (part of the \href{http://peak.telecommunity.com/DevCenter/setuptools}{setuptools} module) is installed
and in your path, this should fetch and install the package from the
\href{http://pypi.python.org/pypi}{Python Package Index}. Make sure you have the appropriate administrative
privileges to install software on your computer.


%___________________________________________________________________________

\hypertarget{installing-from-pre-built-binaries}{}
\pdfbookmark[0]{Installing from pre-built binaries}{installing-from-pre-built-binaries}
\section*{Installing from pre-built binaries}
\label{installing-from-pre-built-binaries}

Pre-built binaries are available for Windows XP and Mac OS X. There are at least
two ways to install these:

1. Download the pre-built binary for your platform from \href{http://pypi.python.org/pypi/pymc/}{PyPI}, which installs them
automatically or,

2. Manually download packages (e.g. from the \href{pymc.googlecode.com}{PyMC site}) and double-click
the executable installation package, then follow the on-screen instructions.

For other platforms, you will need to build the package yourself from source.
Fortunately, this should be relatively straightforward.


%___________________________________________________________________________

\hypertarget{compiling-the-source-code}{}
\pdfbookmark[0]{Compiling the source code}{compiling-the-source-code}
\section*{Compiling the source code}
\label{compiling-the-source-code}

First download the source code tarball from \href{http://pypi.python.org/pypi/pymc/}{PyPI} and unpack it. Then move
into the unpacked directory and follow the platform specific instructions.


%___________________________________________________________________________

\hypertarget{windows}{}
\pdfbookmark[1]{Windows}{windows}
\subsection*{Windows}
\label{windows}

In a terminal, type:
\begin{quote}{\ttfamily \raggedright \noindent
python~setup.py~build~-{}-compiler=mingw32~install
}\end{quote}

This assumes you are using the GCC compiler (recommended). Otherwise,
change the -{}-compiler argument accordingly.


%___________________________________________________________________________

\hypertarget{mac-os-x-or-linux}{}
\pdfbookmark[1]{Mac OS X or Linux}{mac-os-x-or-linux}
\subsection*{Mac OS X or Linux}
\label{mac-os-x-or-linux}

In a terminal, type:
\begin{quote}{\ttfamily \raggedright \noindent
python~setup.py~build~\\
sudo~python~setup.py~install
}\end{quote}

The \titlereference{sudo} command is required to install PyMC into the Python \texttt{site-packages}
directory if it has restricted privileges. You will be prompted for a password,
and provided you have superuser privileges, the installation will proceed.


%___________________________________________________________________________

\hypertarget{development-version}{}
\pdfbookmark[0]{Development version}{development-version}
\section*{Development version}
\label{development-version}

You can check out the bleeding edge version of the code from the \href{http://subversion.tigris.org/}{subversion}
repository:
\begin{quote}{\ttfamily \raggedright \noindent
svn~checkout~http://pymc.googlecode.com/svn/trunk/~pymc
}\end{quote}

Previous versions are available in the \texttt{/tags} directory.


%___________________________________________________________________________

\hypertarget{running-the-test-suite}{}
\pdfbookmark[0]{Running the test suite}{running-the-test-suite}
\section*{Running the test suite}
\label{running-the-test-suite}

\texttt{pymc} comes with a set of tests that verify that the critical components
of the code work as expected. To run these tests, users must have \href{http://somethingaboutorange.com/mrl/projects/nose/}{nose}
installed on their system (this should not be a problem since nose is also
a NumPy dependency). The tests are launched from a python shell:
\begin{quote}{\ttfamily \raggedright \noindent
import~pymc~\\
pymc.test()
}\end{quote}

In case of failures, messages detailing the nature of these failures will
appear. In case this happens (it shouldn't), please report
the problems on the \href{http://code.google.com/p/pymc/issues/list.}{issue tracker}, specifying the version you are using and
the environment.


%___________________________________________________________________________

\hypertarget{bugs-and-feature-requests}{}
\pdfbookmark[0]{Bugs and feature requests}{bugs-and-feature-requests}
\section*{Bugs and feature requests}
\label{bugs-and-feature-requests}

Report problems with the installation, bugs in the code or feature request at
the \href{http://code.google.com/p/pymc/issues/list.}{issue tracker}. Comments and questions are welcome and should be
addressed to PyMC's \href{mailto:pymc-users@fisher.forestry.uga.edu}{mailing list}.



\bibliographystyle{plainnat}
\bibliography{pymc}

\end{document}
 
